\part{Introducción}

\chapter{Hamiltoniano del sólido}
\label{sec:ham}
En el estado sólido la escala de energías ronda (utilizando
$\bar E \sim K_B T$) las decenas de meV, en condiciones
razonables como $T \in (300,1000)K$ .

Para calcular el Hamiltoniano, analizamos las distintas interacciones
existentes: ion-ion, electrón-electrón e ion-electrón. Son todas
principalmente culombianas. Tenemos por tanto
$\mathcal{H} =
\mathcal{H}_{ion}+\mathcal{H}_{el}+\mathcal{H}_{el-ion}$.
\begin{align}
  \mathcal{H}_{ion} &= \sum_j \frac{P_j^2}{2M_j}+ \frac{1}{2} \sum_{j \neq j'} \frac{1}{4\pi\epsilon_0}\frac{Z_j Z_{j'}e^2}{\vert \mathbf{R}_j - \mathbf{R}_{j'} \vert} \\
  \mathcal{H}_{el} &= \sum_i \frac{p_i^2}{2m}+ \frac{1}{2} \sum_{i \neq i'} \frac{1}{4\pi\epsilon_0}\frac{e^2}{\vert \mathbf{r}_i - \mathbf{r}_{i'} \vert} \\
  \mathcal{H}_{ion} &= - \sum_{i,i'} \frac{1}{4\pi\epsilon_0}\frac{Z_je^2}{\vert \mathbf{r}_i - \mathbf{R}_{j} \vert}
\end{align}

El estado ordenado de nuestro sistema permite sustituir las posiciones
$\mathbf{R}_j$ de los iones por posiciones indexadas por enteros.
\begin{equation}
  \mathbf{R}_j \rightarrow \mathbf{R}_n =
  \underbrace{\mathbf{R}_n^0}_{\text{equilibrum}} +
  \underbrace{\mathbf{u}_n}_{\mathrlap{\text{small perturb.}}} 
\end{equation}
Los potenciales quedan como
\begin{equation}
  V(\mathbf{R}_j-\mathbf{R}_j') =   V(\mathbf{R}_n-\mathbf{R}_m') =   V(\mathbf{R}_n^0-\mathbf{R}_m^0) + \delta   V(\mathbf{R}_n-\mathbf{R}_m)
\end{equation}
y con el mismo procedimiento
\begin{equation}
  v(\mathbf{r}_i-\mathbf{R}_n)  =   v(\mathbf{r}_i^0-\mathbf{R}_n^0) + \delta   v(\mathbf{r}_i-\mathbf{R}_n)
\end{equation}
Con los nuevos potenciales, dejamos el Hamiltoniano como 
\begin{align}
  \label{eq:finalH}
  \mathcal{H} &= \sum_n \frac{P_n^2}{2M_n}+\frac{1}{2} \sum_{n \neq m}
                \delta   V(\mathbf{R}_n-\mathbf{R}_m) +  \tag{Lattice dinamics}  \\
              &+ \sum_i \frac{p_i^2}{2m}+\frac{1}{2} \sum_{i \neq i'} \frac{1}{4\pi\epsilon_0}\frac{e^2}{\vert \mathbf{r}_i - \mathbf{r}_{i'} \vert} + \sum_{i,n} v(\mathbf{r}_i-\mathbf{R}_n^0) + \tag{$e^-,V_{red};\ e^-,e^-$ } \\
              &+ \frac{1}{2} \sum_{n \neq m} V (\mathbf{R}_n^0 - \mathbf{R}_m^0) + \tag{$E_{cohesion}$}  \\
              &+ \sum_{i,n} \delta v (\mathbf{r}_i-\mathbf{R}_n) \tag{$e^-,\text{fonon}$}
\end{align}

Respectivamente, los términos corresponden a: 
\begin{itemize}
\item Dinámica de los iones
\item Electrones en un potencial periódico en la red, junto con la
  interacción electrón-electrón.
\item Cohesión del sólido
\item Interacción de los electrones con la red dinámica, también
  denotada interacción electrón-fonón
\end{itemize}

\chapter{Aproximación de Born-Oppenheimer}
\label{sec:bornopp}
Tratamos de resolver el sistema:
\begin{equation}
\label{eq:htot} 
 \begin{cases}
    &\mathcal{H} = \mathcal{H}_{ion} + \mathcal{H}_{el} + \mathcal{H}_{ion-el} \\
    &\mathcal{H}\Psi = E \Psi
  \end{cases}
\end{equation}
Hay muchísimas coordenadas, ya que
$\Psi = \Psi(\{\mathbf{r_i}\},\{\mathbf{R}_n\})$. Pero conocemos que
la masa del electrón es mucho menor que la del protón (por un factor
$10^4$), lo cual me permite suponer que la dinámica de los electrones
responde inmediatamente a la dinámica de los protones. En vista de
ello, realizamos un \emph{educated guess}:

\begin{equation}
  \begin{split}
    &\Psi \sim \psi(\{ \mathbf{r}_i \}, \{ \mathbf{R}_n\})\phi( \{ \mathbf{R}_n\}) \stackrel{\text{not.}}{\equiv} \psi(\{ \mathbf{r} \}, \{ \mathbf{R}\})\phi( \{ \mathbf{R}\})\\
    & (\mathcal{H}_{el}+\mathcal{H}_{el-ion})\psi(\mathbf{r},\mathbf{R})=\varepsilon _{el}(\mathbf{R})\psi(\mathbf{r},\mathbf{R})  
  \end{split}
\end{equation}

Volvemos al Hamiltoniano total de la ec. \ref{eq:htot}, y con la
hipótesis propuesta tratamos de simplificarlo:

\begin{equation}
  (\mathcal{H}_{ion}+\mathcal{H}_{el}+\mathcal{H}_{ion-el})\psi\phi = \mathcal{H}_{ion}[\psi\phi]+ 
  \underbrace{\varepsilon _{el} \psi}_{\mathclap{(\mathcal{H_{\text{el}} + \mathcal{H}_{\text{el-ion}}})\psi}}\cdot \phi=\cdots
\end{equation}
La ecuación sería muy simple si pudiese realizar lo siguiente:

\begin{equation}
\label{eq:ansatz}
  \cdots=\psi \underbrace{\mathcal{H}_{ion} \phi}_{E_{\text{ion}}\phi} + \varepsilon _{el} \psi \phi = \text E \psi \phi
\end{equation}
Lo que me daría la otra ecuación que necesito (autovalores de $\mathcal{H}_{ion})$:

\begin{equation}
  [\mathcal{H}_{ion}+\varepsilon _{el}(\mathbf{R})]\phi(\mathbf{R}) = \text E \phi(\mathbf{R})
\end{equation}
¿Es el ansatz de la ecuación \ref{eq:ansatz} razonable? La respuesta
es sí, en primer orden de perturbaciones. Comencemos por expandir la
parte problemat.ca:
\begin{equation}
  \begin{split}
    \mathcal{H}_{ion} \psi(\mathbf{r},\mathbf{R})\phi(\mathbf{R}) &= \left( \frac{-\hbar^2}{2M} \nabla^2_R + V _{ion} \right) \psi \phi = \\
    &= \frac{-\hbar^2}{2M}\nabla_R^2 (\psi\phi) + V _{ion} \psi \phi = \\
    &= \underbrace{\frac{-\hbar^2}{2M} \left( \nabla_R^2 \psi \right)
      \phi}_{\beta \Psi} +\frac{-\hbar^2}{2M} \psi \left( \nabla_R^2
      \phi \right)  + \\ & +
    \underbrace{\frac{-\hbar^2}{2M}2(\nabla_R  \psi)(\nabla_R
      \phi)}_{\alpha \Psi}+\psi V _{ion}\phi = \\
    &=\psi (\mathcal{H}_{ion}\phi) + \alpha \Psi + \beta \Psi
  \end{split}
\end{equation}

donde se ha usado
$\nabla^2 (XY) = (\nabla^2X)Y+Y(\nabla^2X)+2\cdot \nabla (X)\nabla
(Y)$.
La aproximación de Born-Oppenheimer será valida si $\alpha, \beta$ son
despreciables. Veamos su influencia en la energía en primer orden de
perturbaciones.

Para $\alpha$:
\begin{equation}
  \begin{split}
    \text E ^{(1)}_{\alpha} &= \iint \text{d}\mathbf{r} \ \text{d}\mathbf{R} \psi^* \phi^* \alpha =  \iint \text{d}\mathbf{r}\ \text{d}\mathbf{R}\psi^* \phi^* \left[\frac{-\hbar^2}{2M}2 (\nabla_R\psi)(\nabla_R\phi)\right] = \\
    &= \frac{-2\hbar^2}{2M} \int \text{d}\mathbf{R}\phi^* \nabla_R \phi  \frac{-2\hbar^2}{2M} \int \text{d}\mathbf{r} \psi^* \underbrace{\nabla_R}_{\neq f(\mathbf{r})} \psi \\
    &= -\frac{\hbar^2}{M} \int \text{d}\mathbf{R}\phi^* \nabla_R\phi
    \frac{1}{2}\nabla_R\underbrace{\int
      \text{d}\mathbf{r}\psi^*\psi}_{\mathclap{\text{electron number = cte.}}} \\ 
    &= -\frac{\hbar^2}{M} \int \text{d}\mathbf{R}\phi^* \nabla_R\phi \frac{1}{2} \ \cdot \  \cancelto{0}{\nabla_R \text{N}_{e^-}} \ \ \ = \\ &= 0
  \end{split}
\end{equation}

Para $\beta$:
\begin{equation}
  \begin{split}
    \text E ^{(1)}_{\beta} =\iint \text{d}\mathbf{r} \text{d}\mathbf{R} \psi^* \phi^* \beta &=
    \iint \text{d}\mathbf{r} \text{d}\mathbf{R} \psi^* \phi^* \left[
      \frac{-\hbar^2}{2M}(\nabla^2_R \psi)\phi \right] = \\
    =& \frac{-\hbar^2}{2M} \int \text{d}\mathbf{R}\phi^* \phi
    \underbrace{\int \text{d}\mathbf{r}\psi^* \nabla_R^2 \psi}_{1 ^{\dagger}} = \cdots
  \end{split}
\end{equation}
El término $1 ^{\dagger}$ es similar a la energía cinética de los
electrones, pero está en las coordenadas de los iones
($\mathbf{R}$). Sabemos que
$\psi =\psi(\{\mathbf{R}\},\{\mathbf{r}\})$, y que los electrones
están ligados a sus núcleos.
\begin{itemize}
\item En el caso de mínima dependencia, $\psi \sim \psi(\mathbf{r})$ y
  la derivada con las coordenadas iónicas anula la integral.
\item En el peor de los casos, la dependencia es máxima y los
  electrones están fuertemente ligados a sus núcleos. Tenemos por
  tanto $\psi \sim \psi(\mathbf{r}-\mathbf{R})$ y podemos sustituir
  $\nabla_R \psi$ por $\nabla_r \psi$.
\end{itemize}
Situémonos en el peor caso, $\psi \sim
\psi(\mathbf{r}-\mathbf{R})$. Multiplicando y dividiendo por $m$ :

\begin{equation}
  \cdots = \frac{m}{M} \int \text{d}\mathbf{R} \phi^* \phi \int  
  \frac{-\hbar^2}{2M} \text{d}\mathbf{r} \psi^* \nabla_r^2 \psi =
  \frac{m}{M} \langle T_e \rangle
 \end{equation}
 Como
 $\frac{m}{M} \langle T_e \rangle \sim 10 ^{-4} \langle T_e \rangle$,
 vemos que la corrección de este término es negligible.


%%% Local Variables:
%%% mode: latex
%%% TeX-master: "../fesi"
%%% End:
