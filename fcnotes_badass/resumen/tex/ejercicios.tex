\chapter{Perturbaciones en el pozo bidimensional (2.3)}
\begin{tcolorbox}[halign=left]
  \emph{Una partícula de masa $m$ se encuentra en un pozo
    bidimensional cuadrado entre ${x=\pm a}$, ${y=\pm a}$, de altura
    infinita.
    }

    \emph{
    Calcule los autovalores y autoestados del hamiltoniano.
}
\end{tcolorbox}
La solución del sistema sin perturbar es conocida; es el producto de
las soluciones de dos pozos 1D:
\begin{align}
  E_{n_x,n_y} &= \frac{\pi^2 \hbar^2}{8ma^2}(n_x^2+n_y^2) \\
  \varphi_{n_x,n_y}(x,y) &= \varphi_{n_x}(x)\varphi_{n_y}(y)
\end{align}
donde $m$ es la masa del electrón y $\varphi(x)$ es la solución para el
caso 1D de la función de ondas:
\begin{equation}
  \varphi(x) =
  \begin{cases}
   \sqrt{\frac{1}{a}} \cos \frac{n\pi x}{a} & n = 1,3,5\ldots \\
   \sqrt{\frac{1}{a}} \sin \frac{n\pi x}{a} & n = 2,4,6\ldots 
  \end{cases}
\end{equation}
El nivel fundamental es el $E_{1,1}$, cuyo autoestado correspondiente
es el $\varphi_{1,1}$, el cual no está degenerado. El primer excitado
$E_{12}=E_{21}$ sí que está degenerado, teniendo autoestados $\varphi_{1,2},\varphi_{2,1}$.
\begin{tcolorbox}[halign=left]
  \emph{Suponga una perturbación $V(x,y)=V_0$ en el cuadrado
    limitado por $x,y>0$ (y nula en el resto del pozo).
    Calcule en primer orden de perturbaciones la corrección a la
    energía del nivel fundamental y a la del primer nivel excitado.}
\end{tcolorbox}
\section{Perturbación del nivel fundamental}
No está degenerado, así que de manera directa:\jokemargin{Uno puede buscar la
solución en wikipedia y decir ``¿Quién habrá hecho esto? Un desalmado
que no tiene ni puta idea, seguro.'' o decir ``Ancha es Castilla'' y luego
se hunde la casa y mata a toda la familia.  Menos mal que soy físico teórico y
eso no me afecta.}
\begin{equation}
  \begin{split}
    \Delta E^{(1)} &= \bra{\varphi_{1,1}}V\ket{\varphi_{1,1}} = \iint \dd{x}
    \dd{y} \varphi_{1,1}^*(x,y) V(x,y) \varphi_{1,1}(x,y) = \\ 
    &= \int_0^a \int_0^a \dd{x}
    \dd{y} V_0 |\varphi_{1,1}(x,y)|^2 = \frac{V_0}{4}
  \end{split}
\end{equation}
donde se ha utilizado que la función de ondas está normalizada. 
\section{Perturbación del primer excitado}
Este nivel presenta degeneración. Utilizamos como base de
$E_{1,2}=E_{2,1}$ el subespacio de degeneración
$\{\varphi_{1,2},\varphi_{2,1}\}$, al que llamaremos $\{\ket{12},\ket{21}\}$.
La matriz de la ecuación de perturbaciones a primer orden será:
\begin{equation}
  \begin{pmatrix}
    \bra{12}V\ket{12} & \bra{12}V\ket{21} \\
    \bra{21}V\ket{12} & \bra{21}V\ket{21} 
  \end{pmatrix} = 
  \begin{pmatrix}
    \nicefrac{V_0}{4} & V_0 \left( \nicefrac{4}{3\pi} \right)^2 \\
      V_0 \left( \nicefrac{4}{3\pi} \right)^2 & \nicefrac{V_0}{4} 
  \end{pmatrix}
\end{equation}
Las integrales de la diagonal son inmediatas utilizando que
$\braket{12}{12}=\braket{21}{21}=1$. Fuera de la diagonal tenemos:
\begin{equation}
  \begin{split}
    \bra{\varphi_{1,2}}V\ket{\varphi_{2,1}} &= \iint \dd{x}
    \dd{y} \varphi_{1,2}^*(x,y) V(x,y) \varphi_{2,1}(x,y) = \\
    &= V_0\left(\int_0^a \frac{1}{a} \cos \frac{\pi \zeta}{a} \sin
      \frac{2\pi \zeta}{a} \dd{\zeta} \right)^2 \stackrel{\text{Wolfram}}{=} V_0 \left( \frac{4}{3\pi} \right)^2
  \end{split}
\end{equation}
La otra integral es idéntica. Calculamos los autovalores y
autovectores de la restricción de $V$ al subespacio de degeneración.
\begin{equation}
  \begin{vmatrix}
    \nicefrac{V_0}{4} - \lambda & V_0 \left( \nicefrac{4}{3\pi} \right)^2 \\
      V_0 \left( \nicefrac{4}{3\pi} \right)^2 & \nicefrac{V_0}{4} -\lambda
  \end{vmatrix} = 0 \ \rightarrow \ \lambda = V_0 \left[ \frac{1}{4}
    \pm \left( \frac{4}{3\pi} \right)^2 \right]
\end{equation}
Utilizando los dos autovalores $\lambda_+,\lambda_-$ hallamos los
autoestados correspondientes de la matriz, lo que nos da los autoestados
del hamiltoniano de cada nivel degenerado:
\begin{align}
  \ket{+} &= \frac{1}{\sqrt{2}}\binom{1}{1} \ \rightarrow \
  \frac{1}{\sqrt{2}} \left( \varphi_{1,2} + \varphi_{2,1}  \right)\\
  \ket{-} &= \frac{1}{\sqrt{2}}\binom{1}{-1} \ \rightarrow \
  \frac{1}{\sqrt{2}} \left( \varphi_{1,2} - \varphi_{2,1}  \right)
\end{align}
Como $\lambda_+ \neq \lambda_-$ (que son las correcciones a la
energía) no son iguales el nivel sufrirá un desdoblamiento ya en
primer orden de perturbaciones.


\chapter{Estructura fina del hidrógeno (2.4)}
\begin{tcolorbox}[halign=left]
  \emph{Calcule en primer orden de perturbaciones las correcciones de
    estructura fina a las energías del nivel fundamental y a las del primer
    nivel excitado del átomo de hidrógeno.}
\end{tcolorbox}
\section{Nivel fundamental}
El nivel fundamental, representado por los autoestados
$\ket{n\ \ell\ m\ \mu}=\ket{1\ 0\ 0\ \pm\oh}$, tiene degeneración
doble y las matrices que aparecen son por tanto $2\times2$.
\subsection{Término de masa}
La perturbación al hamiltoniano debida al término de masa es
\begin{equation}
  V_m = \frac{-p^4}{8m^3c^2}
  \label{eq:vm}
\end{equation}
Como el operador $V_m$ no actúa sobre el espacio de espines, tenemos
que $\bra{100\mu}V_m\ket{100\mu'}\propto \delta_{\mu\mu'}$, lo que implicará
una matriz diagonal en el subespacio de degeneración. Por tanto, basta
con calcular $\bra{100}V_m\ket{100}$.

Utilizando que $\frac{p^2}{2m} =  \Ham _0 - V$, podemos escribir
\begin{equation}
p^4 = 4m^2 ( \Ham _0^2 -  \Ham _0V - V  \Ham _0 +
V^2)
\label{eq:p4}
\end{equation}
Calculamos por separado las integrales:
\begin{itemize}
\item Para el hamiltoniano al cuadrado, tenemos simplemente
\begin{equation}
  \bra{100} \Ham _0^2 \ket{100} = E_0^2 \braket{100}{100} = E_0^2
\end{equation}
\item Los términos $ \Ham _0 V$ y $V  \Ham _0 $ se calculan
  igual:
  \begin{equation}
    \bra{100}V  \Ham _0 \ket{100} = 2E_0^2 \braket{100}{100} = 2E_0^2
  \end{equation}
donde se ha utilizado el teorema del virial ($V=2E$ para potenciales centrales).
\item El término de potencial cuadrático no es inmediato:
\begin{equation}
  \begin{split}
    \bra{100}V^2\ket{100} &= \iiint \dd{\Omega} \cancel{r^2} \dd{r}
    R_{10}^*(r){Y_0^0}^*(\Omega) \left( \frac{e^2}{4\pi\varepsilon_0}
    \right)^2 \frac{1}{\cancel{r^2}} R_{10}(r)Y_0^0(\Omega) = \\
    &= \left( \frac{e^2}{4\pi\varepsilon_0} \right)^2
    \underbrace{\int_{4\pi} \dd{\Omega}|Y_0^0(\Omega)|^2}_{=1}
    \int_0^\infty \dd{r} \frac{4}{a_0^{3}} e^{-2r/a_0} = \\
    &=2\left( \frac{e^2}{4\pi\varepsilon_0a_0} \right)^2 = 2m^2c^4\alpha^4
  \end{split}
\end{equation}
Se ha utilizado $\displaystyle R_{10} = \frac{2}{a_0^{3/2}}
\exp(-r/a_0)$ y $\displaystyle \int_0^\infty e^{-2r/\alpha} \dd{r} =
\frac{\alpha}{2}$, además de $a_0 = \frac{\hbar}{m c \alpha } =
\frac{4\pi\varepsilon_0 \hbar^2}{me^2}$.
\end{itemize}

Utilizamos que la energía del nivel fundamental del hidrógeno es $E_0
= -\frac{1}{2}mc^2\alpha^2$, obteniendo
\begin{equation}
  \bra{100}V_m\ket{100} \propto E_0^2 -2E_0^2 -2E_0^2 + 2m^2c^4\alpha^4 = \left( \frac{1}{4}-1+2 \right)m^2c^4\alpha^4
\end{equation}
donde se ha obviado que aún hay que multiplicar por $4m^2$ (ecuación
\eqref{eq:p4}) y dividir por $8m^3c^2$ (ecuación \eqref{eq:vm}). Tras
ello obtenemos:
\begin{equation}
  V_m =  \frac{5}{8}mc^2\alpha^4
  \begin{pmatrix}
\imat{2}
  \end{pmatrix}
\end{equation}

\subsection{Término de Darwin}
La perturbación provocada por el término de Darwin es
\begin{equation}
  V_D =
  \frac{\hbar^2}{8m^2c^2}\frac{e^2}{4\pi\varepsilon_0}4\pi\delta(\boldrm{r})
  \propto \delta(\boldrm{r})
\end{equation}
De nuevo, el operador sólo afecta a la parte espacial, siendo su
matriz diagonal. Calculamos el elemento de la diagonal:
\begin{equation}
  \begin{split}
    \bra{100}V_D\ket{100} &=
    \frac{h^2}{8m^2c^2}\frac{e^2}{\varepsilon_0}|\varphi_{100}(\boldrm{0})|^2
    =\\ & \stackrel{\text{tables}}{=}
    \frac{\hbar^2}{8m^2c^2}\frac{4}{a_0^3}\frac{1}{4\pi} = \frac{1}{2}mc^2\alpha^4
  \end{split}
\end{equation}
 Por tanto,
\begin{equation}
  V_D = \frac{1}{2}mc^2\alpha^4
  \begin{pmatrix}
\imat{2}
  \end{pmatrix}
\end{equation}

\subsection{Término de espín-órbita}
\label{subs:SOfirst}
En este caso el operador sí que actúa sobre el espacio de los espines,
por lo que la matriz no será diagonal.
\begin{equation}
  V_\text{S0} =
  \frac{1}{2m^2c^2}\frac{1}{r}\frac{e^2}{4\pi\varepsilon_0}\frac{1}{r^2}\boldrm{l}\cdot
  \boldrm{s} = \text{cte.}\cdot f(r) \cdot \boldrm{l}\cdot \boldrm{s}
\end{equation}
Escribimos\footnote{Puede demostrarse despejando de $\boldrm{J}^2 =
  (\boldrm{L}+\boldrm{S}^2)$, sabiendo que $\boldrm{L}$ y $\boldrm{S}$
  conmutan.} $\boldrm{l}\cdot \boldrm{s}$ como
$\frac{1}{2}(J^2-L^2-S^2)$ para facilitar las cuentas. Veamos como
actúa este término sobre los autoestados del sistema:
\begin{align}
  S^2 \ket{100\mu'} &= \frac{3}{4}\hbar^2 \ket{100\mu'}\\
  L^2 \ket{100\mu'} &\propto l(l+1)\hbar^2 = 0 \\
  J^2 \ket{100\mu'} &= \frac{3}{4}\hbar^2 \ket{100\mu'}
\end{align}
Para el último término se ha utilizado que $S^2$ no tenía un resultado
nulo pero $L^2$ sí, de forma que los posibles valores de $j$ son
sólamente $\pm s$ (en general van desde $l-s$ hasta $l+s$).

Por tanto,
\begin{equation}
  \frac{1}{2}(J^2-L^2-S^2) \ket{100\mu'} = 0
\end{equation}
No hay término de espín-órbita porque no hay órbita\footnote{Zas.}.

\subsection{Conclusiones}
Sumando todos los términos obtenemos una perturbacion
$V=-\frac{1}{8}mc^2\alpha^4 \mathbb{I}$. Esto nos provoca un
corrimiento del nivel fundamental de aproximadamente
$\SI{1.8e-4}{\eV}$. Es pequeño en comparación con la distancia
energética al siguiente nivel, $\SI{10}{\eV}$.



\section{Primer excitado}
La degeneración en este caso es 8, ya que $\ell\in\{0,1\}$ y por tanto $m$
puede ser o $0$ o $0,\pm 1$ y el espín puede ser $\mu = \pm
\oh$.
Hay muchos elementos de matriz, así que hay que buscar una base en que
las matrices sean lo más digonales posible. Para ello nos apoyamos en
el siguiente teorema:
\begin{thm}
  Si $[A,B]=0$ y $
  \begin{cases}
    A \boldrm{v}_1 &= \alpha_1 \boldrm{v}_1 \\
    A \boldrm{v}_2 &= \alpha_2 \boldrm{v}_2 
  \end{cases}
$, donde $\alpha_1 \neq \alpha_2$ y $A=A^\dagger$ es hermítico\jokenote{A los matemáticos les importan una higa los operadores hermíticos, pero nosotros tenemos la costumbre de medir. Quizá seríamos más felices si no midiéramos. La ignorancia hace feliz a la gente. O no... No.} entonces
\begin{equation}
  \bra{\boldrm{v}_1}B \ket{\boldrm{v}_2} = 0
\end{equation}
\label{thm:conmutediagonal}
\end{thm}
\begin{proof}
  Utilizando que $A$ es hermítico y conmuta con $B$,
  \begin{align}
    \matrixel{\boldrm{v}_1}{AB}{\boldrm{v}_2} &= \alpha_1
\matrixel{\boldrm{v}_1}{B}{\boldrm{v}_2} \\
    \matrixel{\boldrm{v}_1}{AB}{\boldrm{v}_2} &= \alpha_2
\matrixel{\boldrm{v}_1}{B}{\boldrm{v}_2} 
  \end{align}
  Por lo tanto
  \begin{equation}
    \alpha_1 \matrixel{\boldrm{v}_1}{B}{\boldrm{v}_2} = \alpha_2 \matrixel{\boldrm{v}_1}{B}{\boldrm{v_2}}
  \end{equation}
  Supuesto que $\alpha_1 \neq \alpha_2$, necesariamente
  \begin{equation}
    \matrixel{\boldrm{v}_1}{B}{\boldrm{v}_2} = 0
  \end{equation}

\end{proof}
Por lo que $B$ es diagonal por cajas en la base de autovectores de $A$,
si conmutan:
\begin{equation}
  B = 
\begin{tikzpicture}[baseline=(current bounding box.center)]%
\matrix[matrix of math nodes,
inner sep=0,
nodes={draw,outer sep=0,inner sep=2pt},
every left delimiter/.style={xshift=1ex},%tighter delimiter spacing
every right delimiter/.style={xshift=-1ex},
left delimiter={(},right delimiter={)},
column sep=-\pgflinewidth,row sep=-\pgflinewidth] (r) {
|[inner sep=3mm]|d_1&&\\
 & |[inner sep=5mm]|d_2&\\
&& |[inner sep=2mm]| \ddots\\
};
\end{tikzpicture}
\end{equation}
El tamaño de las cajas es $d_i \times d_i$, donde las $d_i$ son las degeneraciones
de cada autoestado.

Una posible base de $ \Ham $ con $n=2$ sería $
\{\ket{2\ 0\ 0 \ \mu},\ket{2 \ 1 \ m \ \mu}\}$ con $\mu=\pm
\oh$ y $m=0,\pm 1$; posee 8 elementos del tipo\footnote{El
CSCO es $ \Ham , S^2, L^2, L_z , S_z$, pero se omite en los kets
el término de $S^2$ por ser todo el rato igual. En el helio sí será
necesario escribir explícitamente este término.}
$\ket{ \Ham \ L^2\ L_z \ S_z}$. Recordando el teorema
\ref{thm:conmutediagonal}, comprobamos si los términos perturbativos
conmutan con $L^2,L_z,S_z$. No es necesario comprobarlo con
$ \Ham $ ya que el autovalor sabemos que será el mismo, siendo
inaplicable el teorema.

\subsection{Término de masa}
Sabemos que $V_m \sim p^4$. Extendemos al espacio de espinores para
comprobar si conmuta con $S_z$:
\begin{equation}
  \begin{split}
    [V_m,S_z] &=V_m S_z - S_z V_m = \\
    &= (V_m \otimes \mathbb{I})(\mathbb{I}\otimes S_z) -
    (\mathbb{I}\otimes S_z)(V_m \otimes \mathbb{I})
  \end{split}
\end{equation}
Si aplicamos ambos términos de la resta a $[\varphi \otimes \chi]$
obtenemos el mismo resultado, $V_m\varphi\otimes S_z \chi$. Por tanto,
$V_m$ conmuta con $S_z$, como era de esperar. Con esto garantizamos
que, al menos, $V_m$ tendrá dos cajas:

\begin{equation}
  V_m = 
\begin{tikzpicture}[
    node distance=1mm and 0mm,
    baseline]
\matrix (M1) [matrix of nodes,{left delimiter=(},{right delimiter=)}]
{
  $\mu = \oh $& 0  \\
  0 & $\mu = \nicefrac{-1}{2}$  \\
};
\draw  (M1-1-1.north east) -- (M1-2-2.south west);
\draw  (M1-1-1.south west) -- (M1-2-2.north east);
\end{tikzpicture}
\end{equation}
Para el operador $L_z$ reescribimos el término de masa con ayuda de
\begin{equation}
  T = \frac{p^2}{2m}= \frac{-\hbar^2}{2m} \underbrace{\left[
      \frac{1}{r^2} \pdv{r} \left( r^2 \pdv{r} \right) -
      \frac{L^2}{\hbar r^2} \right]}_{\nabla^2}
\end{equation}
Vemos que por tanto $p$ se puede escribir en función de $r,L$. Como
$[L^2,\boldrm{L}]=0$, $[\boldrm{L},T]=0$. Por lo tanto,
$[\boldrm{L},p^2]=0$ y por tanto $[L_z,p^2]=0$. Para $p^4$ obtenemos
los mismos resultados, deduciendo que $V_m$ conmuta con $L^2,L_z,S_z$
y es completamente diagonal en la base elegida para $ \Ham $:
\begin{equation}
  \matrixel{2\ \ell \ m\ \mu}{ V_m}{2\ \ell' \ m'\ \mu'} \sim \delta_{\ell \ell'} \delta_{mm'}\delta_{\mu\mu'}
\end{equation}

\subsection{Término de Darwin}
\label{subs:darwintermex}
El término es básicamente $V_D \sim \delta(\boldrm{r})$. Naturalmente
conmuta con los operadores en el espacio de espines, y con
$\boldrm{L}$ por ser puramente radial. Tenemos que, al igual que
$V_m$, el término será completamente diagonal:
\begin{equation}
  \matrixel{2\ \ell \ m\ \mu}{ V_D}{2\ \ell' \ m'\ \mu'} \sim \delta_{\ell \ell'} \delta_{mm'}\delta_{\mu\mu'}
\end{equation}

\subsection{Término de espín-órbita}
Aquí ya no hay tanta suerte. $\boldrm{l} \cdot \boldrm{s}$ no varía si
giro a la vez $\boldrm{l}$ y $\boldrm{s}$, pero sí lo hace al girarlos
por separado; el término conmuta con $\boldrm{J}$.

Escribiendo el término como $V_\text{SO}=
\frac{1}{2}f(r)[J^2-L^2-S^2]$ vemos que conmuta con $L^2,S^2$ pero no
con $L_z$. Nos aseguramos de esta última afirmación:
\begin{equation}
  \begin{split}
    [V_\text{S0},L_z] &\sim [L_xS_x,L_z] + [L_yS_y,L_z] +
    \cancelto{0}{[L_zS_z,L_z]} = \\ &= L_xS_xL_z -L_zL_xS_x+L_yS_yL_z
    -L_zL_yS_y = \\ &= [L_x,L_z]S_x + [L_y,L_z]S_y = -i \hbar L_yS_x+i
    \hbar L_xS_y \neq 0
  \end{split}
\end{equation}
donde se ha utilizado que $L_i,S_i$ pertenecen a espacios distintos y
por tanto conmutan. Visto que conmuta con $\boldrm{J}$ (y por tanto
con $J^2,J_z$), nos
preguntamos si sería mejor escoger una base\jokenote{La base astuta.} en
$ \Ham ,S^2,L^2,J^2,J_z$ en lugar de la
utilizada hasta ahora,
$ \Ham , S^2, L^2, L_z , S_z$. Comprobamos si se satisfacen las
otras dos relaciones de conmutación que nos interesan; empezando por
$[V_\text{S0},L^2] = [J^2-L^2-S^2,L^2]=0$ obtenemos
\begin{align}
    [L^2,\boldrm{L}]= [L^2,\boldrm{S}] = 0 \rightarrow
    [L^2,\underbrace{\boldrm{L}+\boldrm{S}}_{\boldrm{J}}] = 0
    \rightarrow [J^2,L^2]&=0\\
  [L^2,L^2] &= 0 \\
  [S^2,L^2]&=0
\end{align}
Para $[V_\text{SO},S^2]$ obtenemos los mismos resultados. Por lo
tanto, en la base propuesta la matriz resultante es diagonal para
$V_\text{SO}$. Es fácil ver que los demás operadores también serán
diagonales en dicha base, y obtenemos:
\begin{equation}
  \matrixel{2\ \ell\ j\ m}{V}{2 \ \ell' \ j'  \ m'} = \delta_{\ell\ell'}\delta_{jj'}\delta_{mm'}
\end{equation}
Ahora que tenemos una buena base, podemos calcular los valores numéricos de
los elementos de matriz.
\subsection{Cálculo de $V_m$}
\paragraph{$\ell=0$}
\label{paragraph:l0vm}


Para $\ell=0$ tenemos, dada su independencia del espín,
\begin{equation}
  \ev{V_m}{2\ 0 \ \oh\ \oh} = \ev{V_m}{2\ 0 \ 0} \cdot \cancelto{1}{\ip{\oh\ \oh}}
\end{equation}
Donde se ha cambiado de la nueva base $ \Ham ,S^2,L^2,J^2,J_z$ a
la antigua $ \Ham ,S^2,L^2,L_z,S_z$  mediante
$\ket*{\underset{n}{2} \ \underset{\ell}{0} \ \underset{j}{\oh}\  \underset{m}{\oh}} = 1 \cdot
\ket*{\underset{n}{2} \ \underset{\ell}{0}\ 
  \underset{m}{0}}\otimes\ket*{\underset{s}{\oh} \
  \underset{\mu}{\oh}}$. Notar que ambas $m$ no se refieren al mismo
número cuántico; una es el número cuántico magnético
$m\in\{-\ell,\cdots,0,\cdots,\ell\}$ y la otra el autovalor correspondiente
a $J_z$.
Escribimos $V_m$ como $\frac{-1}{2mc^2} ( \Ham ^2 -  \Ham V
- V  \Ham  + V^2)$, y calculamos por separado las integrales:
\begin{itemize}
\item En primer lugar, $\ev{ \Ham ^2}{2\ 0\ 0} = E_2^2$
\item Por el teorema del virial, 
  \begin{equation}
    \begin{split}
      \ev{ \Ham V}{2\ 0 \ 0} &= \ev{V \Ham }{2\ 0 \ 0} =\\&=
      E_2\ev{V}{2\ 0 \ 0} = E_2(-2E_2)
    \end{split}
  \end{equation}
\item Por último, hallamos $\ev{V^2}{2\ 0\ 0}$ por integración:
\begin{equation}
  \begin{split}
    \bra{200}V^2\ket{200} &= \iiint \dd{\Omega} \cancel{r^2} \dd{r}
    R_{20}^*(r){Y_0^0}^*(\Omega) \left( \frac{e^2}{4\pi\varepsilon_0}
    \right)^2 \frac{1}{\cancel{r^2}} R_{20}(r)Y_0^0(\Omega) = \\
    &= \left( \frac{e^2}{4\pi\varepsilon_0} \right)^2
    \underbrace{\int_{4\pi} \dd{\Omega}|Y_0^0(\Omega)|^2}_{=1}
    \int_0^\infty \dd{r} 4 \frac{1}{8a_0^3} \left( 1- \frac{r}{2a_0}
    \right)^2 e^{-r/a_0} = \\
    &= \cdots =  \frac{-13}{128} m c^2 \alpha^4
  \end{split}
\end{equation}

\end{itemize}

Con esto ya tenemos toda la caja de $\ell=0$, ya que el espín no es
relevante en este término. 
\paragraph{$\ell=1$}

Para la caja $\ell=1$ calculamos un
término cualquiera, en este caso utilizamos $\ev{V_m}{2\ 1\ \nicefrac{1}{2}\
  \nicefrac{1}{2}}$. Con una tabla de coeficientes de
Clebsch-Gordan\footnote{Tenemos dos momentos angulares, el espín
  ($\nicefrac{1}{2}$) y el orbital ($1$).}
vemos que:

\begin{equation}
  1 \times \frac{1}{2}:\ \ \ \ket*{\underset{J}{\oh}\
    \underset{M}{\oh}}  = \sqrt{\frac{2}{3}} \ket*{\underset{m_1}{+1}\
    \underset{m_2}{-\oh}} -\sqrt{\frac{1}{3}} \ket*{\underset{m_1}{0}\ \underset{m_2}{\oh}}
\end{equation}
Por tanto,
\begin{equation}
  \ket*{2\  1\  \oh\   \oh} = -\sqrt{\frac{1}{3}} \ket*{\underset{n}{2}\
  \underset{\ell}{1}\ \underset{m}{0}}\ket*{\underset{s}{\oh}\
  \underset{\mu}{\oh}}+\sqrt{\frac{2}{3}} \ket*{\underset{n}{2}\
  \underset{\ell}{1}\ \underset{m}{1}}\ket*{\underset{s}{\oh}\
  \underset{\mu}{-\oh}}
\end{equation}
El valor medio queda:
\begin{equation}
  \begin{split}
    \ev{V_m}{2\ 1\ \nicefrac{1}{2}\ \nicefrac{1}{2}} =& \frac{1}{3}
    \ev{V_m}{2 \ 1\ 0}\cancelto{1}{\braket{\oh}{\oh}} + \\ &+ \frac{2}{3}
    \ev{V_m}{2 \ 1\ 1}\cancelto{1}{\braket{-\oh}{-\oh}} - \\ &-
    \sqrt{\frac{1}{3}}\sqrt{\frac{2}{3}} \matrixel{2\ 1\ 0}{V_m}{2\ 1\
    1} \cancelto{0}{\braket{\oh}{-\oh}} - \\ &-
    \sqrt{\frac{2}{3}}\sqrt{\frac{1}{3}} \matrixel{2\ 1\ 1}{V_m}{2\ 1\
    0} \cancelto{0}{\braket{-\oh}{+\oh}}
  \end{split}
\end{equation}
Tenemos a priori dos integrales a resolver, pero podemos poner la
segunda en función de la primera: 
\begin{equation}
  \begin{split}
    &\ev{V_m}{\alpha\ j\ m+1} = 
\\ &=\left(
      \frac{1}{\hbar\sqrt{j(j+1)-m(m+1)}} \right)^2
    \ev{J_-VJ_+}{\alpha \ j \ m}
\\ &=\left(
      \frac{\hbar\sqrt{j(j+1)-m(m+1)}}{\hbar\sqrt{j(j+1)-m(m+1)}} \right)^2
    \ev{V}{\alpha \ j \ m} =
    \\ & = 
    \ev{V_m}{\alpha\ j\ m}
  \end{split}
  \label{eq:exercises_recurrence}
\end{equation}

Se ha utilizado que $J_- = J_+^\dagger$, y por tanto necesitamos $J_-$
para los bra y $J_+$ para los ket. Vemos que ambas
integrales tienen el mismo resultado.
Esto era de esperar, ya que $[V_m,\boldrm{S}]=[V_m,\boldrm{L}]=0$
implica qe $[V_m,\boldrm{J}]=0$, y por tanto de $\ket{n\ l\ j\ m}$ son
irrelevantes tanto $j$ como $m$. 

Integramos uno de los términos:

\begin{equation}
  \ev{V_m}{2\ 1\ 1} = (E_2^2 -2E_2^2 - 2E_2^2 + \langle  V^2 \rangle) \left( \frac{-1}{2mc^2} \right)
\end{equation}
\begin{equation}
  \begin{split}
    \langle V^2 \rangle &= \left( \frac{e^2}{4\pi\varepsilon_0} \right)^2 
    \iint \dd{\Omega}r^2 \dd{r}|R_{21}(r)|^2 \frac{1}{r^2} |Y_1^1|^2 =
    \\ &=
    \cdots = \left( \frac{e^2}{4\pi\varepsilon_0} \right)^2 \frac{1}{12a_0^2}
  \end{split}
\end{equation}
Por tanto,
\begin{equation}
  \ev{V_m}{2\ 1\ 1} = \frac{-7}{384}mc^2 \alpha^4 = \ev{V_m}{2\ 1 \ 0}
\end{equation}


\paragraph{Resultado}
Juntando los resultados anteriores, obtenemos
\begin{equation}
  V_m = mc^2 \alpha^4
\begin{tikzpicture}[
    node distance=1mm and 0mm,
    baseline]
\matrix (M1) [matrix of nodes,{left delimiter=(},{right delimiter=)}]
{
  $-\frac{13}{128}$ & $ $ & $ $ & $ $ &  $ $ & $ $ & $ $ & $ $ \\
  $ $ & $-\frac{13}{128}$ & $ $ & $ $ &  $ $ & $ $ & $ $ & $ $ \\
  $ $ & $ $ & $-\frac{7}{384}$ & $ $ &  $ $ & $ $ & $ $ & $ $ \\
  $ $ & $ $ & $ $ & $-\frac{7}{384}$ &  $ $ & $ $ & $ $ & $ $ \\
  $ $ & $ $ & $ $ & $ $ &  $-\frac{7}{384}$ & $ $ & $ $ & $ $ \\
  $ $ & $ $ & $ $ & $ $ &  $ $ & $-\frac{7}{384}$ & $ $ & $ $ \\
  $ $ & $ $ & $ $ & $ $ &  $ $ & $ $ & $-\frac{7}{384}$ & $ $ \\
  $ $ & $ $ & $ $ & $ $ &  $ $ & $ $ & $ $ & $-\frac{7}{384}$ \\
};
\draw[red,thick] 
        (M1-1-1.north west) -| (M1-2-2.south east) -| (M1-1-1.north
        west) node[above right=0.2cm] {$\ell=0,j=\oh$};
\draw[red,thick] 
        (M1-3-3.north west) -| (M1-4-4.south east) -| (M1-3-3.north
        west) node[above right=0.2cm] {$\ell=1,j=\oh$};
\draw[red,thick] 
        (M1-5-5.north west) -| (M1-8-8.south east) -| (M1-5-5.north
        west)  node[above right=0.2cm] {$\ell=1,j=\nicefrac{3}{2}$};
\end{tikzpicture}
\end{equation}

\subsection{Cálculo de $V_D$}
De nuevo, se desacopla la parte de espín. Utilizando que $V_D\sim\delta(\boldrm{r})$,
\begin{equation}
  \ev{V_D}{2\ 0\ 0} = \frac{\hbar^2}{8m^2c^2}
  \frac{e^{^2}}{\varepsilon_0}
  \underbrace{|\varphi_{200}(\boldrm{r}=\boldrm{0})|^2}_{1/ 8\pi a_0^3} =
  \frac{1}{16}mc^2 \alpha^4
\end{equation}
Para $l=1$ tenemos:
\begin{equation}
  \begin{split}
    \bra{210}V_D\ket{210} &\propto \iiint R_{21}^*(r){Y_{1}^{0}}^*(\Omega)
    \delta(\boldrm{r})  R_{21}(r){Y_{1}^{0}}(\Omega) \dd{V} \propto \\ &\propto
    |\varphi_{210}({\boldrm{0}})|^2=0
  \end{split}
\end{equation}

Obtenemos la siguiente matriz:

\begin{equation}
  V_D =  mc^2 \alpha^4
\begin{tikzpicture}[
    node distance=1mm and 0mm,
    baseline]
\matrix (M1) [matrix of nodes,{left delimiter=(},{right delimiter=)}]
{
  $\frac{1}{16}$ & $ $ & $ $ & $ $ &  $ $ & $ $ & $ $ & $ $ \\
  $ $ & $\frac{1}{16}$ & $ $ & $ $ &  $ $ & $ $ & $ $ & $ $ \\
  $ $ & $ $ & $0$ & $ $ &  $ $ & $ $ & $ $ & $ $ \\
  $ $ & $ $ & $ $ & $0$ &  $ $ & $ $ & $ $ & $ $ \\
  $ $ & $ $ & $ $ & $ $ &  $0$ & $ $ & $ $ & $ $ \\
  $ $ & $ $ & $ $ & $ $ &  $ $ & $0$ & $ $ & $ $ \\
  $ $ & $ $ & $ $ & $ $ &  $ $ & $ $ & $0$ & $ $ \\
  $ $ & $ $ & $ $ & $ $ &  $ $ & $ $ & $ $ & $0$ \\
};
\draw[red,thick] 
        (M1-1-1.north west) -| (M1-2-2.south east) -| (M1-1-1.north
        west) node[above right=0.2cm] {$\ell=0,j=\oh$};
\draw[red,thick] 
        (M1-3-3.north west) -| (M1-4-4.south east) -| (M1-3-3.north
        west) node[above right=0.2cm] {$\ell=1,j=\oh$};
\draw[red,thick] 
        (M1-5-5.north west) -| (M1-8-8.south east) -| (M1-5-5.north
        west)  node[above right=0.2cm,fill=white] {$\ell=1,j=\nicefrac{3}{2}$};
\end{tikzpicture}
\end{equation}

\subsection{Cálculo de $V_\text{SO}$}
Este término presenta mayor complejidad, al no desacoplarse por completo.

\paragraph{$\ell=0$}
De manera similar a lo visto en la subsección \ref{subs:SOfirst}, el término es nulo
por no haber órbita.

\paragraph{$\ell=1,j=\oh$}


Calculamos $V_\text{S0}$ en $\ket{2\ 1\  \oh \ \oh}$, por ejemplo
\begin{equation}
  \begin{split}
    &\ev{\text{cte}\cdot f(r) \frac{1}{2}
      (\boldrm{J}^2 -\boldrm{L}^2 -\boldrm{S}^2)  }{2\ 1\  \oh \  \oh}= \\
    &= \text{cte}\cdot (-\hbar^2) \bra{2\ 1\ \oh \ \oh}
    \frac{1}{r^3} \ket{2\ 1 \ \oh \ \oh} \stackrel{\text{like }V_m}{=}
    \\ &= \text{cte}\cdot(-\hbar^2)
    \bra{2\ 1\ 0}\frac{1}{r^3}\ket{2\ 1\ 0} = \\ &= - \frac{1}{48}mc^2\alpha^4
  \end{split}
\end{equation}
donde se ha utilizado\footnote{Basándonos en los autovalores de los operadores.} que
\begin{equation}
  \begin{split}
    &\frac{1}{2} (\boldrm{J}^2 -\boldrm{L}^2 -\boldrm{S}^2) \ket{2\ 1\
      \oh \ \oh} = \\ &= \frac{1}{2} \left[ \frac{1}{2} \left( \frac{1}{2}
        +1 \right) - 1(1+1)-\frac{1}{2}\left( \frac{1}{2}+1 \right)
    \right] \hbar^2 \ket{2\ 1\ \oh \ \oh}
  \end{split}
\end{equation}



\paragraph{$\ell=1,j=\nicefrac{3}{2}$}
De manera similar al caso anterior,
\begin{equation}
  \begin{split}
    &\bra{2\ 1\  \nicefrac{3}{2}\ m} \text{cte}\cdot \frac{1}{r^3}
    \underbrace{\frac{1}{2} (\boldrm{J}^2 - \boldrm{L}^2 -
      \boldrm{S}^2) \ket{ 2\ 1\  \nicefrac{3}{2}\ m}}_{\frac{1}{2}\left(
        \frac{15}{4}-2-\frac{3}{4} \right)\hbar^2 \ket{2\ 1\ 
        \nicefrac{3}{2}\ m}}= \\ &=  \cdots  = \frac{1}{96}mc^2 \alpha^4
\end{split}
\end{equation}

\paragraph{Resultado}
La matriz obtenida es:

\begin{equation}
  V_\text{SO} = mc^2 \alpha^4
\begin{tikzpicture}[
    node distance=1mm and 0mm,
    baseline]
\matrix (M1) [matrix of nodes,{left delimiter=(},{right delimiter=)}]
{
  $0$ & $ $ & $ $ & $ $ &  $ $ & $ $ & $ $ & $ $ \\
  $ $ & $0$ & $ $ & $ $ &  $ $ & $ $ & $ $ & $ $ \\
  $ $ & $ $ & $-\frac{1}{48}$ & $ $ &  $ $ & $ $ & $ $ & $ $ \\
  $ $ & $ $ & $ $ & $-\frac{1}{48}$ &  $ $ & $ $ & $ $ & $ $ \\
  $ $ & $ $ & $ $ & $ $ &  $\frac{1}{96}$ & $ $ & $ $ & $ $ \\
  $ $ & $ $ & $ $ & $ $ &  $ $ & $\frac{1}{96}$ & $ $ & $ $ \\
  $ $ & $ $ & $ $ & $ $ &  $ $ & $ $ & $\frac{1}{96}$ & $ $ \\
  $ $ & $ $ & $ $ & $ $ &  $ $ & $ $ & $ $ & $\frac{1}{96}$ \\
};
\draw[red,thick] 
        (M1-1-1.north west) -| (M1-2-2.south east) -| (M1-1-1.north
        west) node[above right=0.2cm] {$\ell=0,j=\oh$};
\draw[red,thick] 
        (M1-3-3.north west) -| (M1-4-4.south east) -| (M1-3-3.north
        west) node[above right=0.2cm] {$\ell=1,j=\oh$};
\draw[red,thick] 
        (M1-5-5.north west) -| (M1-8-8.south east) -| (M1-5-5.north
        west)  node[above right=0.2cm,fill=white] {$\ell=1,j=\nicefrac{3}{2}$};
\end{tikzpicture}
\end{equation}

\subsection{Conclusión}

La matriz de perturbación resultante, tras sumar
$V_m,V_D$ y $V_\text{SO}$, queda:

\begin{equation}
  V = \frac{1}{128}mc^2 \alpha^4
\begin{tikzpicture}[
    node distance=1mm and 0mm,
    baseline]
\matrix (M1) [matrix of nodes,{left delimiter=(},{right delimiter=)}]
{
    -5 & & & & & & & \\
       &-5 & & & & & & \\
       & & -5& & & & & \\
       & & & -5&& & & \\
       & & & & -1& & & \\
       & & & & &-1 & & \\
       & & & & & &-1 & \\
       & & & & & & &-1 \\
};
\draw[red,thick] 
        (M1-1-1.north west) -| (M1-2-2.south east) -| (M1-1-1.north
        west) node[above right=0.2cm] {$\ell=0,j=\oh$};
\draw[red,thick] 
        (M1-3-3.north west) -| (M1-4-4.south east) -| (M1-3-3.north
        west) node[above right=0.2cm] {$\ell=1,j=\oh$};
\draw[red,thick] 
        (M1-5-5.north west) -| (M1-8-8.south east) -| (M1-5-5.north
        west)  node[above right=0.2cm] {$\ell=1,j=\nicefrac{3}{2}$};
\end{tikzpicture}
\end{equation}


Notar la simetría oculta, las dos primeras cajas son iguales.


\chapter{Método variacional en el litio (3.1)}
\begin{tcolorbox}[halign=left]
  \emph{
    Calcule con el método de variaciones una cota superior a la
    energía del nivel fundamental del átomo de Li con una función
    prueba cuya parte espacial sea
    $\varphi(\boldrm{r}_1;\alpha)\varphi(\boldrm{r}_2;\alpha)\varphi(\boldrm{r}_3;\alpha)$,
    con $\alpha$ un parámetro adimensional. Si el valor experimental
    es \SI{-203.5}{\eV}, ¿es correcto su cálculo? ¿Por qué?
}
\end{tcolorbox}

Tenemos que
\begin{equation}
  \varphi(\vec{\boldrm{r}},\alpha) = \left( \frac{\alpha^3}{\pi a_0^3}
  \right)^{1/2} e^{-\alpha r/a_0}
\end{equation}
con $\varphi_{100}(\boldrm{r})$ autoestado de
$\frac{-\hbar^2}{m}\nabla^2- \frac{\alpha e^2}{r}$. Podemos expresar
el hamiltoniano del sistema como
\begin{equation}
  \Ham = T_1+T_2+T_3+V_1+V_2+V_3+V_{12}+V_{23}+V_{31}
\end{equation}
Como función prueba, emplearemos $\psi =
\varphi(\boldrm{r}_1,\alpha)\varphi(\boldrm{r}_2,\alpha)\varphi(\boldrm{r}_3,\alpha)$.

Hallamos $T_1$ (las otras dos son idénticas) con ayuda del teorema del
virial (figura \ref{fig:virialnote}).

\begin{marginfigure}
  \begin{tikzpicture}[xscale=1.5,yscale=0.5]
    \draw [color=blue, thick] (0,1) -- (2,1);
    \draw node [right] at (2,1) {$T$};
    % 
    \draw [thin, dashed] (0,0) -- (2,0);
    \draw node [right] at (2,0) {$0$};
    % 
    \draw [color=blue, thick] (0,-1) -- (2,-1);
    \draw node [right] at (2,-1) {$E$};
    % 
    \draw [color=blue, thick] (0,-2) -- (2,-2);
    \draw node [right] at (2,-2) {$V$};
    % 
    \draw [thick,->] (-0.2,-2) -- (-0.2,1);
    \draw node [left, rotate=90] at (-0.4,0.5) {$\text{Energy}$};
  \end{tikzpicture}
  \caption{Teorema del virial}
  \label{fig:virialnote}
\end{marginfigure}
\begin{equation}
  \begin{split}
    \langle T_1 \rangle &= \ev{T_1}{\varphi(\boldrm{r}_1,\alpha)}
    \underbrace{\ip{\varphi(\boldrm{2})}}_{=1} \underbrace{\ip{\varphi(\boldrm{3})}}_{=1} = \\
    & \stackrel{\text{virial}}= \frac{1}{2} \frac{e^2}{a_0}\alpha^2
  \end{split}
\end{equation}
como hay tres términos $\langle T_i\rangle$ obtengo el triple de este
resultado. Para los potenciales $\langle V_i \rangle$,
\begin{equation}
  \begin{split}
    \langle V_1 \rangle  &= \ev{V_1}{\varphi(\boldrm{r}_1,\alpha)}
    1\cdot 1  = \\
    & \stackrel{\text{virial}}=  \frac{-e^2}{a_0}\alpha^1 Z
  \end{split}
\end{equation}

y para los términos cruzados, como ya se vió\footnote{Vete a saber donde},
\begin{equation}
  \begin{split}
    \langle V_{12} \rangle  &= \ev{V_{12}}{\varphi(\boldrm{r}_1,\alpha)\varphi(\boldrm{r}_2,\alpha)}
     = \\
    &= \cdots = \frac{5}{8} \frac{e^2}{a_0}\alpha^1 
  \end{split}
\end{equation}
Si sumamos todo obtenemos
\begin{equation}
  \langle \Ham \rangle_{\psi(\alpha)} = 3 \frac{e^2}{a_0} \left(
    \frac{\alpha^2}{2}-Z\alpha+ \frac{5}{8}\alpha \right) \geq E_0 =
  \SI{-203.5}{\eV},\ \ \forall \alpha
\end{equation}
El mínimo está en $\alpha = Z- \nicefrac{5}{8}$, que en el litio
($Z=3$) resulta en \SI{-230.1}{\eV}. Vemos que el resultado es menor
que la energía mínima del sistema, por lo que está mal.

Esto se debe a que hemos utilizado una función que no es
antisimétrica, a pesar de ser los electrones fermiones.

\chapter{Acoplo de dos electrones (3.2)} 
\begin{tcolorbox}[halign=left]
  \emph{
    Calcule la función de ondas de dos electrones en la configuración
    $(2p)(3p)$ acoplados a un momento angular total $L=2$, $M_L=2$ y
    con espín total $S=1$,$M_S=-1$. ¿Pueden acoplarse a $L=1$, $M_L=1$
    con $S=0$, $M_S=0$? Si es así, calcule su función de ondas. Si
    ambos electrones están en la configuración $(2p)^2$, ¿pueden
    obtenerse los resultados anteriores?
}
\end{tcolorbox}
Tenemos una función de ondas global 
\begin{equation}
\psi = \ket{(2p)(3p) L S  M_L M_S
}=\ket{(2p)(3p)\  2\ 1 \ +2\ -1 }
\end{equation}
Utilizamos la tabla de Clebsch-Gordan para la parte espacial\footnote{Estamos acoplando dos
  momentos angulares $1\otimes1$. Queremos pasar de la base
  $\ell_1=1,\ell_2=1$ (dos orbitales \emph{p}) a la base $L=2,M_L=2$.
  Si consultamos la tabla $1\otimes 1$, vemos que el coeficiente es simplemente la
  unidad:
  \begin{center}
    \begin{tabular}{cc|c|}
      \cline{3-3}
      &    & 2          \\ \cline{3-3} 
      &    & 2          \\ \hline
      \multicolumn{1}{|l|}{-1} & -1 & \textbf{1} \\ \hline
    \end{tabular}
  \end{center}
}y la de espín\footnote{Estamos acoplando dos
  momentos angulares $\oh\otimes\oh$. Queremos pasar de la base
  $s_1=\oh,s_2=\oh$ (dos electrones) a la base $S=1,M_S=-1$.
  Si consultamos la tabla $\oh\otimes\oh$, vemos que el coeficiente es simplemente la
  unidad:
  \begin{center}
    \begin{tabular}{cc|c|}
      \cline{3-3}
      &    & 1          \\ \cline{3-3} 
      &    & -1          \\ \hline
      \multicolumn{1}{|l|}{$-\oh$} & $-\oh$ & \textbf{1} \\ \hline
    \end{tabular}
  \end{center}
}, y obtenemos



\newcommand{\tikzmark}[1]{\tikz[overlay,remember picture] \node (#1) {};}

\tikzset{square arrow/.style={
    to path={-- ++(0,-.25)  -| (\tikztotarget) \tikztonodes},below,pos=.25}}
\begin{equation}
  \psi = \underbrace{+1}_{\text{C.G.}} \cdot
  R_{2p}\tikzmark{a}(\boldrm{r}_1)Y_1^{+1}\tikzmark{b}(\Omega_1) \cdot
  R_{3p}\tikzmark{c}(\boldrm{r}_2)Y_1^{+1}\tikzmark{d}(\Omega_2) \otimes \underbrace{+1}_{\text{C.G.}}\cdot\ket{-}_1 \ket{-}_2
  \tikz[overlay,remember picture]
  {
   \path[
   color=gray,draw,out=-35,in=-145,
   line width=0.5mm,
   cap=round,
   dash pattern=on .05mm off 8mm,
   line cap=round,
   dotted
   ] ([xshift=3pt,yshift=-5pt]a.south) to ([xshift=-3pt,yshift=-5pt]b.south) ;
   \draw[color=gray,thick] ([xshift=0pt,yshift=0pt]a.south) circle (0.2cm);
   \draw[color=gray,thick] ([xshift=0pt,yshift=0pt]b.south) circle (0.2cm);
   %
   \path[
   color=gray,draw,out=-35,in=-145,
   line width=0.5mm,
   cap=round,
   dash pattern=on .05mm off 8mm,
   line cap=round,
   dotted
   ] ([xshift=3pt,yshift=-5pt]c.south) to ([xshift=-3pt,yshift=-5pt]d.south) ;
   \draw[color=gray,thick] ([xshift=0pt,yshift=0pt]c.south) circle (0.2cm);
   \draw[color=gray,thick] ([xshift=0pt,yshift=0pt]d.south) circle (0.2cm);
    }
\end{equation}
Notar como los términos radial y angular no son separables
\jokenote{Esto es inseparable, va con esposas, como Mario Conde.
  Bueno, Mario Conde lo dudo.}.

La parte espacial no es simétrica ni antisimétrica, así que sumándole
y restándole su transposición ($1 \leftrightarrow 2$) podemos
simetrizarla o antisimetrizarla a placer. Como su transposición es
ortogonal, el factor de normalización será $\frac{1}{\sqrt{2}}$. La
parte de espín es simétrica, por lo que necesitamos una parte espacial
antisimétrica:
\begin{equation}
  \begin{split}
    \psi &= \frac{1}{\sqrt{2}} \Big[
      R_{2p}(\boldrm{1})Y_1^{+1}(\boldrm{1}) \cdot
      R_{3p}(\boldrm{2})Y_1^{+1}(\boldrm{2}) - \\
      &- R_{2p}(\boldrm{2})Y_1^{+1}(\boldrm{2}) \cdot
      R_{3p}(\boldrm{1})Y_1^{+1}(\boldrm{1}) \Big] \otimes \ket{-}_1
    \ket{-}_2
  \end{split}
\end{equation}

Si en cambio nos dan una función de ondas

\begin{equation}
\psi' = \ket{(2p)^2 L S M_L M_S
}=\ket{(2p)^2\  2\ 1 \ +2\ -1 }
\end{equation}

Ya no podremos antisimetrizar la parte espacial, ya que es
completamente simétrica. Su transposición es la misma función, y si se
la restamos obtendríamos $\psi'=0$.

Esta función ya no permite el acoplo, ya que la función sería o nula
($\Psi=0$) o violaría el postulado de simetrización al ser simétrica
sobre fermiones.

\chapter{Momento angular del nivel fundamental (3.3)}
\begin{tcolorbox}[halign=left]
  \emph{
    Suponga dos electrones en una configuración $2p^2$. ¿Cuál es la
    degenaricoń de esta configuración? ¿Cuál es la degeneración de
    cada uno de los niveles obtenidos en acoplamiento LS? Calcule, en
    esta configuración, una función de ondas con $J=0$ y otra con
    $J=2$ y $M=+2$.
}
\end{tcolorbox}
Tenemos una función de ondas del tipo $(2p)^2$, con $J=0$ (por lo que
$M=0$) o con $J=2,M=+2$. De los cuatro números cuánticos sólo se dan
dos en el enunciado, por lo que quizá halla varias funciones de onda.
Hacemos una tabla\footnote{Se descartan las combinaciones que
  imposibilitan una función de ondas global antisimétrica.} de acoplamiento L-S:
\begin{center}
  \begin{tabular}{ccl}
    L & S & J \\ \hline
    $2_S$ & $0_A$ & $2$  \\ 
    $1_A$ & $1_S$ & $0,1,2$ \\ 
    $0_S$ & $0_A$ & $0$ \\ 
  \end{tabular}
\end{center}
Tenemos dos funciones de onda para cada enunciado, las dos primeras
filas valen para $J=2,M=+2$ y las dos últimas para $J=0,M=0$. 

La más simple es el ``tope de la escalera'', $2\otimes 0$ para dar $J=2, M=+2$. El
coeficiente de Clebsch-Gordan correspondiente\footnote{Si se consulta
una tabla se verá que no existe por ser obvio} es la unidad:
\marginnote{ Recordar que la notación $F_{(2p)^2}  \smqty{M_L\\ L}$ tiene
  el significado de $F_{\ell\ell'}\ket{M_L\
    L}(\boldrm{r}_2,\boldrm{r}_1)$, con $F$ la función espacial
  correspondiente (parte radial y angular).}
\begin{equation}
  \begin{split}
    \Psi &=+1 \underbrace{F_{(2p)^2} \smqty{+2\\ 2} (\boldrm{1},\boldrm{2})}_{
      \mathclap{
        R_{2p}\boldrm{1}Y_1^1(\boldrm{1})R_{2p}(\boldrm{2})Y^1_1(\boldrm{2})
      } } \ket{0,0} = \\
    &= \text{sim} \otimes \text{antisim} = \text{antisim} \  \checkmark
  \end{split}
\end{equation}

A continuación, resolvemos el caso $1\otimes 1$ para dar de nuevo
$J=2,M=+2$. La tabla de Clebsch-Gordan nos da un coeficiente unitario
de nuevo:
\begin{equation}
  \begin{split}
    \ket*{(2p)^2\ 
      \underset{L}{1}\ \underset{S}{1}\ \underset{J}{2}\ \underset{M}{2}}\tikzmark{a} &=
    +1 \cdot F_{(2p)^2} \smqty{+1 \tikzmark{c}\\1} (\boldrm{1},\boldrm{2}) \ket{1
      \ {+1}\tikzmark{b}} = \\
    &\text{antisim} \otimes \text{sim} = \text{antisim} \ \checkmark
  \end{split}
  \tikz[overlay,remember picture]
  {
   \draw[color=red,thin] ([xshift=-10pt,yshift=3pt]a) circle (0.2cm); 
   \draw[color=red,thin] ([xshift=-3pt,yshift=2.3pt]b) circle (0.3cm);
   \draw[color=red,thin] ([xshift=-3pt,yshift=2.3pt]c) circle (0.15cm);
    }
\end{equation}
Notar como las componentes marcadas en rojo a ambos lados de la
ecuación suman lo mismo. Esto debería ocurrir siempre.

La función espacial es simétrica en este caso\footnote{ Se tiene 
  \begin{equation*}
    \textstyle
    \begin{split}
      F_{(2p)^2} \smqty{+1\\1} &= \frac{1}{\sqrt{2}}
      R_{2p}(\boldrm{1})Y_1^{+1}(\boldrm{1})
      R_{2p}(\boldrm{2})Y_1^0(\boldrm{2}) - \\
      &- \frac{1}{\sqrt{2}}
      R_{2p}(\boldrm{2})Y_1^{0}(\boldrm{2})
      R_{2p}(\boldrm{1})Y_1^{+1}
    \end{split}
  \end{equation*}
}.

A continuación, consideremos el caso $0\otimes 0$ para dar $J=0, M=0$.
Se tiene de nuevo un Clebsch-Gordan unitario.
\begin{equation}
  \begin{split}
    \ket*{(2p) \underset{L}{0}\ \underset{S}{0}\ \underset{J}{0}\
      \underset{M}{0}} &= +1 \cdot F_{(2p)^2} \smqty{0\\0}
    (\boldrm{1},\boldrm{2}) \ket{0 \ {0}} = \\
    &=\text{sim}\otimes \text{antisim} = \text{antisim} \ \checkmark
  \end{split}
\end{equation}

Se puede comprobar que el factor $F_{(2p)^2}\smqty{0\\0}$ es par
expandiéndolo:
\begin{equation}
  \begin{split}
    F_{(2p)^2} \smqty{0\\0}&= 
    \frac{+1}{\sqrt{3}} R_{2p}(\boldrm{1})Y_1^{1\tikzmark{a}}(\boldrm{1})
    R_{2p}(\boldrm{2}) Y_1^{-1\tikzmark{b}}(\boldrm{2}) -\\
    &- \frac{1}{\sqrt{3}} R_{2p}(\boldrm{1})Y_1^{0\tikzmark{c}}(\boldrm{1})
    R_{2p}(\boldrm{2}) Y_1^{0\tikzmark{d}}(\boldrm{2}) + \\
    & +\frac{1}{\sqrt{3}} R_{2p}(\boldrm{1})Y_1^{-1\tikzmark{e}}(\boldrm{1}) R_{2p}(\boldrm{2}) Y_1^{1\tikzmark{f}}(\boldrm{2})
  \end{split}
  \tikz[overlay,remember picture]
  {
   \draw[color=red,thin] ([xshift=-3pt,yshift=2.3pt]a) circle (0.2cm);
   \draw[color=red,thin] ([xshift=-3pt,yshift=2.3pt]b) circle (0.2cm);
   \draw[color=red,thin] ([xshift=-3pt,yshift=2.3pt]c) circle (0.2cm);
   \draw[color=red,thin] ([xshift=-3pt,yshift=2.3pt]d) circle (0.2cm);
   \draw[color=red,thin] ([xshift=-3pt,yshift=2.3pt]e) circle (0.2cm);
   \draw[color=red,thin] ([xshift=-3pt,yshift=2.3pt]f) circle (0.2cm);
    }
\end{equation}

donde se han utilizado la fila correspondiente de los coeficientes de Clebsch-Gordan
correspondientes\footnote{Utilizamos la tabla $1\otimes 1$, y
  obtenemos
\begin{center}
\begin{tabular}{ll|l|l|l|}
\cline{3-5}
                         &    & 2   & 1    & \textbf{0} \\ \cline{3-5} 
                         &    & 0   & 0    & \textbf{0} \\ \hline
\multicolumn{1}{|l|}{+1} & -1 & 1/6 & 1/2  & 1/3        \\ \hline
\multicolumn{1}{|l|}{0}  & 0  & 2/3 & 0    & -1/3       \\ \hline
\multicolumn{1}{|l|}{-1} & +1 & 1/6 & -1/2 & 1/3        \\ \hline
\end{tabular}
\end{center}
Luego los coeficientes son $\nicefrac{1}{3}$, $\nicefrac{-1}{3}$ y
$\nicefrac{1}{3}$. Para los círculos rojos empleamos las dos filas de
la izquierda ,$\smqty{+1&-1\\ 0 & 0 \\ -1 & +1}$.
} en los índices marcados en rojo.

Por último, resolvemos el $1\otimes 1$ para dar $J=0,M=0$ (se omiten los
otros dos). En este caso, el coeficiente de Clebsch-Gordan ya no es
trivial\footnote{Hay que consultar una tabla $1\otimes 1$, y pasar a
$\ket*{\underset{J}{0}\ \underset{M}{0}}$. Son los mismos coeficientes
que en el último caso, $\nicefrac{1}{3}$, $\nicefrac{-1}{3}$ y
$\nicefrac{1}{3}$.}:
\begin{equation}
  \begin{split}
    \ket*{(2p)^2\underset{L}{1}\ \underset{S}{1}\ \underset{J}{0}\
      \underset{M}{0}} &= \frac{1}{\sqrt{3}} F_{(2p)^2}\smqty{1\tikzmark{a}\\1}
    \ket{1 \ {-1\tikzmark{b}}} - \\
& -\frac{1}{\sqrt{3}} F_{(2p)^2}\smqty{0\tikzmark{c}\\1}
    \ket{1 \ {0\tikzmark{d}}} + \\
& +\frac{1}{\sqrt{3}} F_{(2p)^2}\smqty{-1\tikzmark{e}\\1}
    \ket{1 \ {+1\tikzmark{f}}} - \\
  \end{split}
  \tikz[overlay,remember picture]
  {
   \draw[color=red,thin] ([xshift=-3pt,yshift=2.3pt]a) circle (0.2cm);
   \draw[color=red,thin] ([xshift=-3pt,yshift=2.3pt]b) circle (0.3cm);
   \draw[color=red,thin] ([xshift=-3pt,yshift=2.3pt]c) circle (0.2cm);
   \draw[color=red,thin] ([xshift=-3pt,yshift=2.3pt]d) circle (0.25cm);
   \draw[color=red,thin] ([xshift=-3pt,yshift=2.3pt]e) circle (0.2cm);
   \draw[color=red,thin] ([xshift=-3pt,yshift=2.3pt]f) circle (0.3cm);
    }
\end{equation}
Nuevamente, sustituímos en los círculos rojos las filas de la tabla de
Clebsch-Gordan. Cada una de las $F$ se puede volver a sustituir
utilizando la tabla de Clebsch-Gordan, y se obtiene que la son todas
antisimétricas. De igual manera, todos los espinores son simétricos,
por lo que la función de ondas total es antisimétrica, como se deseaba.

\chapter{Incertidumbre en el oscilador armónico (4.1)}
\begin{tcolorbox}[halign=left]
  \emph{
    Calcule, con ayuda de los operadores de creación y
    destrucción, el producto de incertidumbre $\Delta X \Delta P$ para
  el autoestado n-ésimo del oscilador armónico unidimensional.}
\end{tcolorbox}
Tenemos que
\begin{align}
a &= \frac{1}{\sqrt{2}} ( \hat{X}+i\hat{P}) \\
a^\dagger &= \frac{1}{\sqrt{2}} ( \hat{X}-i\hat{P}) 
\end{align}
Necesitamos calcular $(\Delta X \Delta P)^2=(\langle X^2 \rangle-\langle
X\rangle^2)(\langle P^2 \rangle-\langle P\rangle^2)$, donde $X =
\sqrt{\frac{\hbar}{m\omega}}\hat{X}$ y $P = \sqrt{m \hbar
  \omega}\hat{P} $. Calculamos las dos primeras integrales (los
$\langle a \rangle$) despejando $X,P$ en función de $a,a^\dagger$:

\begin{align}
  \mel{n}{X}{n} &= \sqrt{\frac{\hbar}{m\omega}} ( \underbrace{\mel{n}{a^\dagger}{n}}_{{=\braket{n}{n+1}=0}}
  + \underbrace{\mel{n}{a}{n}}_{=\braket{n}{n-1}=0})
  \frac{1}{\sqrt{2}} = 0 \\
  \mel{n}{P}{n} &= \sqrt{m \hbar\omega} ( \underbrace{\mel{n}{a^\dagger}{n}}_{{=\braket{n}{n+1}=0}}
  - \underbrace{\mel{n}{a}{n}}_{=\braket{n}{n-1}=0})
  \frac{1}{\sqrt{2}} = 0 
\end{align}
Utilizando $[a,a^\dagger]=1$ deducimos que $aa^\dagger-a^\dagger a =
1$, y por tanto $a a^\dagger = 1 + a^\dagger a = 1 + N$. De ahí
deducimos que
\begin{align}
  \mel{n}{X^2}{n} &= \cdots = \frac{\hbar}{2m\omega}(1+2n)\\
  \mel{n}{P^2}{n} &= \cdots = \frac{m\hbar\omega}{2}(1+2n)
\end{align}
con el mismo procedimiento que antes. Finalmente,
\begin{equation}
  \Delta X \Delta P = \frac{\hbar}{2}(1+2n)
\end{equation}

\chapter{Elementos de matriz del hidrógeno (4.2)}
\label{ej:integralloca}
\begin{tcolorbox}[halign=left]
  \emph{Calcule en el átomo de hidrógeno el elemento de matriz 
    $\mel*{1\  0 \ \oh \ \oh \ {+\oh}}{r_0^1}{2\ 1  \ {\oh} \
      {\nicefrac{3}{2}} \ {+\oh}}$ 
  donde los kets indican los elementos $\ket{n\ \ell\ s \ j \ m}$.}
\end{tcolorbox}

De manera explícita, los kets pueden escribirse como
\begin{align}
  \ket{1\ 0 \ \oh \ \oh \ {+\oh}} &= \underbrace{1}_{\text{C.G.}} 
  \varphi_{100}(\boldrm{r}) \ket{\oh \ {+\oh}} \\
  \ket{2\ 1 \ \oh \ \nicefrac{3}{2} \ {+\oh}} &= \underbrace{\sqrt{\frac{1}{3}}}_{\text{C.G.}} 
  \varphi_{211}(\boldrm{r}) \ket{\oh \ {-\oh}} + \underbrace{\sqrt{\frac{2}{3}}}_{\text{C.G.}} 
  \varphi_{210}(\boldrm{r}) \ket{\oh \ {+\oh}}
\end{align}
En el caso no trivial se ha empleado la tabla $1\otimes \oh$. Con los
elementos desarrollados, pasamos a calcular el elemento de matriz
(recordar que $r_0^1=z$):
\begin{equation}
  \begin{split}
    \mel*{\cdots}{r_0^1}{\cdots} &= \sqrt{\frac{1}{3}}
    \mel{\varphi_{100}}{r_0^1}{\varphi_{211}} \underbrace{\braket{\oh\
        {+\oh}}{\oh \ {-\oh}}}_{=0} + \\ &+\sqrt{\frac{2}{3}} 
    \mel{\varphi_{100}}{r_0^1}{\varphi_{210}} \underbrace{\braket{\oh\
        {+\oh}}{\oh \ {+\oh}}}_{=1} = \\
    &=\sqrt{\frac{2}{3}} \mel{\varphi_{100}}{r_0^1}{\varphi_{210}}
  \end{split}
  \label{eq:electriceye}
\end{equation}
Vemos que la integral, a priori, no es nula:
\begin{itemize}
\item Tenemos que
  $\mel{\varphi_{1\textcolor{red!80!black}{\boldrm{0}}0}}{r_0^{\textcolor{red!80!black}{\boldrm{1}}}}{\varphi_{2\textcolor{red!80!black}{\boldrm{1}}0}}$
  cumple $\textcolor{red!80!black}{0} \in
  \textcolor{red!80!black}{1}\otimes \textcolor{red!80!black}{1} =
  \{0,1,2\}$.
\item También se cumple para
  $\mel{\varphi_{10\textcolor{green!80!black}{\boldrm{0}}}}{r^1_{\textcolor{green!80!black}{\boldrm{0}}}}{\varphi_{21\textcolor{green!80!black}{\boldrm{0}}}}$
  que $\textcolor{green!80!black}{0}=\textcolor{green!80!black}{0} + \textcolor{green!80!black}{0}$.
\end{itemize}
Si se incumpliera una sóla de estas reglas no necesitariamos hacer la
integral, pero no ha habido suerte.

Notando que $r_0^1=z=r \sqrt{\frac{4\pi}{3}} Y_1^0(\Omega)$,
realizamos la integral:
\begin{equation}
  \begin{split}
    \mel{\phi_{100}}{r_0^1}{\phi_{210}} &= \sqrt{\frac{4\pi}{3}} \int_0^\infty R_{10}^* (r) r
    R_{21}(r) r^2 \dd{r} \underbrace{\int_{4\pi}
    Y_0^{0*}(\Omega)Y_{1}^0(\Omega)Y_1^0(\Omega) \dd{\Omega}}_{1/\sqrt{4\pi}} = \\
    &=\frac{1}{\sqrt{3}} \int_0^\infty \left( \frac{1}{a_0} \right)^{\nicefrac{3}{2}} 2
    e^{-r/a_0} r \left( \frac{1}{2a_0} \right)^{\nicefrac{3}{2}}
    \frac{r}{\sqrt{3}a_0} e^{\frac{-r}{2a_0}} r^2 \dd{r} = \cdots
  \end{split}
\end{equation}
La integral de los armónicos esféricos resulta
$\frac{1}{\sqrt{4\pi}}$, puede consultarse en tablas o notarse que es
una integral de normalización\footnotemark.
\footnotetext{Se tiene que $Y_0^0=\text{cte.}=\frac{1}{\sqrt{4\pi}}$,
  y que $Y_1^0 = Y_1^{0*}$.}

\begin{equation}
  \begin{split}
    \cdots &= \frac{1}{\sqrt{3}} \frac{1}{a_0^3} \frac{1}{a_0}
    \frac{2}{\sqrt{3}} \frac{1}{2^{3/2}} \int_0^\infty r^4 \exp \left( \frac{-3}{2}
      \frac{r}{a_0} \right) = \\
    &= \frac{\sqrt{2}}{2\cdot3 a_0^4} \times 24\cdot \left(
      \frac{2a_0}{3} \right)^5 = \sqrt{2} \frac{2^7}{3^5}a_0
  \end{split}
\end{equation}
donde se ha utilizado que $\int r^4 e^{-ar} \dd{r} = \frac{24}{a^5}$. Notar que aún es necesario multiplicar la integral por $\nicefrac{2}{3}$, debido a los coeficientes de \eqref{eq:electriceye}.

\begin{tcolorbox}[halign=left]
  \emph{
    Con ayuda del elemento de matriz anterior y las tablas de
    Clebsch-Gordan calcule los elementos (no nulos) de matriz siguientes:
  }
  \begin{align*}
    &\mel*{1\  0 \ \oh \ \oh \ {m_f}}{r^1_0}{2\ 1  \ {\oh} \      {\nicefrac{3}{2}} \ {m_i}} \\
    &\mel*{1\  0 \ \oh \ \oh \ {m_f}}{r^1_{+1}}{2\ 1  \ {\oh} \      {\nicefrac{3}{2}} \ {m_i}} \\
    &\mel*{1\  0 \ \oh \ \oh \ {m_f}}{r^1_{-1}}{2\ 1  \ {\oh} \      {\nicefrac{3}{2}} \ {m_i}}
  \end{align*}
  \emph{
    para cada posible pareja $m_f,m_i$.
  }
\end{tcolorbox}

Las posibles transiciones son:
\begin{center}
  \begin{tikzpicture}
    \tikzstyle energylevel=[thick,blue]
    \tikzstyle photon=[thick,yellow!60!black,snake=snake,line after snake=0.2cm,->]
    \tikzstyle fotonloco=[ultra thick,yellow!60!black,snake=snake,line after snake=0.2cm,->]
    \tikzstyle 
    photonlabel=[yellow!20!black,midway,fill=white]

    % labels
    \node [above] at (5.5,0.5) {$\ket{2\ 1\ \oh \ \nicefrac{3}{2}\ m_i}$};
    \node [below] at (5.5,-3.5) {$\ket{1\ 0\ \oh \ \nicefrac{1}{2}\ m_f}$};
    % 3/2, m_inicial
    \draw[energylevel] (0,0) -- (2,0);
    \node[fill=white] (a) at (1,0) {$\nicefrac{3}{2}$};
    % 1/2, m_inicial
    \draw[energylevel] (3,0) -- (5,0);
    \node[fill=white] (b) at (4,0) {$\nicefrac{1}{2}$};
    % -1/2, m_inicial
    \draw[energylevel] (6,0) -- (8,0);
    \node[fill=white] (c) at (7,0) {$\nicefrac{-1}{2}$};
    % -3/2, m_inicial
    \draw[energylevel] (9,0) -- (11,0);
    \node[fill=white] (d) at (10,0) {$\nicefrac{-3}{2}$};
    % +1/2, m_final
    \draw[energylevel] (1.5,-3) -- (4.5,-3);
    \node[fill=white] (e) at (3,-3) {$\nicefrac{+1}{2}$};
    % -1/2, m_final
    \draw[energylevel] (6.5,-3) -- (9.5,-3);
    \node[fill=white] (f) at (8,-3) {$\nicefrac{-1}{2}$};
    % transitions
    \draw[photon] (a) -- (e) node[photonlabel] {$r_{-1}^1$};
    \draw[photon] (b) -- (e) node[photonlabel] {$r_{0}^1$};
    \draw[photon] (b) -- (f) node[photonlabel] {$r_{-1}^1$};
    \draw[photon] (c) -- (e) node[photonlabel] {$r_{+1}^1$};
    \draw[photon] (c) -- (f) node[photonlabel] {$r_{0}^1$};
    \draw[fotonloco] (d) -- (f) node[photonlabel] {$r_{+1}^1$};
  \end{tikzpicture}
\end{center}



Notar como las transiciones siempre cumplen $m_i + \zeta = m_f $  con
$\zeta$ el subíndice del tensor $r_\zeta^1$. Las demás transiciones no
lo cumplen y se han omitido.

Calculamos, a modo de ejemplo, el elemento de matriz correspondiente a
la transición marcada con la línea gruesa:
\begin{equation}
  \mel{1\ 0\ \oh \ \nicefrac{1}{2}\ \nicefrac{-1}{2}}{r_{+1}^1}{2\ 1\
    \oh \ \nicefrac{3}{2} \ \nicefrac{-3}{2}} = \cdots
\end{equation}

El teorema de Wigner-Eckart nos dice que podemos relacionar esta
integral y la ya realizada mediante los coeficientes de
Clebsch-Gordan (\emph{nuevo} y \emph{ya calculado}):
  \begin{equation}
    \begin{split}
      \mel {1\ 0\ \nicefrac{1}{2} \ \textcolor{red!60!black}{\oh\
          \nicefrac{-1}{2}}}
      {r_{\textcolor{blue!60!black}{+1}}^{\textcolor{blue!60!black}{1}}}
      {2\ 1\ \oh \ \textcolor{green!60!black}{\nicefrac{3}{2}\
          \nicefrac{-3}{2}}} &= (
      \textcolor{green!60!black}{\nicefrac{3}{2}\ \nicefrac{-3}{2}}\
      \textcolor{blue!60!black}{1 \ +1} | \textcolor{red!60!black}{\oh
        \ \nicefrac{-1}{2}} ) \cdot \\ &\cdot
      \mel{\oh}{|r^1|}{\nicefrac{3}{2}} 
    \end{split}
      \end{equation}
      \begin{equation}
        \begin{split}
          \mel {1\ 0\ \nicefrac{1}{2} \ \textcolor{red!60!black}{\oh\
              \oh}}
          {r_{\textcolor{blue!60!black}{0}}^{\textcolor{blue!60!black}{1}}}
          {2\ 1\ \oh \ \textcolor{green!60!black}{\nicefrac{3}{2}\
              \nicefrac{1}{2}}} &= (
          \textcolor{green!60!black}{\nicefrac{3}{2}\
            \nicefrac{1}{2}}\ \textcolor{blue!60!black}{1 \ 0} |
          \textcolor{red!60!black}{\oh \ \oh} ) \cdot \\
          &\cdot
          \mel{\oh}{|r^1|}{\nicefrac{3}{2}} 
        \end{split} 
  \end{equation}
donde los paréntesis indican coeficientes de Clebsch-Gordan; por
ejemplo $(
      \textcolor{orange!60!black}{\nicefrac{3}{2}}\ 
      \textcolor{purple!60!black}{\nicefrac{-3}{2}}\
      \textcolor{orange!60!black}{1}
      \ \textcolor{purple!60!black}{+1}
      |
      \oh \ \nicefrac{-1}{2}
      )$ es el coeficiente de Clebsch-Gordan que acopla dos
      \textcolor{orange!60!black}{momentos angulares}
      $\nicefrac{3}{2}$ y $1$ de \textcolor{purple!60!black}{componentes}
      $\nicefrac{-3}{2}$ y $+1$, para dar un momento total $\oh$ con
      tercera componente $\nicefrac{-1}{2}$.\footnote{Consultando la tabla
        $\nicefrac{3}{2}\times 1$, vemos
        que el valor buscado es
        \begin{center}
          \begin{tabular}{ll|l|l|l|}
            \cline{3-5}
            &             & 5/2  & 3/2   & \textbf{1/2}  \\ \cline{3-5} 
            &             & -1/2 & -1/2  & \textbf{-1/2} \\ \hline
            \multicolumn{1}{|l|}{+1/2}          & -1          & 3/10 & 8/15  & 1/6           \\ \hline
            \multicolumn{1}{|l|}{-1/2}          & 0           & 3/5  & -1/15 & -1/3          \\ \hline
            \multicolumn{1}{|l|}{\textbf{-3/2}} & \textbf{+1} & 1/10 & -2/5  & \textcolor{red!60!black}{1/2}           \\ \hline
          \end{tabular}
        \end{center}
        El del primer elemento de matriz visto está en otro recuadro.
      }

Sustityendo los Clebsch-Gordan correspondientes en las ecuaciones
obtenemos
\begin{align}
\mel{\text{new}}{r_{+1}^1}{\text{new}} &= \sqrt{\frac{1}{2}}
                                         \mel{\oh}{|r^{1}}{\nicefrac{3}{2}} \\
\mel{\text{old}}{r_{0}^1}{\text{old}} &= -\sqrt{\frac{1}{3}}
                                         \mel{\oh}{|r^{1}}{\nicefrac{3}{2}}
                                        = \frac{\sqrt{2}\cdot2^8}{3^6}a_0
\end{align}

Dividendo ambas ecuaciones encontramos que
\begin{equation}
  \begin{split}
    \mel{\text{new}}{r_{+1}^1}{\text{new}} &=
    -\sqrt{\frac{1}{2}}\sqrt{\frac{3}{1}}\cancelto{1}{\frac{\mel{\oh}{|r^{1}}{\nicefrac{3}{2}}}{\mel{\oh}{|r^{1}}{\nicefrac{3}{2}}}}
    \mel{\text{old}}{r_{+1}^1}{\text{old}} = \\
    &= - \frac{\sqrt{3}\cdot 2^7}{3^6} a_0 
  \end{split}
\end{equation}

Procediendo de manera análoga es posible calcular todos los elementos
no nulos restantes.

\chapter{Esquema de niveles del carbono (4.4)}

\begin{tcolorbox}[halign=left]
  \emph{Dibuje el esquema de niveles de la configuración fundamental
    1s\textsuperscript{2}2s\textsuperscript{2}2p\textsuperscript{2} y
    de la excitada
    1s\textsuperscript{2}2s\textsuperscript{2}2p\textsuperscript{1}3p\textsuperscript{1}
  del átomo de carbono. Indique las transiciones dipolares eléctricas
  posibles entre los niveles de ambas configuraciones.}
\end{tcolorbox}

En el nivel inferior se tienen dos orbitales $p$ de momento angular
orbital $\ell=1$, de forma que
$\ell=1\otimes1=0_S,1_A,2_S$ y $S=0_A,1_S$. Como ambos electrones
están en el mismo nivel energético, no podemos simetrizar y
antisimetrizar la parte espacial a placer y sólo son posibles
las combinaciones simétricas.

Su ordenamiento, por el principio de Hund, sería
${}^{3}P,{}^{1}D,{}^{1}S$ listando de menor a mayor
energía\footnotemark.
\footnotetext{Comenzamos por escribirlos en orden de mayor
  multiplicidad:
\[{}^{3}P,\ {}^{1}D \text{ y } {}^{1}S\]
Para deshacer el ``empate'' entre ${}^{1}D$ y ${}^{1}S$, empleamos que
para una multiplicidad dada el de mayor $L$ es el menos energético, en
este caso ${}^{1}D$ tiene $L=2$ frente a $L=0$ de ${}^{1}S$.
}

De forma similar, en el nivel superior obtenemos ${}^{3}P,{}^{1}P$.

Por otra parte, cada triplete ${}^3P$ está subdividido en tres
subniveles ${}^3P_0$,${}^3P_1$ y ${}^3P_2$, que se ordenan de forma
que la $J$ sea creciente con la energía (tercera regla de Hund) por
estar la capa incompleta menos que semillena (multiplete normal).\footnotemark

\footnotetext{Hay un poco de duda porque no hay una única capa
  incompleta, pero por analogía con los ejemplos simples es la
  distribución más razonable. En átomos más complicados no es imposible
  esperar un multiplete normal y obtener multipletes invertidos o
  incluso mezclas.}

Veamos las transciones a distinto nivel de simplificación del
hamiltoniano:
\begin{description}
\item[$\Ham_0$]
  A este nivel (partículas independientes) sólo hay dos estados y
  una transición, permitida por ser $\Delta\ell =1$.
  \begin{center}
    \begin{tikzpicture}[xscale=2,yscale=1]
      \draw [color=blue, thick] (0,0) -- (5,0);
      \draw [color=blue, thick] (0,-3) -- (5,-3);
      \draw node [above] at (3,0) {$(\text{c.c.})(2p)^1(3s)^1$};
      \draw node [above] at (3,-3) {$(\text{c.c.})(2p)^2$};
      \draw [thick,yellow!60!black,
      snake=snake,
      line after snake=0.2cm, ->] (4.5,-0.5) -- (4.5,-2.5);
    \end{tikzpicture}
  \end{center}
\item[V] Cuando introducimos la repulsión coulombiana el esquema de
  niveles se complica. Las únicas transciones válidas son las de
  singlete a singlete y las de multiplete a multiplete, ya que el
  operador tensorial no transporta espín ($\Delta S = 0$). Además,
  hemos de cumplir $\ell_f \in \ell_i\otimes 1$.
  \begin{center}
    \begin{tikzpicture}[xscale=2,yscale=1]
      \tikzstyle photon=[thick,yellow!60!black,
      snake=snake, line after snake=0.2cm, ->]
      \draw [color=blue, thick] (0,0) -- (5,0);
      \draw [color=blue, thick] (0,-0.5) -- (5,-0.5);
      \draw [color=blue, thick] (0,-3) -- (5,-3);
      \draw [color=blue, thick] (0,-3.5) -- (5,-3.5);
      \draw [color=blue, thick] (0,-4) -- (5,-4);
      \draw node [above] at (4.5,0) {${}^1P$};
      \draw node [above] at (4.5,-0.5) {${}^3P$};
      \draw node [above] at (4.5,-3) {${}^1S$};
      \draw node [above] at (4.5,-3.5) {${}^1D$};
      \draw node [above] at (4.5,-4) {${}^3P$};
      \draw [photon] (1,0) -- (1,-3);
      \draw [photon] (1.5,0) -- (1.5,-3.5);
      \draw [photon] (3,-0.5) -- (3,-4);
    \end{tikzpicture}
  \end{center}
\item[V\textsubscript{fine}] No se muestran todas las posibles
  transiciones para no saturar el dibujo, sólo las que empiezan en
${}^3P_i$ (que tienen momento angular $J=i\in\{0,1,2\}$). Notar que hemos de cumplir
  que $J_f \in J_i\otimes 1$:
  \begin{itemize}
  \item Si partimos de ${}^3P_0$, con momento angular $J_i=0$, sólo
    podemos acceder a los estados finales con $J_f\in0\otimes1=1$.
  \item Para ${}^3P_1$ estamos limitados a $J_f \in 1\otimes 1 =
    \{0,1,2\}$.
  \item En ${}^3P_2$ se tiene $J_f \in 2\otimes 1 = \{1,2,3\}$. Notar
    que no hay ningún estado en el triplete inferior con $J=3$, y que
    no podemos acceder a $J=0$.
  \end{itemize}
  \begin{center}
    \begin{tikzpicture}[xscale=2,yscale=1]
      \tikzstyle photon=[thick,yellow!60!black,
      snake=snake, line after snake=0.2cm, ->]
      \draw [color=blue, thick] (0,0.5) -- (4,0.5);
      \draw node [right] at (4,0.5) {${}^1P$};
      \draw [color=blue] (0,-0.3) -- (4,-0.3);
      \draw [color=blue, thick] (0,-0.5) -- (4,-0.5);
      \draw [color=blue] (0,-0.7) -- (4,-0.7);
      \draw node [right] at (4.2,-0.5) {${}^3P$};
      \draw [color=blue, thick] (0,-2) -- (4,-2);
      \draw node [right] at (4,-2) {${}^1S$};
      \draw [color=blue, thick] (0,-3) -- (4,-3);
      \draw node [right] at (4,-3) {${}^1D$};
      \draw [color=blue] (0,-3.8) -- (4,-3.8);
      \draw [color=blue, thick] (0,-4) -- (4,-4);
      \draw [color=blue] (0,-4.2) -- (4,-4.2);
      \draw node [right] at (4.2,-4) {${}^3P$};
      \draw node [right] at (4,-4.2) {$\scriptstyle{0}$};
      \draw node [right] at (4,-4) {$\scriptstyle{1}$};
      \draw node [right] at (4,-3.8) {$\scriptstyle{2}$};
      \draw node [right] at (4,-0.7) {$\scriptstyle{0}$};
      \draw node [right] at (4,-0.5) {$\scriptstyle{1}$};
      \draw node [right] at (4,-0.3) {$\scriptstyle{2}$};
      \draw [photon] (1,-0.3) -- (1,-3.8);
      \draw [photon] (1.2,-0.3) -- (1.2,-4);
      \draw [photon] (2,-0.5) -- (2,-3.8);
      \draw [photon] (2.2,-0.5) -- (2.2,-4);
      \draw [photon] (2.4,-0.5) -- (2.4,-4.2);
      \draw [photon] (3,-0.7) -- (3,-4);
    \end{tikzpicture}
  \end{center}
\end{description}


\chapter{Enero, 2016 (1)}
\label{chapter:e161}
\begin{tcolorbox}[halign=left]
  \emph{Un electrón se encuentro en un pozo de potencial de paredes
    impenetrables en la región $x\in(0,L)$, $y\in(0,L)$,
    $z\in(0,L/10)$. 
  Hay, además, una perturbación $W=V(x,y,z)S_z/\hbar$ donde
  $V(x,y,z)=V_0$ en la región $x\in(0,L)$, $y\in(L/2,L)$,
  $z\in(0,L/10)$. Calcule en primer orden de perturbaciones la
  corrección a la energía del nivel fundamental y del primer nivel excitado.}
\end{tcolorbox}

\section*{Nivel fundamental}
Siendo $\varphi_n(x) = \sqrt{\frac{2}{L}}\sin \frac{n\pi x}{L}$ los
autoestados de un pozo de potencial cuadrado 1D, el estado fundamental
de este pozo será $\ket{111}=\varphi_1(x)\otimes\varphi_1(y)\otimes\varphi_1(z)$.

Calculamos la perturbación, teniendo en cuenta la degeneración del
nivel debida al espín:
\begin{equation}
  \begin{split}
    \Delta E &= \mel{111\pm}{W}{111\pm} =
    \mel{111}{V_0}{111}\mel{\pm}{S_z/\hbar}{\pm} \\
    &= V_0\braket{11}{11}\underbrace{\mel{1}{\vartheta(L/2)}{1}}_{=\nicefrac{1}{2} } \cdot \frac{1}{\hbar}
    \mel{\pm}{S_z}{\pm}
  \end{split}
\end{equation}
donde $\vartheta$ es la función escalón y se ha utilizado que la función de ondas está normalizada.
\footnote{No integramos en todo el espacio sino únicamente en
  $y\in(L/2,L)$, ya que la perturbación sólo existe en dicha región.
  La simetría de la función de ondas hace que la integral de sólo una
  mitad sea $\oh$ del valor en todo el espacio, $1$.}

Como $S_z$ es $\frac{\hbar}{2}\smqty(1& \\ &-1)$ obtenemos que
\begin{equation}
  \begin{pmatrix}
    \mel{111+}{W}{111+} &  \mel{111+}{W}{111-}\\
    \mel{111-}{W}{111+} &  \mel{111-}{W}{111-}
  \end{pmatrix} =  \frac{V_0}{4}
  \begin{pmatrix}
    1 &  \\
     & -1 
  \end{pmatrix} 
\end{equation}

Los autovalores de la diagonal son las correcciones a la energía, así
que obtenemos que la energía se desdobla en dos niveles, uno con
$E=E_1+\nicefrac{V_0}{4}$ y otro con $E=E_1-\nicefrac{V_0}{4}$.

\section*{Primer excitado}
La energía del sistema es, por ser un pozo cuadrado de potencial,
\begin{equation}
  E = n_1^2 \frac{\hbar^2\pi^2}{2mL^2} + n_2^2
  \frac{\hbar^2\pi^2}{2mL^2} + n_3^2 \frac{\hbar^2\pi^2}{2m \left(
      \nicefrac{L}{10} \right)^2}, \ \ n_i \in \mathbb{N}/\{0\}
\end{equation}
En unidades de $\frac{\hbar^2\pi^2}{2mL^2}$, obtenemos
\begin{center}
  \begin{tabular}{c|ccc}
    $E$&$n_1$&$n_2$&$n_3$ \\
    $1+1+1\cdot100=102$ & $1$ & $1$ & $1$ \\ \hline
    $2+1+1\cdot100=103$ & $2$ & $1$ & $1$ \\
    $1+2+1\cdot100=103$ & $1$ & $2$ & $1$ \\ \hline
    $3+1+1\cdot100=104$ & $3$ & $1$ & $1$ \\
    $1+3+1\cdot100=104$ & $1$ & $3$ & $1$ \\
    $2+2+1\cdot100=104$ & $2$ & $2$ & $1$ \\ \hline
    $\vdots$ & $\vdots$ & $\vdots$ & $\vdots$ \\ 
  \end{tabular}
\end{center}

Vemos que el nivel fundamental tiene degeneración dos debido a la
dimensión del problema, además de seguir teniendo la degeneración
debida al espín. Notamos que los elementos de matriz
$\mel{ij}{V_0}{ji}=0$ por la ortogonalidad\footnotemark  de los elementos de la
base; 
\footnotetext{Si bien $\mel{ij}{V_0}{ji}=0$ a priori no es nulo
  porque no estamos integrando a todo el espacio, podemos
  descomponerlo como
  $\mel{ij}{V_0}{ji}=\braket{i}{j}\mel{i}{V_0}{j}=0\cdot
  \underbrace{\mel{i}{V_0}{j}}_{\neq 0} = 0$}
utilizando los resultados anteriores con $S_z$ obtenemos una
matriz similar para el subespacio de degeneración, puramente diagonal:
\begin{equation}
  \frac{V_0\hbar}{4}
  \begin{pmatrix}
    1 &   & &  \\
      & -1 & &  \\
      & &1&  \\
      & & &-1 \\
  \end{pmatrix}
\end{equation}
Se ha utilizado que
$\mel{ij\pm}{W}{ij\pm}=V_0\braket{ij}{ij}\mel{\pm}{S_z/\hbar}{\pm} =
V_0\mel{\pm}{S_z/\hbar}{\pm}$. Nuevamente obtenemos un desdoblamiento
de los niveles, pero en este caso los cuatro estados sólo se desdoblan
en dos, no eliminándose completamente la degeneración.

\chapter{Enero, 2016 (2)}
\begin{tcolorbox}[halign=left]
  \emph{Explique que son las integrales directas y de intercambio en
    el átomo de helio.}
\end{tcolorbox}
Ver página \pageref{paragraph:intinterc}.
\begin{tcolorbox}[halign=left]
  \emph{Sabiendo que la diferencia de energías entre los términos
    ${}^1S$ y ${}^3S$ de la configuración $(1s)(2s)$ del átomo de helio
  son \SI{0.78}{\eV}, ¿cuánto vale la integral de intercambio en este caso?}
\end{tcolorbox}

La energía de un nivel es
\begin{equation}
    E = E_{n\ell} + E_{n'\ell'} + \Delta (n\ell,n'\ell',L,S) 
\end{equation}
donde $\Delta  = D \pm C$. El signo depende del observable
$S$,\footnote{Es positivo para $S=0$ y negativo para $S=1$} por
lo que se obtiene
\begin{align}
  E_{{}^1\!S} &= E_{11} + E_{21} + D(1121L) + C(1121L) \\
  E_{{}^3\!S} &= E_{11} + E_{21} + D(1121L) - C(1121L) \\
\end{align}
y por lo tanto $\Delta E({}^1,S{}^3S) = \SI{0.78}{\eV} = 2C$, de donde
obtenemos
\begin{equation}
  \boxed{
  C = \SI{0.39}{\eV}
  }
\end{equation}


\chapter{Enero, 2016 (3)}
\begin{tcolorbox}[halign=left]
  \emph{Calcule la vida media, en segundos, del nivel $2p_{\oh}$ del
    átomo de hidrógeno.}
\end{tcolorbox}

Del esquema de niveles (figura \ref{fig:finehidrogen}) podemos ver como la energía del hidrógeno entre
los niveles $1s_{\oh}$ y $2p_{\oh}$ es significativamente mayor que
la de las transiciones entre los niveles con $n=2$, por lo que aún sin
saber\jokenote{Efecto Lamb, obviamente} que el nivel $2p_{\oh}$ es el
de menor energía podemos notar que $\Lambda \sim k^3$ y aproximar
todas las demás $\Lambda$ a cero.

\begin{marginfigure}
\begin{tikzpicture}[xscale=1,yscale=2]
  \draw [color=blue, thick] (0,0) -- (2,0);
  \draw node [left] at (0,0) {$n=1$};
  \draw node [right] at (2,0) {$1s_{\oh}$};
  \draw [color=blue, thick] (0,1) -- (1.3,1);
  \draw node [left] at (0,1) {$n=2$};
  \draw [color=blue, thick] (1.3,1.2) -- (2,1.2);
  \draw node [right] at (2,1.2) {$2p_{\nicefrac{3}{2}}$};
  \draw [color=blue, thick] (1.3,1.05) -- (2,1.05);
  \draw node [right] at (2,1.05) {$2s_{\oh}$};
  \draw [color=blue, thick] (1.3,0.8) -- (2,0.8);
  \draw node [right] at (2,0.8) {$2p_{\oh}$};
\end{tikzpicture}
  \caption{Primeros niveles del hidrógeno, estructura fina.}
  \label{fig:finehidrogen}
\end{marginfigure}

Ese nivel está doblemente degenerado, al igual que el fundamental, por
el espín de la partícula. Obtenemos cuatro posibles transiciones desde
$\ket{n\ \ell \ s \ j \ m}$ hasta $\ket{n'\ \ell' \ s' \ j' \ m'}$:
\begin{center}
  \begin{tikzpicture}
    \tikzstyle energylevel=[thick,blue]
    \tikzstyle photon=[thick,yellow!60!black,snake=snake,line after snake=0.2cm,->]
    \tikzstyle fotonloco=[ultra thick,yellow!60!black,snake=snake,line after snake=0.2cm,->]
    \tikzstyle 
    photonlabel=[yellow!20!black,midway,fill=white]

    % labels
    \node [right] at (10,0.3) {$2p_{\oh}$};
    \node [right] at (10,-0.3) {$\ket{2\ 1\ \oh\ \oh \ m_i}$};
    \node [right] at (10,-2.7) {$1s_{\oh}$};
    \node [right] at (10,-3.3) {$\ket{1\ 0\ \oh\ \oh \ m_f}$};
    % 1/2, m_inicial
    \draw[energylevel] (3,0) -- (5,0);
    \node[fill=white] (a) at (4,0) {$\nicefrac{+1}{2}$};
    % -1/2, m_inicial
    \draw[energylevel] (6,0) -- (8,0);
    \node[fill=white] (b) at (7,0) {$\nicefrac{-1}{2}$};
    % +1/2, m_final
    \draw[energylevel] (1.5,-3) -- (4.5,-3);
    \node[fill=white] (c) at (3,-3) {$\nicefrac{+1}{2}$};
    % -1/2, m_final
    \draw[energylevel] (6.5,-3) -- (9.5,-3);
    \node[fill=white] (d) at (8,-3) {$\nicefrac{-1}{2}$};
    % transitions
    \draw[fotonloco] (a) -- (c) node[photonlabel] {$r_{0}^1$};
    \draw[photon] (a) -- (d) node[photonlabel] {$r_{-1}^1$};
    \draw[photon] (b) -- (c) node[photonlabel] {$r_{+1}^1$};
    \draw[photon] (b) -- (d) node[photonlabel] {$r_{0}^1$};
  \end{tikzpicture}
\end{center}
donde la $\zeta$ en $r_\zeta^1$ proviene de la regla $m_f = m_i +
\zeta$. Sabemos que $m_i$ se corresponde al espín y no tiene que ver
con el momento angular porque, como dice el subíndice del enunciado, $j=\oh$.

 Realizamos una integral (la transición marcada en linua gruesa
en el diagrama), las demás nos las dará el teorema de Wigner-Eckart.

Desacoplamos los kets:
\begin{align}
  \ket{2 \ 1 \ \oh \ \oh \ +\oh} \stackrel{1\otimes\oh \text{ CG}}{=} &
  \sqrt{\frac{2}{3}} \varphi_{211} \ket{-} -
  \sqrt{\frac{1}{3}}\varphi_{210}\ket{+} \\
  \ket{1 \ 0 \ \oh \ \oh \ +\oh} \stackrel{0\otimes\oh \text{ CG}}{=} &
   \varphi_{100} \ket{+} 
\end{align}
De las dos integrales $\mel{\text{F}}{r^1_\zeta}{\text{I}}$
correspondientes\footnote{$\mel{a+b}{W}{c}=\mel{a}{W}{c}+\mel{b}{W}{c}$}
aquella en la que el espín no es igual a ambos lados se anula, al
salir este del elemento de matriz y quedar $\braket{+}{-}=0$. En la
otra obtenemos:

\begin{equation}
    \mel{\varphi_{100}+}{r_0^1}{\varphi_{210}+} = \braket{+}{+}
    \mel{\varphi_{100}}{r_0^1}{\varphi_{210}} = \frac{\sqrt{2}\cdot2^7}{3^5}a_0
\end{equation}

Puede consultarse la resolución en el
ejercicio \ref{ej:integralloca}. Recordar el factor
$\sqrt{\frac{1}{3}}$ que llevaba el ket.

El teorema de Wigner-Eckart nos permite hallar la relación de este
elemento de matriz (${\mel{1\ 0\ \oh\ \oh\ +\oh}{r_0^1}{2\ 1\ \oh\ \oh\ +\oh}}$) con los demás:
\begin{fullwidth}
  \begin{align}
    \mel{1\ 0\ \oh\ \oh\ +\oh}{r_0^1}{2\ 1\ \oh\ \oh\ +\oh} 
    &=\underbrace{(\oh\ +\oh\ 1 \ 0 | \oh \ +\oh)}_{-1/\sqrt{3}} \mel{\oh}{|r^1|}{\oh} \simeq
      0.43 a_0 \\
    \mel{1\ 0\ \oh\ \oh\ -\oh}{r_{-1}^1}{2\ 1\ \oh\ \oh\ +\oh} 
    &=\underbrace{(\oh\ +\oh\ 1 \ -1 | \oh \ -\oh)}_{-\sqrt{\frac{2}{3}}} \mel{\oh}{|r^1|}{\oh}\\
    \mel{1\ 0\ \oh\ \oh\ +\oh}{r_{+1}^1}{2\ 1\ \oh\ \oh\ -\oh} 
    &=\underbrace{(\oh\ -\oh\ 1 \ +1 | \oh \ +\oh)}_{\sqrt{\frac{2}{3}}} \mel{\oh}{|r^1|}{\oh}\\
    \mel{1\ 0\ \oh\ \oh\ -\oh}{r_{0}^1}{2\ 1\ \oh\ \oh\ -\oh} 
    &=\underbrace{(\oh\ -\oh\ 1 \ 0 | \oh \ -\oh)}_{1/\sqrt{3}} \mel{\oh}{|r^1|}{\oh}
  \end{align}
\end{fullwidth}
Vemos que $\mel{\oh}{|r^1|}{\oh} \simeq -0.745a_0$, y por lo tanto
  \begin{align}
    \abs{\mel{1\ 0\ \oh\ \oh\ +\oh}{r_0^1}{2\ 1\ \oh\ \oh\ +\oh}}^2
    &\simeq 0.185  a_0^2 \\
    \abs{\mel{1\ 0\ \oh\ \oh\ -\oh}{r_{-1}^1}{2\ 1\ \oh\ \oh\ +\oh}}^2
    &\simeq 0.370 a_0^2\\
    \abs{\mel{1\ 0\ \oh\ \oh\ +\oh}{r_{+1}^1}{2\ 1\ \oh\ \oh\ -\oh}}^2
    &\simeq 0.370 a_0^2\\
    \abs{\mel{1\ 0\ \oh\ \oh\ -\oh}{r_{0}^1}{2\ 1\ \oh\ \oh\ -\oh}}^2
    &\simeq 0.185 a_0^2 
  \end{align}

Como las transiciones son posibles únicamente con una coordenada, no
hay que reagruparlas\footnote{Si una transición fuera posible con
  varios $r_\zeta^1$, sería necesario calcular
  $\abs{\mel{\star}{\boldrm{r}}{\star}}^2 =
  \abs{\mel{\star}{r_{+1}^1}{\star}}^2 +
  \abs{\mel{\star}{r_{0}^1}{\star}}^2 + \abs{\mel{\star}{r_{-1}^1}{\star}}^2$}.

Obtenemos
\begin{equation}
  \begin{split}
    \Lambda &= \frac{1}{2} \sum_{ij} \Lambda_{ij} \simeq \frac{1}{2}
    \frac{4e^2}{3 \hbar} k^3
    \underbrace{(0.185+0.370+0.370+0.185)}_{\sum
        \abs{\mel{\star}{r_i}{\star}}^2} a_0^2=\\
    &= \frac{2e^2}{3 \hbar} k^3 a_0^2 \times 1.1 \text{ s}
  \end{split}
\end{equation}
donde el $\frac{1}{2}$ proviene del número de estados iniciales. 
Realizando las sustituciones necesarias\footnote{
Empleamos que la $\lambda$ entre los niveles $1$ y $2$ del hidrógeno
es \SI{121}{\nano\metre} y que $k = \frac{2\pi}{\lambda}$.
}
en sistema cegesimal\footnotemark se
obtiene un valor $\tau = \Lambda^{-1}\simeq \SI{1.6}{\nano\second}$,
cercano al experimental.
\footnotetext{Basta con utilizar MKS tras multiplicar $\Lambda$ por un
prefactor $\frac{1}{4\pi\varepsilon_0}$}
\chapter{Junio, 2015 (1)}
\begin{tcolorbox}[halign=left]
  \emph{Una partícula de masa $m$ y espín $1$ está sometida a un
    potencial central $V(r)=\frac{1}{2}kr^2$, con $k=\text{cte.}$, y a
    una perturbación}
  \begin{equation*}
    - \frac{p^4}{8m^3c^2} + \frac{1}{2m^2c^2}\frac{1}{r} \dv{V}{r}
    \boldrm{\ell}\cdot \boldrm{s}
  \end{equation*}
  \emph{Calcule en primer orden de perturbaciones el número de niveles
  en los que se desdobla el primer nivel excitado y la separación
  entre ellos.}

\emph{Recuerde que el primer nivel excitado del problema sin perturbar
es una onda $p$ ($\ell=1$) con}
\begin{equation*}
  R(r) = \frac{8}{3}
  \frac{
    \beta^{\nicefrac{3}{2}}
  }{
    \pi^{\nicefrac{1}{4}}
  } \beta r e^{-\frac{1}{2}\beta^2 r^2}
\end{equation*}
\emph{donde $\beta^4 = km/\hbar^2$.}
\end{tcolorbox}

La diferencia entre niveles perturbados no dependerá del primer
sumando ($\sim p^4$) ya que sus elementos de matriz serán los mismos
para todas las funciones del subespacio de degeneración (sólo varían
en su $m$). Así pues, nos concentramos en el segundo sumando:
\begin{equation}
  \begin{split}
    \frac{1}{2m^2c^2} \frac{1}{r} \dv{V}{r}
    \boldrm{\ell} \cdot \boldrm{s} &\sim \boldrm{\ell} \cdot
    \boldrm{s} \\
    &\sim \frac{1}{2}(J^2 - L^2 -S^2)
  \end{split}
\end{equation}
donde se ha utilizado que $J^2 = (\boldrm{L}+\boldrm{S})^2$ y que
$\frac{1}{r}\dv{V}{r} = \frac{1}{r} kr=k$. Vemos los posibles valores
de $\boldrm{\ell}\cdot \boldrm{s}$ para las funciones de onda del
subespacio de degeneración:
\begin{center}
  \begin{tabular}{ccc|l}
    $J^2 \in 1\otimes 1$ & $L^2$ & $S^2$ & $\boldrm{\ell}\cdot
                                           \boldrm{s} = \frac{1}{2}(J^2-L^2-S^2)$
    \\ \hline
    $0$ & $1$ & $1$ & $\frac{1}{2}(0-1-1)=-1$ \\
    $1$ & $1$ & $1$ & $\frac{1}{2}(1-1-1)=\moh$ \\
    $2$ & $1$ & $1$ & $\frac{1}{2}(2-1-1)=0$ \\
  \end{tabular}
\end{center}

La matriz de perturbación será
\begin{equation}
  \ev{\frac{-p^4}{8m^3c^2}}{\varphi_0} \mathbb{I}  - \frac{k}{2m^2c^2}
  \begin{pmatrix}
    +1 & & \\
       &+\oh& \\
       & & 0
  \end{pmatrix}
\end{equation}

Obtenemos un desdoblamiento en tres niveles, con separación
$\frac{1}{2} \frac{k}{2m^2c^2}$ entre ellos y
degeneración\footnote{Por conmutar la perturbación con $J_z$} $2J+1$:
\begin{center}
  \begin{tikzpicture}[yscale=2]
    \tikzstyle energylevel=[ultra thick,blue]
    \tikzstyle transition=[thin,blue]
    \draw[energylevel] (0,0) -- (3,0);
    \node[above] at (1,0) {$E_0$};
    \draw[transition] (3,0) -- (4,-1);
    \draw[energylevel] (4,-1) -- (5,-1);
    \draw[thin,dashed,black] (3,0) -- (5,0);
    \draw[<->,black] (4.7,-1) -- (4.7,0) node 
    [midway, right] {$\langle\frac{-p^4}{8m^3c^2}\rangle$};
    \draw[transition] (5,-1) -- (6.5,-1);
    \draw[energylevel] (6.5,-1) -- (8,-1);
    \node[above] at (7,-1) {$J=2$};
    \draw[transition] (5,-1) -- (6.5,-1.5);
    \draw[transition] (5,-1) -- (6.5,-2);
    \draw[energylevel] (6.5,-1.5) -- (8,-1.5);
    \node[above] at (7,-1.5) {$J=1$};
    \draw[energylevel] (6.5,-2) -- (8,-2);
    \node[above] at (7,-2) {$J=0$};
    \draw[<->,black] (7.8,-1.5) -- (7.8,-2) node[midway, right] {$\frac{k}{4m^2c^2}$};
    \draw[<->,black] (7.8,-1) -- (7.8,-1.5) node[midway, right] {$\frac{k}{4m^2c^2}$};
    \draw[->,>=stealth,black] (-1,0) -- (-1,-2) node[midway, left] {$E$};
  \end{tikzpicture}
\end{center}


\chapter{Junio, 2015 (2)}
\begin{tcolorbox}[halign=left]
  \emph{Escriba la función de onda de los electrones de un átomo de
    helio en la configuración (1s)(2s) si uno de ellos apunta en la
    dirección $\hat{x}$ y el otro en la dirección $\hat{z}$.}
\end{tcolorbox}
Notando que $\ket{+}_x = \frac{1}{\sqrt{2}}(\ket{+}+\ket{-})$,
escribimos
\begin{align}
  \varphi_1 &= \varphi_{100}(r,\Omega) \ket{+} \\
  \varphi_2 &= \frac{1}{\sqrt{2}} \varphi_{200}(r,\Omega) (\ket{+} + \ket{-})
\end{align}
Y obtenemos
\begin{equation}
  \begin{split}
    \varphi_1(1)\otimes\varphi_2(2) &= \frac{1}{\sqrt{2}}
    \varphi_{100}(1)\varphi_{200}(2)\ket{+}_1\ket{+}_2 + \\
    &+ \frac{1}{\sqrt{2}}\varphi_{100}(1)\varphi_{200}(2)\ket{+}_1\ket{-}_2 = \\
    &= \frac{1}{\sqrt{2}}\varphi_{100}(1)\varphi_{200}(2) \left(\chi_1^1(1,2) +
    \frac{\chi_1^0(1,2)}{\sqrt{2}} + \frac{\chi_0^0(1,2)}{\sqrt{2}}
       \right)
  \end{split}
\end{equation}
donde se ha empleado la tabla de Clebsch-Gordan $1\otimes 1$.
Antisimetrizamos la función, como $\varphi_1\perp\varphi_2$ la
constante de normalización será simplemente $\sqrt{2}$:
\begin{equation}
  \begin{split}
    \sqrt{2}\Psi &= \varphi_1(1)\otimes\varphi_2(2) -
    \varphi_1(2)\otimes\varphi_2(1) = \\
    &= \frac{1}{\sqrt{2}}\varphi_{100}(1)\varphi_{200}(2) \chi_1^1(1,2) + \\
    &+ \frac{1}{2}\varphi_{100}(1)\varphi_{200}(2) \chi_1^0(1,2) + \\
    &+ \frac{1}{2}\varphi_{100}(1)\varphi_{200}(2) \chi_0^0(1,2) - \\
    &- \frac{1}{\sqrt{2}}\varphi_{100}(2)\varphi_{200}(1) \chi_1^1(2,1) - \\
    &- \frac{1}{2}\varphi_{100}(2)\varphi_{200}(1) \chi_1^0(2,1) - \\
    &- \frac{1}{2}\varphi_{100}(2)\varphi_{200}(1) \chi_0^0(2,1) 
  \end{split}
\end{equation}




\begin{tcolorbox}[halign=left]
  \emph{¿Qué posibles valores y probabilidades pueden obtenerse si se
    mide el espín total?}
\end{tcolorbox}

Como tenemos $\Psi$ en una base de espín total, no hay más que sumar
los coeficientes al cuadrado. Las $\varphi$ no son un problema, puesto
que su módulo al cuadrado en todo es espacio está normalizado.

Las probabilidades de medir cada espinor serán:
\begin{align}
  \text{P}(\chi_0^0) &= \abs{\frac{1}{2\sqrt{2}}}^2 +
                       \abs{\frac{1}{2\sqrt{2}}}^2 = \frac{1}{4} \\
  \text{P}(\chi_1^0) &= \abs{\frac{1}{2\sqrt{2}}}^2 +
                       \abs{\frac{1}{2\sqrt{2}}}^2 = \frac{1}{4} \\
  \text{P}(\chi_1^1) &= \abs{\frac{1}{\sqrt{2}\sqrt{2}}}^2 +
                       \abs{\frac{1}{\sqrt{2}\sqrt{2}}}^2 = \frac{1}{2}
\end{align}

\begin{tcolorbox}[halign=left]
  \emph{Calcule $\langle (\boldrm{r}_1-\boldrm{r}_2)^2\rangle$}
\end{tcolorbox}
Hay tres integrales a calcular, ya que $(\boldrm{r}_1-\boldrm{r}_2)^2
= r_1^2 + r_2^2 - \boldrm{r}_1 \cdot\boldrm{r}_2$. Utilizamos la base
desacoplada, y calculamos $\langle r_1^2\rangle$:
\begin{fullwidth}
  \begin{equation}
    \begin{split}
      \mel{\Psi}{r^2(1)}{\Psi} &= \mel{ \frac{1}{\sqrt{2}}
        \varphi_{100}(1)\varphi_{200}(2)\ket{+,1}\ket{+,2}_x
        - \frac{1}{\sqrt{2}}
        \varphi_{100}(2)\varphi_{200}(1)\ket{+,2}\ket{+,1}_x
      }{r^2(1)}{\text{ídem}} = \\
      &= \frac{1}{2}\mel{
        \varphi_{100}^{(1)}\varphi_{200}^{(2)}\ket{+,1}\ket{+,2}_x}{r^2(1)}{
        \varphi_{100}^{(1)}\varphi_{200}^{(2)}\ket{+,1}\ket{+,2}_x} -\\
      &- \frac{1}{2} \mel{
        \varphi_{100}^{(1)}\varphi_{200}^{(2)}\ket{+,1}\ket{+,2}_x}{r^2(1)}{
        \varphi_{100}^{(2)}\varphi_{200}^{(1)}\ket{+,2}\ket{+,1}_x} -\\
      &- \frac{1}{2} \mel{
        \varphi_{100}^{(2)}\varphi_{200}^{(1)}\ket{+,2}\ket{+,1}_x}{r^2(1)}{
        \varphi_{100}^{(1)}\varphi_{200}^{(2)}\ket{+,1}\ket{+,2}_x} +\\
      &+ \frac{1}{2}\mel{
        \varphi_{100}^{(2)}\varphi_{200}^{(1)}\ket{+,2}\ket{+,1}_x}{r^2(1)}{
        \varphi_{100}^{(2)}\varphi_{200}^{(1)}\ket{+,2}\ket{+,1}_x} = \cdots
    \end{split}
  \end{equation}
\end{fullwidth}
Obtenemos, como en la teoría, integrales iguales a parejas: la
\emph{integral de intercambio} (las cruzadas) y la \emph{integral
  directa} (los términos directos). Realizamos las integrales directas:
\begin{equation}
  \begin{split}
    &\frac{1}{2}\mel{
      \varphi_{100}^{(1)}\varphi_{200}^{(2)}\ket{+,1}\ket{+,2}_x}{r^2(1)}{\text{ídem}}
    = \\
    &= \frac{1}{2}
    \mel{\varphi_{100}^{(1)}}{r^2(1)}{\varphi_{100}^{(1)}}\braket{++_x}{++_x}\braket{\varphi_{200}^{(2)}}{\varphi_{200}^{(2)}}
    = \\
    &= \langle r^2\rangle_{100}
  \end{split}
\end{equation}
\begin{equation}
  \begin{split}
    &\frac{1}{2}\mel{
      \varphi_{100}^{(2)}\varphi_{200}^{(1)}\ket{+,2}\ket{+,1}_x}{r^2(1)}{\text{ídem}}
    = \\
    &= \frac{1}{2}
    \mel{\varphi_{200}^{(1)}}{r^2(1)}{\varphi_{200}^{(1)}}\braket{+_x+}{+_x+}\braket{\varphi_{100}^{(2)}}{\varphi_{100}^{(2)}}
    = \\
    &= \langle r^2\rangle_{200}
  \end{split}
\end{equation}
donde el término de espines es la
unidad\footnote{$\braket{+_x+}{+_x+}=\braket{+_x}{+_x}(1)\braket{+}{+}(2)
  = 1$, por ejemplo.}. Es inmediato que para $\langle r_2^2\rangle$ obtendremos los mismos
resultados, pero en orden inverso. Las integrales cruzadas son nulas,
ya que se tiene para el término espacial brakets del estilo $\braket{\varphi_{100}}{\varphi_{200}}$.

Hemos obtenido que $\langle r_1^2 + r_2^2\rangle = \left( \frac{1}{2}+\frac{1}{2} \right)\langle
r^2\rangle_{100}+ \left( \frac{1}{2}+\frac{1}{2} \right)\langle r^2\rangle_{200}$, ya sólo falta el término
$\langle \boldrm{r}_1 \boldrm{r}_2\rangle$. Sin necesidad de realizar
la integral, notamos que $ \boldrm{r}_1 \boldrm{r}_2$ es
una función impar y que la parte espacial de $\Psi$ es puramente par,
por lo que $\ev{\boldrm{r}_1 \boldrm{r}_2}{\Psi} =0$.

En definitiva,
\begin{equation}
  \boxed{
\langle (r_1 - r_2)^2 \rangle = \langle
r^2\rangle_{100}+ \langle r^2\rangle_{200}
  }
\end{equation}

Las integrales radiales resultan
\begin{equation}
  \begin{split}
    \langle r^2\rangle_{100} &= \mel{R_{10}Y_0^0}{r^2}{R_{10}Y_0^0}
    = \mel{R_{10}}{r^2}{R_{10}}\braket{Y_0^0}{Y_0^0} = \\
    &= \int_0^\infty r^2 \dd{r} \left[ \frac{2}{a_0^{3/2}}
      e^{-r/a_0} \right] r^2 \left[ \frac{2}{a_0^{3/2}}
      e^{-r/a_0} \right] = \\
    &= \frac{4}{a_0^3} \int_0^\infty r^4 e^{-2r/a_0} \dd{r} = \\
    &= \frac{4}{a_0^3}  \frac{4!}{(2/a_0)^5} = 3a_0^2
  \end{split}
  \label{eq:radial}
\end{equation}

\begin{equation}
  \begin{split}
    \langle r^2\rangle_{200} &= \mel{R_{20}Y_0^0}{r^2}{R_{20}Y_0^0}
    = \mel{R_{20}}{r^2}{R_{20}}\braket{Y_0^0}{Y_0^0} = \\
    &= \int_0^\infty r^2 \dd{r}  r^2
    \underbrace{\frac{4}{(2a_0)^3}e^{-r/a_0} \left( 1- \frac{r}{a_0}+\frac{r^2}{4a_0^2} \right)}_{R_{200}^2} = \\
    &= \frac{4}{(2a_0)^3} \left[ 
      \int_0^\infty r^4 e^{-r/a_0} \dd{r} +
      \int_0^\infty \frac{r^5}{a_0} e^{-r/a_0} \dd{r} +
      \int_0^\infty \frac{r^6}{4a_0^2} e^{-r/a_0} \dd{r} 
    \right] = \\
    &= \frac{4}{(2a_0)^3} \left[ 24a^5 + 120a^5 + 180a^5 \right] = 162a_0^2
  \end{split}
\end{equation}

\chapter{Junio, 2014 (1)}
\begin{tcolorbox}[halign=left]
  \emph{Dos partículas de espín $\oh$ están en un pozo de potencial
    tipo oscilador armónico isótropo. Si hay una perturbación
    $W=\alpha S_{1z} +\beta S_{2z}$, con $\alpha$ y $\beta$ constantes
    reales, calcule en primer orden de perturbaciones las correcciones a
    la energía del nivel fundamental en el caso de partículas
    idénticas y en el caso de partículas distintas.}
\end{tcolorbox}

\section*{Partículas idénticas}
Si las partículas son idénticas, $\alpha=\beta$ necesariamente.
Además, hemos de aplicar el postulado de simetrización y crear una
función de ondas completamente antisimétrica (por ser fermiones, si no
simétrica):
\begin{align}
  \varphi_1(1) &=\ket{0}^{(1)}\ket{\pm}^{(1)}\\
  \varphi_2(2) &=\ket{0}^{(2)}\ket{\pm}^{(2)}\\
  \Psi(1,2) &\propto \varphi_1(1)\varphi_2(2) \otimes
              \varphi_1(2)\varphi_2(1) = \ket{00}(\ket{\pm\pm}-\ket{\pm\pm})
\end{align}
Vemos que no todas las $\Psi$ (en principio cuatro) son posibles, y
que dos son idénticas salvo fase:
\begin{itemize}
  \item Para $\ket{+}\otimes\ket{+}$ obtenemos
    \begin{equation}
      \Psi(1,2) = \frac{1}{\sqrt{2}}\ket{00}(\ket{++}-\ket{++}) = 0 
    \end{equation}
    Es de esperar que se anule por ser $\varphi_1\otimes\varphi_2$
    completamente simétrica.
  \item Para $\ket{-}\otimes\ket{-}$ obtenemos el mismo resultado.
  \item Con los espines diferentes, obtenemos
    \begin{equation}
      \Psi(1,2) = \frac{1}{\sqrt{2}}\ket{00}(\ket{+-}-\ket{-+})  = \ket{00}\chi_0^0
    \end{equation}
    Si utilizamos $\ket{-}\otimes\ket{+}$ en lugar de
    $\ket{+}\otimes\ket{-}$ se obtiene la misma función pero con un
    factor de fase $e^{i\pi}$, por la antisimetría de $\ket{+-}$.
\end{itemize}

El postulado de antisimetrización ha eliminado toda la degeneración,
así que basta con calcular el único elemento de matriz que queda:
\begin{equation}
  \begin{split}
    \ev{\alpha(S_{1z}+S_{2z})}{00\chi_0^0} &=
    \braket{00}{00}\ev{S_{Tz}}{\chi_0^0} = \\ &= 1\cdot0 = 0
  \end{split}
\end{equation}

Obtenemos que no hay perturbación a la energía, en primer orden de aproximación.

\section*{Partículas distintas}
En este caso todas las combinaciones son posibles, al no haber ningún
postulado limitándonos ni ser necesario antisimetrizar:
\begin{align}
  \Psi_a &= \ket{00}\ket{++} \\
  \Psi_b &= \ket{00}\ket{+-} \\
  \Psi_c &= \ket{00}\ket{-+} \\
  \Psi_d &= \ket{00}\ket{--} 
\end{align}

Obtenemos una matriz de perturbación $4\times4$:
\begin{equation}
  \begin{pmatrix}
    \mel{\Psi_a}{W}{\Psi_a} & \mel{\Psi_a}{W}{\Psi_b} & \mel{\Psi_a}{W}{\Psi_c} & \mel{\Psi_a}{W}{\Psi_d} \\
    \mel{\Psi_b}{W}{\Psi_a} & \mel{\Psi_b}{W}{\Psi_b} & \mel{\Psi_b}{W}{\Psi_c} & \mel{\Psi_b}{W}{\Psi_d} \\
    \mel{\Psi_c}{W}{\Psi_a} & \mel{\Psi_c}{W}{\Psi_b} & \mel{\Psi_c}{W}{\Psi_c} & \mel{\Psi_c}{W}{\Psi_d} \\
    \mel{\Psi_d}{W}{\Psi_a} & \mel{\Psi_d}{W}{\Psi_b} & \mel{\Psi_d}{W}{\Psi_c} & \mel{\Psi_d}{W}{\Psi_d} \\
  \end{pmatrix} 
\end{equation}

Las partes espaciales producen brakets $\braket{00}{00}=1$, los
elementos de la diagonal son:
\begin{fullwidth}
  \begin{align}
    \mel{++}{\alpha S_1 + \beta S_2}{++} 
    &= \alpha\mel{+}{S_1}{+} + \beta \mel{+}{S_2}{+} = \frac{+\alpha
      \hbar}{2} + \frac{+\beta \hbar}{2} = \frac{\hbar
      (\alpha+\beta)}{2} \\
    \mel{+-}{\alpha S_1 + \beta S_2}{+-} 
    &= \alpha\mel{+}{S_1}{+} + \beta \mel{-}{S_2}{-} =
      \frac{-\hbar\alpha}{2}\braket{+}{+} +
      \frac{-\hbar\alpha}{2}\braket{-}{-} = \frac{\hbar(\alpha-\beta)}{2}\\
    \mel{-+}{\alpha S_1 + \beta S_2}{-+} 
    &= \alpha\mel{-}{S_1}{-} + \beta \mel{+}{S_2}{+} =
      \frac{-\hbar\alpha}{2}\braket{-}{-} +
      \frac{-\hbar\alpha}{2}\braket{+}{+} = \frac{\hbar(-\alpha + \beta)}{2}\\
    \mel{--}{\alpha S_1 + \beta S_2}{--} 
    &= \alpha\mel{-}{S_1}{-} + \beta \mel{-}{S_2}{-} = \frac{-\alpha
      \hbar}{2} + \frac{-\beta \hbar}{2} = \frac{-\hbar
      (\alpha+\beta)}{2} 
  \end{align}
\end{fullwidth}

La matríz completa, calculando el resto de elementos, es:
\begin{equation}
  \frac{\hbar}{2}
  \begin{pmatrix}
    \alpha+\beta & \alpha - \beta & \beta -\alpha & -\alpha -\beta \\
    \alpha & \alpha - \beta & 0 & -\beta \\
    \beta & 0 & -\alpha + \beta & -\alpha \\
    0 & -\beta & -\alpha & -\alpha - \beta
  \end{pmatrix}
\end{equation}
Tras algo de
álgebra\footnote{\url{https://www.wolframalpha.com/input/?i={{a\%2Bb,a,b,0},{a,a-b,0,-b},{b,0,-a\%2Bb,-a},{0,-b,-a,-a-b}}}}
vemos que la matríz sólo tiene dos autovalores no nulos, de valor
$\lambda_\pm = \pm 2\sqrt{\alpha^2+\beta^2}$. Obtenemos que la energía se divide en
dos niveles, distanciados $4\sqrt{\alpha^2+\beta^2}$.


\chapter{Junio, 2014 (2)}
\begin{tcolorbox}[halign=left]
  \emph{Una partícula sin espín, de masa $m$ y carga $q$ está
    confinada en una caja de paredes impenetrables entre
$x\in(\frac{-a}{2},\frac{a}{2})$, $y\in(\frac{-a}{2},\frac{a}{2})$ y
$z\in(\frac{-a}{10},\frac{a}{10})$.
Calcule en aproximación dipolar eléctrica la probabilidad de
transición del primer nivel excitado al fundamental.}
\end{tcolorbox}

De manera análoga al problema \ref{chapter:e161} tenemos un problema
prácticamente bidimensional, de forma que el primer excitado está
doblemente degenerado. Hay dos transiciones posibles (figura
\ref{fig:cajacontruenos}), calculamos una de ellas con la cordenada
cartesiana $x$ del
tensor $\boldrm{r}$, empleando que $\psi_n(x) = \sqrt{\frac{2}{L}}\sin
\left( \frac{n\pi}{L} [x+L/2] \right)$:

\begin{marginfigure}
  \begin{tikzpicture}
    \tikzstyle energylevel=[thick,blue] \tikzstyle
    photon=[thick,yellow!60!black,snake=snake,line after
    snake=0.2cm,->] \tikzstyle fotonloco=[ultra
    thick,yellow!60!black,snake=snake,line after snake=0.2cm,->]
    \tikzstyle photonlabel=[yellow!20!black,midway,fill=white]

    \draw[energylevel] (0,2) -- (2,2); \node[fill=white] (a) at
    (1,2) {$\ket{21}$}; \draw[energylevel] (3,2) -- (5,2);
    \node[fill=white] (b) at (4,2) {$\ket{12}$}; \draw[energylevel]
    (1,0) -- (4,0); \node[fill=white] (g) at (2.5,0) {$\ket{11}$};
    % transitions
    \draw[photon] (a) -- (g); \draw[photon] (b) -- (g);
  \end{tikzpicture}
  \caption{Caja con truenos}
  \label{fig:cajacontruenos}
\end{marginfigure}

\begin{equation}
  \begin{split}
    \mel{11}{x}{21} &= \mel{1}{x}{2}\braket{1}{1} = \\
    &= \frac{2}{a}\int_{-a/2}^{a/2} \sin \left( \frac{\pi}{a}(x+a/2)
    \right) x \sin \left( \frac{2\pi}{a}(x+a/2) \right) \dd{x} = \\
    &= \frac{2}{a} \int_0^a \sin \left( \frac{\pi x}{a} \right)
    (x-a/2) \sin \left( \frac{2\pi x}{a} \right) \dd{x} = \\
    &= \cdots = \frac{-16a}{9\pi^2}
  \end{split}
\end{equation}

La integral se resuelve mediante el uso de identidades
trigonométricas. La transición con $z$ es nula por ser $z\simeq 0$ en
toda la caja, y con $y$ por quedar un braket con elementos ortogonales
de la base:
\begin{equation}
  \mel{11}{y}{21} = \braket{1}{2} \mel{1}{y}{1} = 0 \cdot
  \mel{1}{y}{1} = 0
\end{equation}

Obtenemos
\begin{equation}
  \abs{\mel{11}{\boldrm{r}}{21}} = \abs{\mel{11}{x}{21}}^2 +
  \abs{\mel{11}{y}{21}}^2 + \abs{\mel{11}{z}{21}}^2 \simeq 0.0324 a^2
\end{equation}

Para la transición $\mel{11}{\boldrm{r}}{12}$ se obtiene el mismo
resultado, pero la integral no nula es la de la $y$. Obtenemos una
probabilidad de transición
\begin{equation}
  \Lambda = \frac{1}{2} \sum \Lambda_{ij} = \frac{2e^2}{3 \hbar}k^3
  \underbrace{(0.0324+0.0324)a^2}_{\sum
    \abs{\mel{\text{F}}{\boldrm{r}}{\text{I}}}} \stackrel{\text{CGS}}{=} \SI{3e16}{\second}
\end{equation}

donde se ha utilizado que $k(n)=\frac{n\pi}{L}$ para un pozo cuadrado
infinito y multiplicado el resultado por $\frac{1}{4\pi\varepsilon_0}$
para pasar a sistema MKS. Para $L$ se ha tomado \SI{1}{\angstrom}.

\begin{tcolorbox}[halign=left]
  \emph{La transición desde el segundo nivel excitado al nivel
    fundamental, ¿es más probable, menos probable o igual de probable
    que la del apartado anterior? ¿Por qué?}
\end{tcolorbox}

Es menos probable, ya que su $\Lambda$ es nula y por tanto la vida
media infinita. Esto se debe a que las transiciones entre niveles
con simetría par (el fundamental y el segundo excitado lo cumplen\footnote{Se puede
  ver rápidamente dibujándolos que todos los niveles con $n$ impar tienen
funciones de onda pares}) se anulan por ser
el elemento de matriz $\mel{\text{par}}{\text{impar}}{\text{par}}=0$.


\chapter{Junio, 2014 (3)}
\begin{tcolorbox}[halign=left]
  \emph{Escriba una función de ondas de dos electrones en una
    configuración (2p)\textsuperscript{2} con la tercera componente
    del momento angular orbital total nula y con ambos espines
    apuntando en la dirección del eje $\hat{x}$. ¿Cuál es el momento
    angular total de los dos electrones?}
\end{tcolorbox}

La parte de espín y la espacial quedan fijadas como $R_{21}\ket{+}_x$
por el enunciado. Queda ver qué parte angular escoger para cada función.

Hay dos combinaciones de armónicos esféricos que nos dan ${L_{Tz}=0}$,
\begin{itemize}
\item Podemos utilizar en ambos electrones $Y_1^0$, pero entonces
  ambas funciones son idénticas:
  \begin{equation}
    \varphi = R_{21}(r)Y_1^0(\Omega)\ket{+}_x
  \end{equation}
  Esto esta prohibido en general por el postulado de simetrización
  ($\varphi\otimes\varphi'$ es puramente simétrica y no se puede
  antisimetrizar) y en particular por el principio de exclusión de
  Pauli (ambos electrones tienen exactamente los mismos números
  cuánticos y ocupan por lo tanto el mismo estado).
\item Podemos utilizar en una partícula $Y_1^{+1}$ y en la otra
  $Y_1^{-1}$. El orden en que utilicemos los coeficientes da igual, al
  antisimetrizar $\Psi$ simplemente obtendremos una fase $e^{i\pi}$ o
  no.
  \begin{equation}
    \begin{split}
      \Psi &=
      \frac{1}{\sqrt{2}}R_{21}(1)R_{21}(2)\ket{+}_x^{(1)}\ket{+}_x^{(2)}
      \left[ Y_1^{+1}(1)  Y_1^{-1}(2) - Y_1^{-1}(1)Y_1^{+1}(2)
        \right] = \\
        &=F_{(2p)^2} \smqty{0\\ 1} \ket{+}_x^{(1)}\ket{+}_x^{(2)}
      \end{split}
  \end{equation}
\end{itemize}

El momento orbital $L$ será la unidad\footnote{Puede comprobarse
  desarrollando $Y_1^{+1}(1)  Y_1^{-1}(2) - Y_1^{-1}(1)Y_1^{+1}(2)$
  con una tabla de Clebsch-Gordan. El único coeficiente que no se
  simplifica en la expresión es el $\ket{1\ 0}$.}. A continuación,
desarrollamos $\ket{+}_x\ket{+}_x$ en una base de espín total con los
coeficientes de Clebsch-Gordan:
\begin{equation}
  \begin{split}
    \ket{+}_x^{(1)}\ket{+}_x^{(2)} &= \frac{1}{2} \left( \ket{+}^{(1)} -
      \ket{-}^{(1)} \right)\left( \ket{+}^{(2)} + \ket{-}^{(2)}
    \right)= \\ &= \frac{1}{2}\ket{++} + \frac{1}{2}\ket{+-} +
    \frac{1}{2}\ket{-+}  + \frac{1}{2}\ket{++}  = \\
    &= \frac{1}{2} \chi_1^1 + \frac{1}{2} 
    \left( \frac{\chi_1^0}{\sqrt{2}} + \frac{\chi_0^0}{\sqrt{2}} \right)
    + \frac{1}{2}
    \left( \frac{\chi_1^0}{\sqrt{2}} - \frac{\chi_0^0}{\sqrt{2}} \right)
    + \frac{1}{2} \chi_1^{-1} = \\
    &= \frac{1}{2} \chi_1^1 + \frac{1}{\sqrt{2}} \chi_1^0 +
    \frac{1}{2} \chi_1^{-1}
  \end{split}
\end{equation}

Ya sólo falta acoplar el momento orbital $F_{(2p)^2}\smqty{M_L:\ 0 \\ L_T:\ 1}$
con el momento de espín recién hallado, utilizamos la notación
$\ket{L\ L_z}$ y $\ket{S\ S_z}$:
\begin{equation}
  \begin{split}
    \Psi &= R_{21}(1)R_{21}(2) \left[ \textcolor{red!60!black}{\ket{1\ 0}} \right] \otimes
    \left[  \textcolor{green!60!black}{
      \frac{1}{2}\ket{1\ +1} + \frac{1}{\sqrt{2}}\ket{1\ 0} +
      \frac{1}{2}\ket{1\ {-1}}
      }
    \right] = \\
    &= R_{21}(1)R_{21}(2) \left[ \frac{
        \textcolor{red!60!black}{\ket{1\ 0}}
          \textcolor{green!60!black}{\ket{1\ +1}}
        }{2} +
      \frac{
        \textcolor{red!60!black}{\ket{1\ 0}}
       \textcolor{green!60!black}{ \ket{1\ 0}}
      }{\sqrt{2}} + \frac{
       \textcolor{red!60!black}{ \ket{1\ 0}}
        \textcolor{green!60!black}{\ket{1\ -1}}
      }{2} \right] = \\
    &= R_{21}(1)R_{21}(2) \frac{1}{2}\left( 
      \textcolor{blue!60!black}{\frac{\ket{2\ +1}}{\sqrt{2}} - \frac{\ket{1\ +1}}{\sqrt{2}} }
    \right) + \\
    &+ R_{21}(1)R_{21}(2) \frac{1}{\sqrt{2}} \left(
      \textcolor{blue!60!black}{
        \sqrt{\frac{2}{3}}\ket{2\ 0} - \frac{\ket{0\ 0}}{\sqrt{3}}
      }
      \right) + \\
    &+ R_{21}(1)R_{21}(2) \frac{1}{2} \left(
      \textcolor{blue!60!black}{\frac{\ket{2\ -1}}{\sqrt{2}} - \frac{\ket{1\ -1}}{\sqrt{2}}}
      \right) \\
  \end{split}
\end{equation}
donde se ha utilizado la tabla de Clebsch-Gordan $1\otimes1$ para
acoplar kets $\textcolor{red!60!black}{L=1}$ con kets
$\textcolor{green!60!black}{S=1}$ en una base de
$\textcolor{blue!60!black}{J\in1\otimes1=\{0,1,2\}}$. Por ejemplo, en
$\textcolor{red!60!black}{\ket{1\ 0}}
\textcolor{green!60!black}{\ket{1\ +1}}$ Miramos $m_1,m_2=\{0,+1\}$.

En esta base, la probabilidad de medir $J=2$ es la suma de los
coeficientes $\ket{2 \ m}$, la de mediar $J=1$ la de los coeficientes
de los $\ket{1\ m}$ y $J=0$ la de los $\ket{0\ m}$:
\begin{center}
  \begin{tabular}{lcr}
    $J$ & Coeficientes $C_i$ & $\text{P}=\sum \abs{C_i}^2$ \\ \hline
    $2$ & $\frac{1}{2\sqrt{2}}, \frac{1}{\sqrt{2}}\frac{\sqrt{2}}{\sqrt{3}},\frac{1}{2\sqrt{2}}$ & $\frac{7}{12} \simeq 58\%$ \\
    $1$ & $\frac{-1}{2\sqrt{2}},\frac{-1}{2\sqrt{2}}$ & $\frac{1}{4}\simeq 25\%$ \\
    $0$ & $\frac{1}{\sqrt{2}\sqrt{3}}$ & $\frac{1}{6} \simeq 17\%$ \\
    & & \emph{Total: }$100\%$
  \end{tabular}
\end{center}
Notar que se ha utilizado que la parte radial está normalizada al
integrar a todo el espacio al calcular $\abs{C_i}^2$. Vemos que no
obtenemos un único momento angular total al medir, sino varios con
distintas probabilidades.

\chapter{Septiembre, 2014 (1)}
\begin{tcolorbox}[halign=left]
  \emph{Un átomo de hidrógeno está sometido a una perturbación
    $V=\alpha(L_z+2S_z) + \beta(x^2+y^2)$ con $\alpha,\beta\in
    \mathbb{R}$. Calcule en primer orden de perturbaciones la
    corrección a la energía del nivel fundamental.}
\end{tcolorbox}

El nivel fundamental posee una doble degeneración por el espín, así
que las matrices de perturbación serán $2\times2$ en el subespacio de degeneración.

El nivel fundamental no posee momento angular, así que 
$\mel{\varphi_{100}}{L_z}{\varphi_{100}}=0$ en todo el subespacio de
degeneración:
Para $S_z$ obtenemos
$\mel{\varphi_{100}\pm}{S_z}{\varphi_{100}\pm}=\ip{\varphi_{100}}\mel{\pm}{S_z}{\pm}$, donde $S_z =
\frac{\hbar}{2} \smqty(1& \\ & -1)$. Obtenemos
\begin{equation}
  \mel{\Psi}{2\alpha S_z}{\Psi} = \alpha \hbar
  \begin{pmatrix}
    1 & \\ & -1 \\
  \end{pmatrix}
\end{equation}

Para calcular el otro término perturbativo, escribimos $x^2+y^2 =
r^2-z^2 = r^2(1-\cos^2\theta)$:
\begin{equation}
  \begin{split}
    \mel{\varphi_{100}}{r^2(1-\cos\theta)}{\phi_{100}} &=
    \ev{r^2}{\varphi_{100}}\ev{1-\cos^2\theta}{Y_0^0} = \\
    &= \langle r^2\rangle_{10} \int_0^{2\pi} \dd{\varphi}\int_0^\pi
    \dd{\theta}\sin\theta (1-\cos^2\theta)
    \underbrace{\abs{Y_0^0}^2}_{=1/4\pi} = \\ &= \cdots =\frac{2}{3} \langle
    r^2\rangle_{10}
  \end{split}
\end{equation}

El valor de $\langle r^2 \rangle_{10}$ es $3a_0^2$, como ya se calculó
en la ecuación \eqref{eq:radial}. La integral no depende del espín,
así que es constante en todo el subespacio de degeneración. Los
términos con fuera de la diagonal ($\braket{\pm}{\mp}$) serán nulos.

Juntando los tres términos,
obtenemos
\begin{equation}
  \mel{\Psi}{V}{\Psi} =
  \begin{pmatrix}
    0+\alpha \hbar +2\beta a_0^2 & \\
    & 0-\alpha \hbar +2\beta a_0^2  \\
  \end{pmatrix}
\end{equation}

El nivel sube a $E = E_0 +2\beta a_0^2$, y se desdobla en dos niveles
no degenerados separados $2\alpha \hbar$, como puede verse en la
figura \ref{fig:levelfthis}
\begin{marginfigure}
  \begin{tikzpicture}[xscale=0.6, yscale=2]
    \tikzstyle energylevel=[ultra thick,blue]
    \tikzstyle transition=[thin,blue]
    \draw[energylevel] (0,0) -- (3,0);
    \node[above] at (1,0) {$E_0$};
    \draw[transition] (3,0) -- (4,+1);
    \draw[energylevel] (4,+1) -- (5,+1);
    \draw[thin,dashed,black] (3,0) -- (5,0);
    \draw[<->,black] (4.7,+1) -- (4.7,0) node 
    [midway, right] {$2\beta a_0^2$};
    \draw[transition] (5,+1) -- (6.5,+1.5);
    \draw[transition] (5,+1) -- (6.5,0.5);
    \draw[energylevel] (6.5,+1.5) -- (8,+1.5);
    \draw[energylevel] (6.5,0.5) -- (8,0.5);
    \draw[<->,black] (7.8,+0.5) -- (7.8,+1.5) node[midway, right]
    {$2\alpha \hbar$};
  \end{tikzpicture}
  \caption{Efecto de la perturbación}
  \label{fig:levelfthis}
\end{marginfigure}

\chapter{Septiembre, 2014 (2)}
\begin{tcolorbox}[halign=left]
  \emph{Sea $\Phi_0$ el estado fundamental del átomo de helio y $\Phi$
  un estado de la configuración excitada (1s)(2p).}

\emph{Escoja $\Phi$ de forma que sea autoestado de los observables
  $L^2,S^2,J^2,J_z$ y además $\mel{\Phi_0}{z_1+z_2}{\Phi}\neq 0$}
\end{tcolorbox}

Utilizando la notación $\ket{(n\ell)\ (n'\ell')\ L\ S\ J\ M}$,
escribimos el nivel fundamental como el $\ket{(1s)^2\ 0\ 0\ 0\ 0}$. El
estado excitado propuesto será el $\ket{(1s)(2p)\ L\ S\ J\ M}$:
\begin{itemize}
\item El momento angular orbital total estará en $S\otimes P=0\otimes
  1 = \{1\}$, así que $L=1$.
\item Las transiciones dipolares han de cumplir $\Delta S = 0$; el
  estado fundamental tiene $S=0$ así que el excitado también ha de
  tener espín nulo.
\item Tenemos $L=1$ y $S=0$, así que $J\in1\otimes0=\{1\}$.
\item Para $M$ tendremos como posibles valores $0,\pm 1$. Como la
  transición la realiza $z_1+z_2 = R_0^1$ se ha de cumplir $M_i + 0 =
  M_f$, donde el $0$ proviene del subíndice del $R_0^1$. $M_f$ es la
  del fundamental, que es nula, así que $M$ será nula también para el excitado.
\end{itemize}

Obtenemos:
\begin{equation}
  \Phi = \ket{(1s)(2p)\ 1\ 0 \ 1 \ 0}
\end{equation}

\begin{tcolorbox}[halign=left]
  \emph{Calcule el elemento de matriz anterior tomando funciones de
    onda radiales del átomo de hidrógeno con carga $Z=2$.}
\end{tcolorbox}

El primer paso es desacoplar las funciones de onda.
El nivel fundamental es trivial, ya que $J$
proviene exclusivamente del espín y la parte espacial es puramente simétrica:
\begin{equation}
  \begin{split}
    \Phi_0 &= \ket{(1s)^2\ 0\ 0\ 0\ 0} =
    R_{10}(1)R_{10}(2)Y_0^0(1)Y_0^0(2)\chi_0^0 = \\
    &= \varphi_{100}(1)\varphi_{100}(2) \chi_0^0
  \end{split}
\end{equation}

En el excitado tenemos de nuevo $S=0$, así que $J$ proviene
exclusivamente del momento angular orbital. Como el orbital $s$ aporta
momento orbital nulo, y la tercera componente tiene que ser nula
($M=0$), el armónico esférico del orbital $p$ tiene que ser $Y_1^0$ de
entre todos los posibles\footnote{A priori,
  $\{Y_0^0,Y_0^{+1},Y_0^{-1}\}$}.
Obtenemos $\varphi_1 = R_{10}Y_0^0$ y $\varphi_2=R_{21}Y_1^0$, de
forma que la función de ondas del sistema será (salvo normalización)
$(\varphi_1(1)\otimes\varphi_2(2)+
\varphi_1(2)\otimes\varphi_2(1))\chi_0^0$ al acoplar junto al
espín\footnote{Normalmente se utiliza $\varphi_1(1)\otimes\varphi_2(2)
  - \varphi_1(2)\otimes\varphi_2(1)$, pero en esos casos estamos
metiendo en $\varphi$ también el espín, así que directamente
antisimetrizamos. En este caso se estan simetrizando por separado la
parte espacial y el espín, y como este último ($\chi_0^0$) es ya
antisimétrico necesitamos una parte espacial simétrica, no
antisimétrica, de forma que $\Psi$ sea antisimétrica.}.

\begin{equation}
  \Phi = \ket{(1s)(2p)\ 1\ 0\ 1\ 0} = [\varphi_{100}(1)\varphi_{210}(2)
   + \varphi_{210}(1)\varphi_{100}(2)]
  \chi_0^0
\end{equation}

Empezamos por la primera integral, $z_1$:
  \begin{equation}
    \begin{split}
      &\mel{\varphi_{100}(1)\varphi_{100}(2)\chi_0^0}{z_1}{[\varphi_{100}(1)\varphi_{210}(2)
        + \varphi_{210}(1)\varphi_{100}(2)] \chi_0^0
      } = \\
      &= \ip{\chi_0^0}
      \mel{\varphi_{100}(1)\varphi_{100}(2)}{z_1}{\varphi_{100}(1)\varphi_{210}(2)}
      + \\
      &+\ip{\chi_0^0}
      \mel{\varphi_{100}(1)\varphi_{100}(2)}{z_1}{\varphi_{210}(1)\varphi_{100}(2)}
      = \\
      &= 
      \mel{\varphi_{100}}{z}{\varphi_{100}}\cancelto{0}{\braket{\varphi_{100}}{\varphi_{210}}}
      +\mel{\varphi_{100}}{z}{\varphi_{210}}\braket{\varphi_{100}}{\varphi_{100}}
      = \\
      &= \mel{\varphi_{100}}{r_0^1}{\varphi_{210}} =
      \mel{\varphi_{100}}{\sqrt{\frac{4\pi}{3}}rY_1^0}{\varphi_{210}}
      = \\
      &= \sqrt{\frac{4\pi}{3}}\int_0^\infty r^2 \dd{r} R_{10} r R_{21}
      \underbrace{\int_{4\pi}  \dd{\Omega} Y_0^0 Y_1^0
        Y_1^0}_{=\frac{1}{\sqrt{4\pi}}\int \dd{\Omega} \abs{Y_1^0}^2 =
        \frac{1}{\sqrt{4\pi}}} = \\
      &= \frac{1}{\sqrt{3}} \int_0^\infty r^2 \dd{r}
      \frac{2}{a_0^{3/2}} e^{-r/a_0} r \frac{1}{\sqrt{3}(2a_0)^{3/2}}
      \frac{r}{a_0} e^{\frac{-r}{2a_0}} = \\
      &= \frac{\sqrt{2}}{6a_0^4} \int \dd{r} r^4 e^{\frac{-2r}{3a_0}}
      = \\
      &= \frac{\sqrt{2}\cdot3^5}{2^3} a_0
    \end{split}
  \end{equation}

  En $z_2$ obtenemos el mismo resultado:

  \begin{equation}
    \begin{split}
      &\mel{\varphi_{100}(1)\varphi_{100}(2)\chi_0^0}{z_2}{[\varphi_{100}(1)\varphi_{210}(2)
        + \varphi_{210}(1)\varphi_{100}(2)] \chi_0^0
      } = \\
      &= \ip{\chi_0^0}
      \mel{\varphi_{100}(1)\varphi_{100}(2)}{z_2}{\varphi_{100}(1)\varphi_{210}(2)}
      + \\
      &+ \ip{\chi_0^0}
      \mel{\varphi_{100}(1)\varphi_{100}(2)}{z_2}{\varphi_{210}(1)\varphi_{100}(2)}
      = \\
      &= 
      \braket{\varphi_{100}}{\varphi_{100}}\mel{\varphi_{100}}{z}{\varphi_{210}}
      +\cancelto{0}{\braket{\varphi_{100}}{\varphi_{210}}}\mel{\varphi_{100}}{z}{\varphi_{100}}
      = \\
      &= \cdots = \frac{\sqrt{2}\cdot3^5}{2^3} a_0
    \end{split}
  \end{equation}

  En resumen,
  \begin{equation}
    \boxed{
      \mel{\Phi_0}{z_1+z_2}{\Phi} = \frac{\sqrt{2}\cdot 3^5}{4}a_0
    }
  \end{equation}

\chapter{Examen extra (1)}
\begin{tcolorbox}[halign=left]
  \emph{Un electrón está confinado en un cubo de paredes
    impenetrables y lado $L$ con una de sus caras sobre la
superficie terrestre. Suponga que $L$ es mucho menor que el radio de
la tierra, de forma que la energía potencial gravitatoria del electrón
es $V(x,y,z)=mgz$, donde $z$ es la altura respecto a la superficie
terrestre.}

\emph{Calcule en primer orden de perturbaciones las correcciones de la
interacción gravitatoria a las energías del nivel fundamental y del
primer excitado del electrón.}
\end{tcolorbox}

\section*{Nivel fundamental}



En el nivel fundamental obtenemos integrales del estilo de
\begin{equation}
  \begin{split}
    \Delta E &= \mel{111}{mgz}{111} = mg \ip{11}{11} \mel{1}{z}{1} =
    \\
    &= mg \int_0^L \dd{z} \sqrt{\frac{2}{L}} \sin \left( \frac{\pi z}{L} 
    \right) z \sqrt{\frac{2}{L}} \sin \left( \frac{\pi z}{L} \right)  =\\ 
&= \frac{2mg}{L} \int_0^L \dd{z} \sqrt{\frac{2}{L}} \sin^2 \left(
  \frac{\pi z}{L} \right) = \\
&= \frac{2mg}{L} \int_0^L \dd{z}\frac{z - z\cos \left( \frac{2\pi z}{L}
  \right)}{2} = \\ &= \frac{mg}{L}  
\left[ \frac{z^2}{2} - \frac{zL}{2\pi}\sin(\frac{2\pi z}{L}) - \frac{L}{2\pi}
  \cos(\frac{2\pi z}{L}) \right]_0^L  = \\
&= \frac{mg}{L} \left[ \frac{L^2}{2} + 0 \right] = \frac{mgL}{2}
  \end{split}
\end{equation}
donde se ha utilizado que $\int x\cos ax \dd{x} = \frac{x}{a} \sin(ax)
+ \frac{1}{a^2}\cos(ax) $ (integración por partes).
El valor de la integral no depende del espín, así que obtenemos para
el subespacio de degeneración\footnote{Hay degeneración doble por el
  espín}
\begin{equation}
  \begin{pmatrix}
    \mel{111+}{V}{111+} &  \mel{111+}{V}{111-} \\
    \mel{111-}{V}{111+} &  \mel{111-}{V}{111-}
  \end{pmatrix} =
  \frac{mgL}{2}
  \begin{pmatrix}
    1 &  \\ & 1
 \end{pmatrix}
\end{equation}

Los dos niveles tienen la misma corrección\footnote{La matriz tiene
  ambos autovalores idénticos}, de valor $\frac{1}{2}mgL$, así que no
se desdoblan.

\section*{Primer excitado}
A la degeneración del espín se suma una triple degeneración debida a
la simetría de la caja:

\begin{center}
  \begin{tabular}{c|ccc}
    $E$&$n_1$&$n_2$&$n_3$ \\
    $1+1+1=3$ & $1$ & $1$ & $1$ \\ \hline
    $2+1+1=4$ & $2$ & $1$ & $1$ \\
    $1+2+1=4$ & $1$ & $2$ & $1$ \\ 
    $1+1+2=4$ & $1$ & $1$ & $2$ \\ \hline
    $3+1+1=5$ & $3$ & $1$ & $1$ \\
    $1+3+1=5$ & $1$ & $3$ & $1$ \\
    $1+1+3=5$ & $1$ & $1$ & $3$ \\
    $1+2+2=5$ & $1$ & $2$ & $2$ \\
    $2+2+1=5$ & $2$ & $2$ & $1$ \\ \hline
    $\vdots$ & $\vdots$ & $\vdots$ & $\vdots$ \\ 
  \end{tabular}
\end{center}

Las integrales $\ev{V}{211}$, $\ev{V}{121}$ y todas aquellas con
función en $z$ $\ket{1}$ darán el mismo
resultado que antes, porque se
reducen a calcular $\ev{V}{1}$. Para $\ev{V}{2}$
obtenemos:
\begin{equation}
  \begin{split}
    \ev{V}{2} &= \frac{2mg}{L}\int_0^L \dd{z} z \sin^2 \left(
      \frac{2\pi
        z}{L} \right) = \\
    &= \cdots = \frac{mgL}{2}
  \end{split}
\end{equation}

El resultado es el mismo que en la integral anterior porque el término
que cambia al desarrollar el $\sin^2(x)$ se anula.

La integral $\mel{2}{V}{1}$ es innecesaria porque, como veremos, esos términos resultan nulos por los factores que les acompañan.

Los bloques
de fuera de la diagonal se anulan por ser los espines del braket
ortogonales, y en la diagonal tendremos dos bloques $3\times 3$
\begin{equation}
  \begin{pmatrix}
    \mel{211}{V}{211} &  \mel{211}{V}{121} & \mel{211}{V}{112} \\
    \mel{121}{V}{211} &  \mel{121}{V}{121} & \mel{121}{V}{112} \\
    \mel{112}{V}{211} &  \mel{112}{V}{121} & \mel{112}{V}{112} 
  \end{pmatrix}
\end{equation}

Vemos que los términos fuera de la diagonal del bloque son nulos, porque la parte en $\hat{x},\hat{y}$ de los brakets nos da factores del estilo $\braket{i}{j}$, que se anulan por la ortogonalidad de los elementos de la base.

Utilizando las integrales que acabamos de realizar, podemos calcular la
expresión explícita de la matriz completa como $\frac{1}{2}mgL \cdot \mathbb{I}_{\ 6\times 6}$

Vemos que nuevamente no hay desdoblamiento.

\begin{tcolorbox}[halign=left]
  \emph{Estime para qué valor de $L$ puede decirse que el primer orden
  de perturbaciones ya no es una buena aproximación. ¿Es entonces $L$
  mucho menor que el radio de la tierra?}
\end{tcolorbox}

Imponemos que $\Delta E$ sea como máximo un $10\%$ del $E$, y
obtenemos para el nivel fundamental ($E_{111} = 3\cdot\frac{\hbar^2
  \pi^2}{2m L^2}$):
\begin{equation}
  \frac{\Delta E}{E_{111}} = \frac{m^2 g L}{3 \hbar^2 \pi^2} \simeq
  25\cdot10^6 L < 0.1
\end{equation}
Por lo tanto obtenemos $L < \SI{4}{\nm}$. Es mucho menor que el radio
de la tierra, pero muy grande en comparación con las distancias
nucleares típicas.


\chapter{Examen extra (2)}
\begin{tcolorbox}[halign=left]
  \emph{Dos partículas idénticas sin espín $(S=0)$ están en un
    potencial central en una configuración (2p)\textsuperscript{2}.}

  \emph{Calcule la degeneración de esta configuración.}
\end{tcolorbox}
Sus funciones de onda, antes de antisimetrizar, serán del estilo $\varphi_m = R_{21}Y_1^m$. Como $m\in\{-1,0,1\}$, obtenemos varias combinaciones posibles para 
$\Psi_{m,m'} = \varphi_{m}(1)\otimes\varphi_{m'}(2) - \varphi_{m}(2)\otimes\varphi_{m'}(1)$:
\begin{align}
  \Psi_{+1,+1} &= 0 \ \ (\varphi=\varphi') \\
  \Psi_{+1,0} &=\frac{1}{\sqrt{2}} R_{21}(1)R_{21}(2) [Y_1^{+1}(1)Y_1^0(2)-Y_1^{0}(1)Y_1^{+1}(2)] \\
  \Psi_{+1,-1} &=\frac{1}{\sqrt{2}} R_{21}(1)R_{21}(2) [Y_1^{+1}(1)Y_1^{-1}(2)-Y_1^{-1}(1)Y_1^{+1}(2)] \\
  \Psi_{0,+1} &=\frac{1}{\sqrt{2}} R_{21}(1)R_{21}(2) [Y_1^{0}(1)Y_1^{+1}(2)-Y_1^{+1}(1)Y_1^{0}(2)] \\
  \Psi_{0,0} &= 0 \ \ (\varphi=\varphi') \\
  \Psi_{0,-1} &=\frac{1}{\sqrt{2}} R_{21}(1)R_{21}(2) [Y_1^{0}(1)Y_1^{-1}(2)-Y_1^{-1}(1)Y_1^{0}(2)] \\
  \Psi_{-1,+1}&=\frac{1}{\sqrt{2}} R_{21}(1)R_{21}(2) [Y_1^{-1}(1)Y_1^{+1}(2)-Y_1^{+1}(1)Y_1^{-1}(2)] \\
  \Psi_{-1,0} &=\frac{1}{\sqrt{2}} R_{21}(1)R_{21}(2) [Y_1^{-1}(1)Y_1^0(2)-Y_1^{0}(1)Y_1^{-1}(2)] \\
  \Psi_{-1,-1} &= 0 \ \ (\varphi=\varphi') 
\end{align}

No obstante, notamos que $\Psi_{-1,+1}= -\Psi_{+1,-1}$ y que $\Psi_{0,+1}= -\Psi_{+1,0}$,$\Psi_{0,-1}= -\Psi_{-1,0}$, de forma que sólo hay dos funciones posibles: $\Psi_{0,0}$ y $\Psi_{-1,+1}$. La degeneración será doble.


\begin{tcolorbox}[halign=left]
  \emph{Suponga que la tercera componente del momento angular orbital
    de una partícula es $+\hbar$ y la de la otra es $-\hbar$. Escriba
    una función de ondas que satisfaga estas condiciones.}
\end{tcolorbox}

Necesariamente será la $\Psi_{+1,-1}$.

\begin{tcolorbox}[halign=left]
  \emph{¿Puede escribir dos funciones de onda que satisfagan las
    condiciones anteriores y que sean ortogonales entre sí? ¿Por qué?}
\end{tcolorbox}

No, ya que únicamente nos queda $\Psi_{0,0}$ y no cumple la condición de tener momentos angulares $\pm \hbar$.

\begin{tcolorbox}[halign=left]
  \emph{Suponga que se mide el momento angular total del sistema
    descrito por la función de ondas del segundo apartado. ¿Qué
    valores pueden obtenerse y con qué probabilidades?}
\end{tcolorbox}

Acoplamos ambos $\ell$ con la tabla de Clebsch-Gordan $1\otimes 1$:
\begin{equation}
  \begin{split}
    \Psi_{+1,-1} &= \frac{1}{\sqrt{2}}R_{21}(1)R_{21}(2) [Y_1^{+1}(1)Y_1^{-1}(2)-Y_1^{-1}(1)Y_1^{+1}(2)] \\
     &= \frac{1}{\sqrt{2}}R_{21}(1)R_{21}(2) [\frac{2}{\sqrt{2}} \ket{2 0}] \\
     &= R_{21}(1)R_{21}(2) \ket{2 0} \\
  \end{split}
\end{equation}

Obtenemos una función con momento angular orbital total $\ket{2 0}$ y sin espín, luego $J\in 2\otimes 0 = \{2\}$. Hay una probabilidad del 100\% de medir $J=2$.

\chapter{Último examen}
Se deja su resolución como ejercicio para el lector.

\begin{tcolorbox}[halign=left]
  \emph{Una partícula sin espín y de masa $m$ está en un potencial
    tipo oscilador anisótropo $V=\oh (kx^2+ky^2+k'z^2)$ donde
    $k'=k+\Delta k$ y $\Delta k \ll k$.}

  \emph{Si escribe $V=\oh (kx^2+ky^2 +kz^2) + \oh(\Delta k z^2)$ puede
  considerar que el ltimo sumando es una perturbación al oscilador
  armónico isótropo. Calcule en primer orden de perturbaciones la
  corrección a la energía del nivel fundamental y del primer nivel excitado. }

\emph{Escriba las energías de la solución exacta del oscilador
  anisótropo y calcule su aproximación a primer orden en $\Delta k$.
  ¿Son iguales a los resultados del apartado anterior? (spoiler: sí)}
\end{tcolorbox}

\begin{tcolorbox}[halign=left]
  \emph{Dos bosones idénticos independientes de espín 1 están en un
    potencial central. ¿Cuáles son los términos espectroscópicos de
    las configuraciones (1s)\textsuperscript{2}, (1s)(2s) y
    (2p)\textsuperscript{2}? ¿Cuál es la degeneración de cada una de
    estas configuraciones?}

  \emph{Escriba una función de ondas de la configuración (1s)(2s) con
    el máximo momento angular orbital total y el mínimo espín total.}

  \emph{Suponga que los 6 electrones del átomo de carbono tuvieran
    espín 1 (Ojo, entonces son bosones y se apelotonan). ¿Cuáles serían los
    términos espectroscópicos de la configuración fundamental y de la
    primera excitada?}
\end{tcolorbox}


\begin{tcolorbox}[halign=left]
  \emph{
  Describa el efecto Zeeman. Si se observa en el átomo de sodio, ¿cuál
  de las transiciones del nivel ${}^{2}\! P_{\oh}$ de la primera
  configuración excitada al nivel fundamental ${}^2\! S_{\oh}$ es la
  más probable?}
\end{tcolorbox}

%%% Local Variables:
%%% mode: latex
%%% TeX-master: "../resumen"
%%% End:
