\section{2016-05-11}
Vimos que $T_q^k$ es tensorial de rango $k$ bajo $\boldrm{J}$, con $k\in
\mathbb{Z}^+$.
$[J_z,T_q^k] = \hbar q T_q^k$
$[J_{\pm},T_q^k] =  \hbar \sqrt{k(k+1)-q(q\pm1)} T_{q\pm1}^k$

De la primera (
$[J_z,T_q^k] = \hbar q T_q^k$) deducimos que
$\braket{\alpha'j'm'}{T_q^k}{\alpha jm}=0$ si $m'\neq q+m$. De la segunda
definición vemos que ( o vamos a ver o algo)
\begin{equation}
  \begin{split}
    \mel{\alpha'j'm'}{[J_\pm,T_q^k]}{\alpha j m}  \to  \mel{\alpha'JM}{[J_\pm,T_{m_2}^{J_2}]}{\alpha j_1 m_1}
  \end{split}
\end{equation}
Hemos hecho un cambio de notación. Notar que por def (la segunda, como
hemos dicho) el el. de matriz se puede expandir. Utilizando la nueva
notación:
OJO: utilizamos $J_+$ (luego $J_-$)
\begin{equation}
  \begin{split}
    \mel{\alpha'JM}{[J_\pm,T_{m_2}^{J_2}]}{\alpha j_1 m_1} = c_+ (j_2 m_2)
    \mel{\alpha J M}{T_{m_2+1}^J_2}{\alpha j_1 m_1}
  \end{split}
\end{equation}
Para el $J_-$ sale $c_- (JM)
    \mel{\alpha J M-1}{T_{m_2}^{J_2}}{\alpha j_1 m_1}
$.

Luego ha puesto
$c_- (JM)
    \mel{\alpha J M-1}{T_{m_2}^{J_2}}{\alpha j_1 m_1} - c_+ (j_1 m_1)
    \mel{\alpha J M}{T_{m_2}^{J_2}}{\alpha j_1 m_1+1}
$

Despejando, me queda esta formulica:
\begin{equation}
  \begin{split}
    c_-(JM) \mel{\alpha' J M-1}{T_{m_2}^{J_2}}{\alpha j_1 m_1} &= c_- ( j_1 m_1 +
    1) \mel{\alpha' J M}{T_{m_2}^{J_2}}{\alpha j_1 m_1 + 1}+ \\
    &+ c_-(j_2 m_2+1)\mel{\alpha' J M}{T_{m_2+1}^{J_2}}{\alpha j_1 m_1}
  \end{split}
\end{equation}

Dice que todo esto era con $J_+$. Si calculo con $J_$ hay varios cambios
de signo

\begin{equation}
  \begin{split}
    c_+(JM) \mel{\alpha' J M+1}{T_{m_2}^{J_2}}{\alpha j_1 m_1} &= c_+ ( j_1 m_1 -
    1) \mel{\alpha' J M}{T_{m_2}^{J_2}}{\alpha j_1 m_1 - 1}+ \\
    &+ c_+(j_2 m_2-1)\mel{\alpha' J M}{T_{m_2-1}^{J_2}}{\alpha j_1 m_1}
  \end{split}
\end{equation}

obtenemos
\begin{equation}
  \boxed{ c_-(j_1 m_1) = c_+ (j_1 m_1 -1)}
\end{equation}
Esta notación ayuda a entender las tablas de Clebsch-Gordan.
Es decir,
$(j_1 j_2 m_1 m_2 | J M)$ y $\mel{\alpha J M}{ T_{m_2}^{J_2}}{\alpha j_1 m_1}$
satisfacen el mismo sistema lineal homogéneo. Ejeplo:
\begin{equation}
  \begin{cases}
    x+y = 0 \\
    2x+2y = 0 \\
    3z+3t = 0 \\
    4z+4t = 0 \\
    5w = 0 \\
  \end{cases}
\end{equation}
Los els de matriz satisfacen
\begin{equation}
  \begin{cases}
    A+B = 0 \\
    2A+2B = 0 \\
    3C+3D = 0 \\
    4D+4D = 0 \\
    5E = 0 \\
  \end{cases}
\end{equation}
Ésto significa que los els de matriz dependen unos de otros, tienen
los mismos grados de libertad.

Para empezar, significa que Clebsch-Gordan que se anula, el de matriz
que se anula, por ejemplo, el $5w=0$. Pero, ¿cuándo un Clebsch-Gordan
es nulo? $\mel{\alpha J M}{ T_{m_2}^{J_2}}{\alpha j_1 m_1}$ es nulo si
$M\neq m_2 + m_1$ o si $J_1 \otimes J_2$ no contiene a $J$. La segunda
condición, la de $J_1 \otimes J_2$ significa que $J$ tiene que estar entre
$j_1-j_2 , \cdots , j_1 + j_2$.

Ejemplos
\begin{equation}
  \frac{A}{x} = \alpha \to B  = \alpha y
\end{equation}
\begin{equation}
  \frac{c}{z} = \beta \to D  = \beta z
\end{equation}

Vemos que los grados de libertad no dependen de terceras componentes.
Basta fijar los cocientes entre dos grados de libertad y puedo conocer
todos los elementos de matriz gracias a los Clebsch-Gordan.

Esto significa que yo puedo escribir
\begin{equation}
  \mel{\alpha J M}{T_{m_2}^{J_2}}{\alpha j_1 m_1} = \text{cte.} (\alpha J j_2 \alpha)(j_1
  j_2 m_1 m_2 | JM)
\end{equation}
donde $\text{cte.}(\alpha J j_2 \alpha)(j_1
  j_2 m_1 m_2 | JM)$ sería o la $\alpha$ o la $\beta$. Depende de muchas cosas,
pero no de $M,m_1,m_2$. La dependencia en terceras componentes está
en $(j_1
j_2 m_1 m_2 | JM)$. TEOREMA DE WIGNER-ECKART.\footnote{Requiere que
  lo del medio sea tensor del bicho y lo de los lados del <|> sean
  vecs propios o algo así.}

El teorema dice cosas como;
sean $J=1,j_2=1,j_1=1$. Tendría $3^3=27$ integrales (+1,0,-1,
+1,0,-1,   +1,0,-1). Hay seis integrales distintas de cero (ver tabla,
JM=1,1, 1,0,  1,-1). Cuántas hay que calcular de esas 6? Sólo una,
calculada una, calculadas todas.

Ejemplo de aplicación:
\begin{equation}
  \mel{\alpha' J:1 +1}{T_0^{J_2:1}}{\alpha J_{\text{nosé}}:1 +1} \neq 0
\end{equation}
Esto da
\begin{equation}
  (1 1 +1 0 | 1 +1) \cdot c (\alpha', 1 , T^1, \alpha, 1)
\end{equation}
El primer paréntesis está en la tabla. Por tanto
$c(\alpha, 1, T^1,\alpha,1) = \frac{\mel{\alpha' 1 +1 }{T_0^1}{\alpha 1 +1}}{(1 1 +1 0 | 1
  +1)}$

y $\mel{\alpha' 1 M }{T_{m_2}^1}{\alpha 1 m_1} = c (\alpha 1 T' \alpha 1) \cdot
\underbrace{c(\alpha' 1 T' \alpha 1)}_{\text{Un C-G (tabla)}}$.

Por tanto, de las 27 sólo hay 6. Hago una, y con las tablas ya tengo
las otras 5.

Ojo, $(1 1 +1 0 | 1 +1) \stackrel{??}{\neq}0$. Utilizar la integral
que no sea nula! si no la lías

\subsection{Reglas de selección}\footnote{Ha puesto un $\varvarepsilon1$ o algo
  así tras ``Reglas de selección''}
Las vamos a ver en el hidrógeno. Ya las veremos en átomos de muchos electrones.

Tenemos bichos como $\mel{(nlsjm)_f}{\boldrm{r}}{(nlsjm)_i}$. Los
paréntesis indican los números cuánticos iniciales y finales.
\begin{equation}
\abs{ \mel{(nlsjm)_f}{\boldrm{r}}{(nlsjm)_i} }^2 =
\abs{\mel{f}{x}{i}}^2 + \abs{\mel{f}{y}{i}}^2 + \abs{\mel{f}{z}{i}}^2
= \cdots
\end{equation}
Hay tres integrales. El operador $\boldrm{r}$ se puede poner en
coordenadas cartesianas o en coordenadas tensoriales\footnote{El
  operador $r_q^1$ es un tensor de rango 1 bajo
  $\boldrm{L}=\boldrm{r}\times \boldrm{p}$. Habría que comprobar que
  $[L_z,r_q^1]=q \hbarr_q^1$ y que $[L_\pm,r_q^1]= \hbar
  \sqrt{1(1+1)-q(q\pm1)} r_{q\pm1}^1$. Si te aburres lo haces.}. Tiene tres
coordenadas tensoriales, la $r_0^1=z = \sqrt{\frac{4\pi}{3}} r Y_0^1$ y las $r_{\pm 1}^1= -
\frac{1}{\sqrt{2}}(x\pm iy) \propto r Y_{\pm 1}^1 $. Coinciden con los armónicos esféricos
casi.\footnote{Se llaman armónicos esféricos porque son la serie de
  Fourier de algo en función de $\theta$ y $\varphi$ en una esfera. $F(\theta,\varphi)==sum c Y_i$}
Bueno, pues tenemos tres integrales. Se puede demostrar\footnote{simple como
el mecanismo de un sonajero,  no hay más que despejar y ya de las
formulicas de $r_{0,1,-1}^1$.} que el tocho se puede escribir como
\begin{equation}
\cdots =  \abs{\mel{f}{r_{+1}^1}{i}}^2 + \abs{\mel{f}{r_{0}^1}{i}}^2 + \abs{\mel{f}{r_{-1}^1}{i}}^2
\end{equation}
Bueno, pues tenemos
\begin{equation}
  \abs{\mel{\underbrace{\alpha_fJ_f m_f}_{fijo}}{r_1^1}{\underbrace{\alpha_i J_i m_i}_{fijo}}}^2 +
  \abs{\mel{f}{r_0^1}{i}}^2 +  \abs{\mel{f}{r_{-1}^1}{i}}^2
\end{equation}
Sólo hay que hacer una integral (W-E theorem) pero será no nula si
$m_f=m_i+1$. Pero si la primera no es nula, es imposible que las demás no
sean nulas. Cómo mucho, hay una no nula pues (o todas).

Bueno, supongamos que una no es nula, la primera por ejemplo.
$\mel{\alpha_F j_f m_F}{r_1^1}{\alpha_i j_i m_i} \neq 0 , (m_f=m_1+1)$, al menos
por ahora. WE nos dice que $j_f \subset j_1 \otimes 1 = j_i+1, j_i,
j_i -1$. El 1 es el espín del fotón, que es uno.

En fín, vamos a lo que vamos. Escribo $\mel{f}{\boldrm{r}}{i}$ con
significado de ``la que no se anule de todas''. me da
\begin{equation}
\mel{f}{\boldrm{r}}{i} = \sum_{(m_l \mu)_f, (m_l \mu)_i} (l m_l s \mu
| j m)_f^* (lm_l s\mu | jm)_i \cdot
\mel{(nlm_l)_f}{\boldrm{r}}{(nlm_l)_i} \braket{ \oh \mu_f}{\oh \mu_i}
\end{equation}
Me importa cuánto vale $\mel{(nlm_l)_f}{\boldrm{r}}{(nlm_l)_i} $.
Notar que $\boldrm{r}$ es tensor de rango uno bajo $\boldrm{L}$ y los
paréntesis son vects propios de no se qué, y por tanto $l_f \subset
l_i \otimes 1 = l_i+1,l_i,l_i-1$. \emph{Un dipolo eléctrico o mantiene el momento
angular o lo cambia en una unidad}. Si esto no ocurre, la transición
dipolar eléctrica no existe.\footnote{
\emph{Clarificación a duda:} $\mel{\alpha' J M}{T_{m_2}^{J_2}}{\alpha j_1 m_1} =
c (\alpha' J T J_2 \alpha J_1)(j_1 j_2 m_1 m_2|JM=0)$ Por $J \notsubset j_1 \otimes j_2$
}

Si nos pasa esto, una transición prohibida, es que el dipolo magnético
(que despreciamos) es relevante, u otras cosas que simplificamos.

Nos queda otra regla de selección que no depende del Wigner-Eckart.
Tiene que ver con la paridad.

\begin{equation}
  \begin{split}
    \mel{(nlm_l)_f}{\boldrm{r}}{(nlm_l)_i} &= \int_V \dd{v}
    R^*_{(nl)_f} (r) \underbrace{Y_{l_f}^{m_l_f*}
      (\Omega)}_{(-1)^{l_f}} \underbrace{\{r_1^1,r_0^1,r_1^1\}}_{(-1)}
    R_{(nl)_i}(r) \underbrace{Y_{l_i}^{m_l_i}(\Omega)}_{(-1)^{l_i}} =
    \\
    &= -  \underbrace{(-1)^{l_i} (-1)^{l_f}}_{-1}
    \underbrace{\mel{(nlm)_f}{\boldrm{r}}{(nlm)_i}}_{\neq 0}
  \end{split}
\end{equation}
Los parens dicen como cambian bajo paridad o algo así. Ojo $l_f
\subset l_i-1,l_i,l_i+1$ implica que no puedo coger $l_f=l_i$ porque si no
ya no me sale el $-1$ en la ecuación.

Por lo tanto, en el dipolo eléctrico, $\text{paridad inicial} \neq \text{paridad
inicial} $. Se dice que \emph{transporta paridad}. Notar que esto solo sirve si
la paridad inicial y final están definidas (pares o impares).

Outline:
\begin{enumerate}
\item Miramos els de matriz
\item Clebsch-Gordan
\item Definimos operadores tensoriales si existen (como $\boldrm{r}$).
  Todos los multipolos electromagnéticos lo son.
\item Si se cumplen cosas hacen como Clebsch-Gordan.
\item Calculo de las 6 ints sólo 1.
\item Los 6 que no se anulan son los que cumplen la sreglas de selección.
\end{enumerate}

Si no se cumplieran estas tres reglas de selección:
\begin{enumerate}
\item El dipolo eléctrico transporta momento angular.
\item ???? ahí se ha quedado
\end{enumerate}


En el hidrógeno
\begin{enumerate}
\item Jf contenido en ji otimes 1 ($|\Delta j|=0,1$)
\item lf contenido en li \otimes 1 ($ |\Delta l|=1$)
\item li \neq lf ($0 !\to 0$)
\end{enumerate}
entre parens versión antigua

Recordar, dijimos que $e^{i \boldrm{k} \boldrm{r}} \simeq 1$ y nos salía el
dipolo ($\varvarepsilon_1$, ver fig adjunta).

\section{2016-05-12}

Comenzamos la sección \emph{átomos multielectrónicos} ($Z>2$). Utilizaremos la
aproximación de campo central. Esperamos que los espectros atómicos
sean progresivamente más complicados con $Z$ pero vemos que no es así,
hay periodicidad en las columnas de la tabla periódica. Tratamos de
modelar esto con un \emph{modelo de capas}. El procedimiento es
similar al del átomo de helio.

El hamiltoniano será
\begin{equation}
  \Ham = \sum_{i=1}^Z - \frac{\hbar^2}{2m} \nabla_i^2 - \sum_{i=1}^Z
  \frac{Ze^2}{r_i} + \sum_{i<j} \frac{e^2}{r_{ij}} +
  \cancelto{0}{W_\text{fine}}, \ \ M_N \gg m_e
\end{equation}
Nos olvidamos del término de estructura fina por ahora. Nos molesta
$r_{ij}$ como en el helio, y lo arreglamos igual, con potenciales
centrales virtuales.
\begin{equation}
  \begin{split}
    \Ham &= \sum_{i=1}^Z - \frac{\hbar^2}{2m} \nabla_i^2 + \sum_{i=1}^Z V_z(r_i) - 
    \\
    &-
      \sum_{i=1}^Z V_z(r_i) - \sum_{i=1}^Z
    \frac{Ze^2}{r_i} + \sum_{i<j} \frac{e^2}{r_{ij}} +
    \cancelto{0}{W_\text{fine}}, \ \ M_N \gg m_e 
  \end{split}
\end{equation}

Obtenemos $\Ham = \Ham_0 + W$, con $\Ham_0 \gg W$. Veremos que los
resultados son útiles cuantitativamente.\footnote{Actualmente se
  utilizan más métodos variacionales que perturbativos.}
La parte dificil es ver el potencial $V_z(r_i)$.

Nos preguntamos cómo son los autoestados y autovalores de $\Ham_0$.
Son fáciles de hallar porque $\Ham_0 = \sum_{i}h_i$ (son $Z$
electrones independientes). Basta con estudiar el hamiltoniano de una
sóla partícula: $h = \frac{-\hbar^2}{2m} \nabla^2 + V_z(r)$ con
autoestados $\varphi_{n\ell m
  \mu}(\boldrm{r})=R_{n\ell}(r)Y_l^m(\Omega)\ket{\mu}$\footnote{$\ket{\mu}$ es
  el espín.}. Ojo, el armónico esférico no influye para potenciales
centrales, sólo la parte radial. Hallaremos unos autovalores $E_{n\ell}$
dependientes de $V_z(r)$.Vemos que este potencial es muy cómodo\footnote{Ante la ignorancia, la esfera.}.
La degeneración de $E_{n\ell}$ será $2(2\ell+1)$\footnote{Por el $\ket{\mu}$
  y el $Y_l^m(\Omega)$.}. Ojo, el potencial no tiene
por qué ser de Coulomb. Las energías no tienen por qué ser de ese
estilo\footnote{En el potencial de Coulomb había ciertas
  coincidencias (ver dibujo del tema de potenciales centrales.)}.

\paragraph{Convenio}
Se utiliza $n_{\text{min}}=\ell+1$. Así se tiene algo parecido al átomo
de hidrógeno\footnote{Creo que ya está apuntado por ahí. Conv. de
  física atómica? Vale sí. Lo es. $E_{10},E_{20}\ldots$ y
  $E_{21},E_{31},\ldots$}. Viene bien para tener en un futuro reglas memotécnicas del
llenado de capas.

\paragraph{Autovalores de $\Ham_0$}
Recordamos que $\Ham_0 = \sum_{i=1}^Zh_i$. ¿Cuál es la energía del
sistema? Simplemente,
\begin{equation}
  E = \sum_{i} k(n_i\ell_i) E_{n_i\ell_i}
\end{equation}
donde $k$ es el número de electrones con energía $(n_i\ell_i)$. 
\paragraph{Autoestados de $\Ham_0$}
Los autoestados de $\Ham_0$ son determinantes de Slater
\footnote{Porque los electrones son fermiones independientes, en el
  caso considerado, el modelo de capas}.
Recordamos que la degeneración está definida, y por tanto
$k(n_i\ell_i)<2(2\ell_i+1)$. Si cojo más, el determinante de Slater se anulará.
Llamamos \emph{capas} a los subespacios de degenaración de $E_{n\ell}$.


Por tanto, $k(n\ell)$ es el número electrones en la capa $(n\ell)$, también
llamado \emph{número de ocupación}.
$2(2\ell+1)$ es el número máximo de electrones en la capa $(n\ell)$.


\paragraph{Nivel fundamental de $\Ham_0$}
% Leer apuntes de pablo, pag 77
% [...]
La \emph{regla de Madelung}\footnote{Ojo, no sirve para hallar el
  llenado de capas, sino para hallar la \emph{última} capa rellena.
  Si se utiliza para ver el llenado de capas se encuentran varias
  excepciones, como el cobre o el cromo.} nos dice que tenemos que coger el $n$
mínimos, y dentro del mínimo $n$ el $n+\ell$ mínimo\footnote{Esta regla
  supone el convenio ya visto de $n_{\text{min}}=\ell+1$.}.

\section{2016/5/16}
Regla de Madelung nos da las configuraciones fundamentales. También
necesitamos las excitadas.
\begin{itemize}
\item Los alcalinos tienen siempre capas completas con degeneración 2.
  Como las capas completas no contribuyen al no. det. de Slater, sólo
  tenemos el grado de libertad del espín. Última conf. $(ns)^1$
\item Su primer excitado tienen degeneración 6. Acaba en $(np)^1$
\item Los alcalino-térreos tienen como última capa ``incompleta''
 \footnote{también es completa} $(ns)^2$. La degeneración es 1 (sólo
  hay un det. de Slater).
\item Su primer excitado acaba en $(ns)^1(np)^1$. La degeneración es $12$.
\item Un ejemplo no trivial es el cabono en el fundamental, acabado en
  $2p^2$. La degeneración es $\binom{6}{2}=15$.
\item El primer excitado del carbono acaba en $(2p)^1(3s)^1$. La deg. es 12.
\end{itemize}

Nos vamos al siguiente orden de aproximación, los términos
espectroscópicos.

\subsection{Términos espectroscópicos}
Tenemos (o teníamos, no sé) una aproximación $\Ham = \Ham_0 +
\cancel{W}$. Es nuestro \emph{modelo de capas}.
La perturbación es
\begin{equation}
  W(\boldrm{r}_1,\ldots,\boldrm{r}_Z) = - \sum_{i=1}^Z V_z(r_i)-
  \sum_{i=1}^Z \frac{Ze^2}{r_i} + \sum_{i<k} \frac{e^2}{r_{ij}}
\end{equation}
Consideramos los subespacios de degeneración de $\sum_{i}
k(n_i\ell_i)E_i$\footnote{Notar que la base de estos subespacios son
  determinantes de Slater de tamaño $Z\timesZ$, tantos como la
  degeneración. Ya hemos visto que la degeneración puede ser enorme,
  así que la cosa se complica rápido.}.
% Mejor no encontrárselo por la noche en un sitio sin iluminación.
% [referente a las funciones de onda]
Tenemos que $[W,\boldrm{L}]=0$, al igual que $[W,\boldrm{S}]$.
\footnote{Donde $\boldrm{L} = \sum_{i} \boldrm{L}_i$ e igual para $\boldrm{S}$.}

La base astuta para diagonalizar será $L^2,L_z,S^2,S_z$, pero para
dos electrones no es un CSCO. Ejemplo:
\begin{equation}
  1 \otimes 1 \otimes 1 = \underbrace{(0 \oplus 1 \oplus 2)}_{1\otimes
  1} \otimes 1 = \{1,0,1,2,1,2,3\}
\end{equation}
Notar como se repiten algunos momentos. Hace falta saber la
configuración (la genealogía) para romper esa degeneración. Notar como
si aumentamos el número de electrones, aumenta la degeneración
(momentos repetidos) y necesitamos una genealogía mayor.

Luego nuestro CSCO será $L^2,L_z,S^2,S_z$ y la genealogía.

No obstante, vemos que los átomos no son tan complicados a veces, y
hay regularidades. Esto es debido a que las capas completas no
contribuyen al momento angular.

La primera capa completa es $(ns)^2$, con $L=0$ y $S=0,\cancel{1}$ (descartamos
$L:0,S:1$ por ser una combinación simétrica).

La siguiente capa completa es $(np)^6$, con
$L=1\otimes1\otimes\cdots\otimes1 = \{0,\ldots,6\}$.Notar que algo nos
dice que tiene que ser $L,S=0$ porque la degeneración del det de
slater es 1\footnote{Pocas terceras componentes puedo tener...}.

\subsection{Por qué las capas completas tienen $L=0$}

\begin{equation}
(np)^6 = \frac{1}{\sqrt{6!}} \sum_{p} i_p P\{(++)_1(+-)_2(0+)_3(0-)_4(-1+)_5(-1-)_6\}
\end{equation}\footnote{p de permutaciones}
Hay 720 sumandos\footnote{En la pizarra, $+-$ eran \verb~uparrows~} ortogonales, donde
\begin{equation}
  (1+)_1 = R_{np}(r_1) Y_1^{+1}(1) \ket{+}_1
\end{equation} 
El det de Slater es tocho, con diagonal $\varphi_{np 1+1+}(1)$, $\varphi_{np
1+1-}(2)$, ...

\subsubsection{caso con $Z=6$}
El $L_z$ (recordar que hay muchos $\otimes \mathbb{I}$) será
$L_z=L_{1z}+\cdots+L_{6z}$ y
\begin{equation}
  L_z(np)^6 = \frac{1}{\sqrt{6!}} \sum_{P} i_p L_z P \{\cdots\} =
  \frac{1}{\sqrt{6!}} \sum_{P} i_p P L_z \{\cdots\}
\end{equation}
qué hace $L_z$ sobre eso? pues
\begin{equation}
  (L_{1z}+\cdots+L_{6z}) \{\cdots\} = + \hbar (1+)_1 (1-)_2 \cdots
  (-1-)_6 + (1+)_1 \hbar(1-)_2 \ldots
\end{equation}
Me quedarán ceros. Es la versión en 6 partículas de que las terceras
componentes se suman\footnote{Salen cosas como $(+\hbar \otimes
  \mathbb{I} \otimes \mathbb{I} \otimes \cdots)  +(- \hbar\otimes \mathbb{I} \otimes
  \mathbb{I} \cdots) = 0 $ (\joke a tomar viento)}.

Con $L_x$ y $L_y$ es más complicado porque los vectores considerados
no son propios de estos operadores, hagámoslo con $L_+$ y $L_-$ por
los loles:
\begin{equation}
  L_+ = L_{1+} + \cdots + L_{6+}
\end{equation}
Recordar que faltan muchas identidades.
\begin{equation}
  L_+(np)^6 = \frac{1}{\sqrt{6!}} \sum_{P} i_p \underbrace{L_+
    P\{\}}_{P L_+\{\}} 
  \label{eq:htnht}
\end{equation}

\begin{equation}
  \begin{split}
    (L_{1+}+\cdots L_{6+})\{\cdots\} &=
    \underbrace{0}_{L_{1+}\{\cdots\}} +
    \underbrace{0}_{{L_{2+}\{\cdots\}}} + \\ &+ (1+)_1(1-)_2\sqrt{2} \hbar
    (1+)_3 (0-)_4(-1+)_5(-1-)_6 + \\ &+ (1+)_1(1-)_2 (0+)_3 \sqrt{2}
    \hbar(1-)_4(-1+)_5(-1-)_6 +
  \end{split}
\end{equation}
Los ceros son porque no se puede subir más. Notar como hemos aumentado
el numerico que va con el $\sqrt{2}\hbar$. Es abstracto porque nos
estamos saltando un determinante $6\times6$, 720 sumandos.

Ahora toca la parte de sumar de la eq. \ref{eq:htnht}. Pero ver que en
la última eq. hay muchos términos rep, y por tanto esos determinantes
de slater se anulan\footnote{Al foso!}. Luego nos queda cero.
\begin{equation}
  L_+ (np)^6 = 0
\end{equation}
Con $L_-$ pasa más de lo mismo, pero bajando en vez de subiendo.
\begin{equation}
  L_- (np)^6 = 0
\end{equation}
Por tanto, $L_x y L_y$ serán nulos sobre $(np)^6$. Como $L_z$ también
daba cero, tenemos que $L^2(np)^6=0$. Con $S^2(np)^6$ pasa igual.
Notar que esto pasa porque los electrones son fermiones idénticos
\footnote{El determinante de Slater es como la parca con la guadaña.
  No habéis oído nunca segar alfalfa con la guadaña? fzas fzash fzash
  zashhsh\emph{shshsh}}. Tenemos que

\begin{center}
  $L^2$, $S^2$, $J^2$ dependen sólo de las capas
  \bfseries{incompletas}
\end{center}


\section{2016-05-19}
Teníamos un hamiltoniano
\begin{equation}
  \Ham = \Ham_0 + W + (W_\text{EF})
\end{equation}
Las correcciones del término $W_\text{EF}$ son de la forma
\begin{equation}
  \sum_{i} k(n_i \ell_i) E_{n_i \ell_i} + \Delta(\text{configuración},L,S)
\end{equation}
con ``configuración'' se quiere decir configuración y posible
genealogía. Vemos que no hay influencia de las terceras componentes.
Para $\Delta$ sólo influyen las capas incompletas\footnote{ya que las
completas siempre tienen $L,S=0$}.

para $(capa completa)(n\ell)^1$ y $(capa completa)(n\ell)^2$ (y
complementarias) no se qué.

Ha hablado de configs complementarias y regla de Hund. Buscar en
apuntes extra.

\subsection{Estructura fina}
En $W_\text{EF}$ aparecerán como poco términos espín órbita como poco.
Sólo con el espín órbita ($\mathbf{l}_i\cdot \mathbf{s}_i$) ya vemos
que $[W_\text{EF},\mathbf{S}]\neq 0$ al igual que
$[W_\text{EF},\mathbf{L}]$. En cambio, sí tendremos
$[W_\text{EF},\mathbf{J}]=0$ porque una rotación completa conserva los
ángulos. En el átomo de helio el razonamiento era exactamente igual.

Utilizamos la notación compacta
\begin{equation}
  \ket{ (\text{conf}) ; L S J M}
\end{equation}
Lo primero que hay que recordar\footnote{Se halla igual que en el helio} es que
\begin{equation}
  \mel{(\text{conf})LSJM}{W}{(\text{conf})L'S'J'M'} \sim
  \delta_{LL'}\delta_{SS'}\delta_{JJ'}\delta_{MM'} \Delta(\text{conf};L,S)
\end{equation}
Pasemos a calcular los elementos de matriz. OJO: Ahora ponemos
$W_\text{EF}$ y no $W$:
\begin{equation}
  \begin{split}
    \mel{(\text{conf})LSJM}{W_\text{EF}}{(\text{conf})L'S'J'M'} & \sim
    \delta_{JJ'} \delta_{MM'} \\
    & \nsim \delta_{LL'}\delta_{SS'}
  \end{split}
\end{equation}
Ahora pasamos al acoplamiento L-S. Cuando $W_\text{EF}\ll
W$ los elementos de fuera de la diagonal\footnote{Dice que se llama en
realidad superdiagonal} son irrelevantes, como
ya se vió\footnote{Lo de la matriz $2\times2$ y eso}.

Recordar que $W$ es la repulsión coulombiana y $W_\text{EF}$ son los
términos de estructura fina. 

Tenemos
\begin{equation}
  \ev{W_\text{ef}}{(\text{conf})LSJ\cancel{M}} = \Delta_{EF}(\text{conf};LSJ)
\end{equation}
Notar que no depende de terceras componentes, aún persiste algo de
degeneración. Todo idéntico a lo ya visto.

Veamos algún ejemplo de lo que ocurre

\paragraph{Ejemplo: Alcalinos fundamental}
Son capas completas con $(ns)^1$ al final, con únicamente el término
espectroscópico ${}^2S$. Se tiene $J=\oh$ y por tanto se tiene
${}^2S_\oh$. La primera excitada es $(c.c.)(np)^1$, término
espectroscópico ${}^2P$ y $J\in\{\oh,\nicefrac{3}{2}\}$ (doblete del
sodio). Tenemos por tanto dos términos, ${}^2P_\oh$ y
${}^2P_{\nicefrac{3}{2}}$. El ordenamiento es: a menor $J$ menor
energía. Por lo tanto, (figura 1).


\paragraph{Ejemplo: Alcalinotérreos fundamental}
Los alcalinotérreos son más simples, al menos el fundamental. Son
capas completas y $(ns)^2$, con término ${}^1S$ y te tiene $J=0$ por
lo que ${}^1S_0$. La primera excitada es la $(ns)(np)$, donde tenemos
el ${}^3P$ y el ${}^1P$. En la ${}^1P$ se tiene $J=0\otimes1=1$ y en
la ${}^3P$ obtenemos $J=1\otimes1=\{0,1,2\}$ (figura 2).

Las configuraciones complementarias son exactamente iguales, excepto
que cuando la $J$ disminuye la energía aumenta. A esto se le llama
\emph{multipletes invertidos}. El resto es exactamente igual, acoplar
con Clebsch-Gordan.

Regla curiosa

% Use regla, not teorema
\begin{thm}[Regla del intervalo de Landé]
Bajo dos hipótesis
\begin{itemize}
\item $W_\text{EF}\ll W$ (acoplamiento L-S)
\item Término de espín-órbita ($\sum f(r_i)\mathbf{l}_i \mathbf{s}_i$)
  dominante en $W_\text{EF}$
\end{itemize}
Se tiene en la figura 3 que
\begin{equation}
  \frac{E(LS,J+2)-E(LS,J+1)}{E(LS,J+1)-E(LS,J)} = \frac{J+2}{J+1}
\end{equation}
\end{thm}
Por eso antes se ponía el $a$ en ${}^3 P_a$. Ponían $a$ por la distancia
entre niveles, pero medían el momento angular total sin saberlo.
La regla funciona bien hasta el germanio aprox, luego se rompe la
primera hipótesis\footnote{Se pasa de acoplamiento L-S a acoplamiento
j-j. Eso ya para el año que viene.}.

\subsection{Reglas de selección $\epsilon_1$}

\begin{equation}
  \mathbf{r} \ \rightarrow \ \sum_{i=1}^Z \mathbf{r}_i
\end{equation}

\begin{equation}
  \matrixel{ (\alpha LSJM)_f}{\sum_{i}\mathbf{r}_i}{(\alpha LSJM)_i}
  \label{eq:foobar}
\end{equation}
con $f$ significando final y $i$ inicial. 

La primera regla de selección no tiene nada que ver con el teorema de
Wigner-Eckart. Es que $\sum_{i}\mathbf{r}_i$ es impar, por lo que la
paridad a ambos lados del el. de matriz de \ref{eq:foobar} debe ser
contraria\footnote{Si la integral no sale positiva se tiene que la
  integral es la menos integral, y queremos que la inetgla sea la mas
  integral. Si sale $I=-I$ es nula!}.

Notar que la paridad de $(1s)^2,(2s)^2,\cdots,(n\ell)^k$ son sólo los armónicos esféricos, así que va con
$(-1)^{\ell_1},\cdots,(-1)^{\ell_z}$, con las capas completas teniendo
paridad positiva\footnote{
  Se tiene $(nl)^{2(2\ell+1)}$, luego (+1).
}.

% NOTA: te saltaste una clase. Creo que una sola, sí. Javi dice 2.
\section{2016-05-25}
Último problema de la hoja 4 (el del carbono).
A lo loco, ya casi de memoria.
Se tiene $L=0_S,1_A,2_S$ y $S=0_A,1_S$. Sólo son posibles, por tanto,
las combinaciones simétricas.

El ordenamiento, por el principio de Hund, sería $1S,1D,3P$.
Las $J$ correspondientes son
$0\otimes0=0,0\otimes2=2,1\otimes1=0,1,2$. El ordenamiento del $3P$
será $3P2,3P1,3P0$ (la capa incompleta está menos que semillena).

El primer excitado es $(\text{c.c.})(2p)^1(3s)^1$. El $L=1$ y $S=0,1$.
El $L$ me da igual porque puedo simetrizar y antisimetrizar a gusto,
tenemos $1P$ y $3P$ por lo tanto.

Tendremos $J=0\otimes1=1$ y $J=1\otimes1=0,1,2$, así que
$3P2,3P1,3P0$.\footnotemark

\footnotetext{Hay un poco de duda porque no hay una única
  capa incompleta, pero por analogía nos la jugamos a esa
  distribución. En átomos más complicados, puedes esperar multiplete
  normal y te puede salir en lab invertido o incluso mezclas raras.}

figura 1

\subsection{Moléculas}
Un poquico solo. Aquí perdemos la simetría central. Hay muchos núcleos
que son centros de fuerzas.

El ejemplo más simple es un rebaño de ovejas con moscas (moscas
electrones, lo típico, se mueven rápido y adaptan etc.)\footnote{Donde hay ganado hay mierda, y donde hay mierda
  hay moscas}.

% Peso mosca: 0.19 g
% Peso oveja: 90-300 Lb

La primera aproximación que realizamos será $m_e \ll m_N$, así que la
energía de los electrones es mucho mayor que las de los nucleos.

Estimamos la $E_e$ (electrones), $E_v$ (vibración) y $E_R$ (rotación).
\begin{description}
\item[Energía electrónica] Se tiene, para electrones en una caja,
\begin{equation}
  \begin{split}
    E_e &\sim \frac{\hbar^2}{mR^2} \sim \frac{\hbar^2c^2}{mc^2R^2} \sim
    \frac{(\SI{200}{\MeV\per\femto\metre})^2}{\SI{0.5}{\MeV}
      (\SI{1e5}{\femto\metre})^2} \sim \\
    &\sim \SI{8}{\eV}
  \end{split}
\end{equation}
Está entre el visible y el ultravioleta.
\item[Energía de vibración] Provocada por la vibración de los núcleos.
  Suponemos una $k$ del orden de $kR^2 \sim E_e$, ya que la $E_v$ ha
  de compensar a la $E_e$ para no romperse la molécula o algo así.
\begin{equation}
  \begin{split}
    E_v &\sim \hbar \sqrt{\frac{k}{M}} \sim \frac{\hbar}{\sqrt{M}}
    \frac{\sqrt{E_e}}{R} \sim \left( \frac{m}{M} \right)^{\oh} E_e
    \sim \\
    &\sim 10^{-2}E_e \lesssim \SI{0.1}{\eV}
  \end{split}
\end{equation}
%leq to less or similar
Corresponde al infrarojo cercano.
\item[Energía de rotación] Se tiene
\begin{equation}
  E_R \sim \frac{L^2}{MR^2} \sim \frac{\hbar^2}{MR^2} \sim \left(
    \frac{m}{M} \right)E_e \lesssim \SI{1e-3}{\eV}
\end{equation}
En este caso, las transiciones están en el infrarojo lejano. 

\marginnote{Ahora viene un ladrillo de laplacianas.}
\footnote{Openheimer estaba en el proyecto Manhattan y luego se
  convirtió en un furibundo (no se si había furibundos en los años 60)
  ecologista anti armas nucleares.}
\end{description}


Utilizamos la aproximación de Born-Oppenheimer\footnote{Ver apuntes de
estado sólido}:
\begin{equation}
  \left( \sum_{i=1}^n t_i + \sum_{j=1}^N T_j + V\right) \psi = E \psi
  \label{eq:foobarbaz}
\end{equation}
Con $V$ dependiente de interacción núcleo-núcleo, electrón-electrón
y electrón-núcleo. Las minúsculas son para electrones y las mayúsculas
para núcleos. Suponemos núcleos estáticos en $\mathbf{R}_j$:
\begin{equation}
  \psi(\mathbf{r}_i,\mathbf{R}_j) \sim u_{\mathbf{R}_j}(\mathbf{r}_i)w(\mathbf{R}_j)
\end{equation}
La ecuación de autovalores será
\begin{equation}
  \left[ \sum_{i=1}^n t_i + V \right] u_{\mathbf{R}_j}(\mathbf{r}_i)
  U(\mathbf{R}_j) u_{\mathbf{R}_j(\mathbf{r}_i)}
\end{equation}
Obtenemos para \eqref{eq:foobarbaz}
\begin{equation}
  \left( \sum_{i=1}^n t_i + \sum_{i=1}^N T_j +V \right)
  u_{\mathbf{R}_j(\mathbf{r}_i)} w(\mathbf{R}_j) = E
  u_{\mathbf{R}_j}(\mathbf{r}_i) w(\mathbf{R}_j)
\end{equation}

fuck this.

% me meo la segunda hora






%%% Local Variables:
%%% mode: latex
%%% TeX-master: "../resumen"
%%% End:
