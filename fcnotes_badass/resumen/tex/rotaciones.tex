\chapter{Rotaciones}
En este capítulo se contemplan las rotaciones en $\mathbb{R}^3$
\emph{activas} (que giran el sistema físico y no el marco de
referencia). Son transformaciones lineales $\boldrm{x} \to R\boldrm{x}$ que conservan la norma,
representadas por matrices $3 \times 3$ tales que $\det R = 1$.

Todas las rotaciones en ejes arbitrarios con origen fijo conforman el
\emph{grupo de rotación 3D}, también llamado $SO(3)$. Se puede
parametrizar con un eje de giro y un ángulo o con los tres ángulos de
Euler. $SO(3)$ es un \emph{grupo de Lie}, por lo que se puede ir de
una rotación a otra de manera continua.

Con el convenio de la ``mano derecha'' para los ángulos, podemos ver
las matrices correspondientes a rotaciones en los ejes principales:
\begin{align}
  R_{\text{OX}} &=
  \begin{pmatrix}
    1 & 0 & 0 \\
    0&\cos \theta & -\sin \theta  \\
    0&\sin \theta & \cos \theta  
  \end{pmatrix} \\
  R_{\text{OY}} &=
  \begin{pmatrix}
    \cos \theta &0& \sin \theta  \\
    0 & 1 & 0 \\
    -\sin \theta & 0&\cos \theta  \\
  \end{pmatrix} \\
  R_{\text{OZ}} &=
  \begin{pmatrix}
    \cos \theta & -\sin \theta & 0 \\
    \sin \theta & \cos \theta & 0 \\
    0 & 0 & 1 
  \end{pmatrix}
\end{align}

Investiguemos que ocurre si hacemos una rotación infinitesimal con
ángulo $\varepsilon$. Para ello, tomamos la matriz $R_\text{OZ}$ (más
tarde se extenderá el resultado) y aproximamos $\sin \varepsilon \sim
\varepsilon$ y $\cos \varepsilon \sim 1$:
\begin{equation}
  R_\text{OZ}(\varepsilon) \sim
  \begin{pmatrix}
    1 & -\varepsilon & 0 \\
    \varepsilon &  1 & 0 \\
    0 & 0& 1
  \end{pmatrix}
\end{equation}
La nueva matriz es prácticamente ortogonal, ya que la norma de sus
columnas es $1$ o $1 + \varepsilon^2 \sim 1$. Puede escribirse de forma
especial:
\begin{equation}
  R_\text{OZ}(\varepsilon) = \mathbb{I} - i \varepsilon
  \underbrace{
   \begin{pmatrix}
    0 & -i & 0 \\
    i & 0 & 0 \\
    0 & 0 & 0 
  \end{pmatrix}}_{G_z}
\end{equation}
donde $G_z$ (de ``generador'') es hermítica y ortogonal. Para los
demás ejes tenemos:
\begin{align}
  G_x &=
  \begin{pmatrix}
    0 & 0 & 0 \\
    0&0 & -i  \\
    0&i&0 
  \end{pmatrix} \\
  G_y &=
  \begin{pmatrix}
    0 & 0 & i \\
    0&0 & 0  \\
    -i&0&0 
  \end{pmatrix} \\
  G_z &=
  \begin{pmatrix}
    0 & -i & 0 \\
    i&0 & 0  \\
    0&0&0 
  \end{pmatrix}
\end{align}
Las matrices $G_i$ son los generadores infinitesimales
\index{generador infinitesimal}de rotaciones
en $\mathbb{R}^3$. Se puede ver que $[G_x,G_y]=iG_z$, o de forma
general:
\begin{equation}
  [G_i,G_j] = i\varepsilon_{ijk} G_k
\end{equation}
donde $\varepsilon_{ijk}$ es el \emph{símbolo de Levi-Civita}, evaluado
en $i,j,k = x,y,z$.

Con sumas de estas matrices se puede llegar a cualquier rotación.
Esta álgebra es una \emph{algebra de Lie}. Veamos que, en efecto, es
posible reconstruir una rotación. Rotamos un vector $\boldrm{x}$:
\begin{equation}
  \boldrm{x} \to (\mathbb{I}-i\varepsilon G_z)\boldrm{x} = R_z(\varepsilon)\boldrm{x}
\end{equation}
Como las rotaciones en el mismo eje conmutan entre sí, puedo escribir
\begin{equation}
  \begin{split}
    R_z(\theta+\varepsilon) &= R_z(\varepsilon)R(\theta) = [\mathbb{I} -
    i\varepsilon G_z] R_z(\theta) =\\
    &= R_z(\theta) - i\varepsilon G_z R_z(\theta)
  \end{split}
\end{equation}
Si suponemos $\varepsilon \to 0$ tenemos una ecuación diferencial formal
en $R_z(\theta)$:
\begin{align}
  \frac{R_z(\theta+\varepsilon) - R_z(\theta)}{\varepsilon} &= - i
  G_z R_z(\theta) \\
  \frac{\text{d}}{\text{d}\theta}R_z(\theta) &= -i G_z R_z(\theta)
\end{align}
Es una ecuación para operadores, de solución exponencial:
\begin{equation}
  R_z(\theta) = \exp(-i\theta G_z)\cdot \text{cte.}
\end{equation}
donde $\text{cte.} =1$ ya que $R_z(0) = \mathbb{I}$. Como $G_z$ es
hermítica, la suma exponencial converge sin problemas:
\begin{equation}
  \begin{split}
    e^{-i\theta G_z} &= \mathbb{I}-i \theta G_z -
    \frac{\theta^2}{2!}G_z^2 + i \frac{\theta^3}{3!} G_z^3 + \cdots =
    \\
    &= \mathbb{I} + G_z \left( -i\theta + i \frac{\theta^3}{3!} +
      \cdots \right) + G_z^2 \left( - \frac{\theta^2}{2!} +
      \frac{\theta^4}{4!} + \cdots \right) =\\
    &= \mathbb{I} - i \sin \theta G_z + (\cos\theta -1)G_z^2 = \\
    &= R_{\text{OZ}}
  \end{split}
\end{equation}
donde se ha usado que $G_z^3 = G_z$.

De forma general, para un eje cualquiera $\hat{n}$ se tiene
\begin{equation}
  R_{\hat{n}}(\theta) = e^{-i\theta \boldrm{G}\cdot \hat{n}}
\end{equation}
donde $\boldrm{G} = (G_x,G_y,G_z)$. Notar que $e^{-i\theta
  \boldrm{G}\cdot \hat{n}} $ es factorizable en producto de
exponenciales sólo si los operadores $G_i$ conmutan, y
no es el caso.

\section{Aplicaciones}
Con lo visto, podemos encontrar analíticamente las rotaciones en
diversos espacios vectoriales de la mecánica cuántica.
\subsection{Espacios de Hilbert $ \Ham $}
El espacio de las funciones de onda sin espín es un espacio de
Hilbert. Las rotaciones transforman a las funciones de onda $\varphi$ en
nuevas $\varphi'$. 

La intuición geométrica\index{intuición} nos dice que para una rotación de ángulo
definido la materia se ha movido de $\boldrm{x}$ a $\boldrm{x}'$, y la
función de ondas vale en todos los puntos el valor que había antes en
el ángulo recorrido en sentido contrario. De manera formal:
\begin{equation}
  \varphi'(\boldrm{x}') = \varphi(\boldrm{x}), \ \forall \boldrm{x}' \
  \rightarrow \ \varphi'(\boldrm{x}) = \varphi(R^{-1}\boldrm{x})
\end{equation}
El operador que desplaza a las $\boldrm{x}$ es $R$; nos preguntamos
cuales son las propiedades del endomorfismo $U$ que translada a las
funciones de onda $\varphi$. Damos por descontado\footnote{Cohen no. Ver
allí demostración rigurosa.} que es lineal y unitario. 
Su definición
viene dada de forma implícita por $\varphi'(\boldrm{x}) =
\varphi(R^{-1}\boldrm{x})$; veamos su forma en una rotación
infinitesimal. Comenzamos por calcular $R^{-1}(\varepsilon)$ para el eje
z (de nuevo, posteriormente se verá el caso general):
\begin{equation}
  R^{-1}_z(\varepsilon) \sim
  \begin{pmatrix}
    \cos \theta & \sin \theta & 0 \\
    -\sin \theta & \cos \theta & 0 \\
    0 & 0 & 1 
  \end{pmatrix}_{\theta \to \varepsilon \sim 0} =
  \begin{pmatrix}
    1 & \varepsilon & 0 \\
    -\varepsilon&1 & 0 \\
    0 & 0 & 1
  \end{pmatrix}
\end{equation}
Por tanto,
\begin{equation}
  \varphi'(x,y,z) = \varphi(R^{-1}_z(\varepsilon) \boldrm{x}) =
  \varphi(\boldrm{x}'), \ \ \boldrm{x}' =
  \begin{pmatrix}
    x+\varepsilon y \\
    y-\varepsilon x \\
    z
  \end{pmatrix}
\end{equation}
Como $\varepsilon$ es muy pequeño, efectuamos un desarrollo en serie
de potencias:
\begin{equation}
  \begin{split}
    \varphi'(\boldrm{x}) &= \varphi(\boldrm{x}) + \pdv{\varphi}{x}
    \underbrace{\Delta x}_{\varepsilon y} + \pdv{\varphi}{y}
    \underbrace{\Delta y}_{-\varepsilon x} + \pdv{\varphi}{z}
    \cancelto{0}{\Delta z} = \\
    &= \varphi(\boldrm{x}) - \varepsilon \left[ \pdv{\varphi}{y} x -
      \pdv{\varphi}{x}y \right] = \varphi(\boldrm{x})-\varepsilon i \frac{L_z}{\hbar}\varphi
  \end{split}
\end{equation}
donde $L_z$ es la componente $z$ del operador momento angular.
Concluimos por tanto que
\begin{equation}
  \boxed{
    \varphi'(\boldrm{x}) = \left( \mathbb{I}-i \varepsilon \frac{L_z}{\hbar} \right)\varphi(\boldrm{x})
  }
\end{equation}

Hemos obtenido que $L_z/\hbar = U$ es el generador
infinitesimal\index{generador infinitesimal} de
las rotaciones en el espacio de Hilbert de las funciones de onda sin
espín. De manera similar, en el eje $y$ el operador relevante es
$L_y/\hbar$ y $L_x/\hbar$ en el eje $x$. Tenemos una relación similar
a la de los $G_i$ con los operadores momento angular:
\begin{equation}
  \left[ \frac{L_i}{\hbar}, \frac{L_j}{\hbar} \right] = i
  \varepsilon_{ijk} \frac{L_k}{\hbar}
\end{equation}
Notar que mientras las $G_i$ son matrices, los $L_i$ son operadores
diferenciales. Esto se debe a que los espacios sobre los que actúan
son muy distintos ($\mathbb{R}^3$ y el espacio de Hilbert $ \Ham $).

Con argumentos similares a los usados para $\mathbb{R}^3$, definimos
una rotación finita en el eje $z$ como $U_z(\theta) = e^{-i\theta
  \frac{L_z}{\hbar}}$, y en un eje arbitrario $\hat{n}$ como
\begin{equation}
  U_{\hat{n}} (\theta)= \exp \left(  {-i \theta
      \frac{\boldrm{L}\cdot\hat{n}}{\hbar}} \right)
\end{equation}
Recordar que la exponencial no es factorizable por no conmutar los
$L_i$. En este caso no podemos calcular la exponencial del operador
como serie por no conocer la relación de recurrencia de sus potencias.

\subsection{Espacios complejos $\mathbb{C}^2$}
Los espines de los electrones ($\oh$) pertenecen a
$\mathbb{C}^2$, ya que para definir el espín en cada punto del espacio
son necesarios dos números complejos. Las funciones de onda de estas
partículas necesitan además una extensión a espacio de Hilbert:
\begin{equation}
  \begin{split}
    \varphi &= \binom{f(\boldrm{r})}{g(\boldrm{r})} = f(\boldrm{r})
    \otimes \binom{1}{0} + g(\boldrm{r}) \otimes \binom{0}{1} \\
    &=  f(\boldrm{r})
    \otimes \ket{+} \  + \  g(\boldrm{r}) \otimes \ket{-} \in
     \Ham \otimes \mathbb{C}^2
  \end{split}
\end{equation}
A los \emph{ket} $\ket{+},\ket{-}$ se les suele llamar \emph{espinores}.
Para que $\varphi$ sea de cuadrado integrable, las funciones $f,g$ han de
cumplir que la integral de sus módulos al cuadrado sobre todo el
espacio valgan la unidad.

De manera inocente podría pensarse en que una rotación puede
escribirse como
\begin{equation}
  \binom{f'(\boldrm{r})}{g'(\boldrm{r})} = \binom{f(R^{-1}\boldrm{r})}{g(R^{-1}\boldrm{r})}
\end{equation}
pero no es así, los vectores se transforman de forma más complicada
que los escalares. Necesitamos utilizar los generadores de
$\mathbb{C}^2$, las \emph{matrices de Pauli}:
\begin{equation}
  S_x = \frac{\hbar}{2} \begin{pmatrix} 0 & 1 \\ 1 & 0 \end{pmatrix};  \ \ 
  S_y = \frac{\hbar}{2} \begin{pmatrix} 0 & -i \\ i & 0 \end{pmatrix};  \ \ 
  S_z = \frac{\hbar}{2} \begin{pmatrix} 1 & 0 \\ 0 & -1 \end{pmatrix}
\end{equation}
Las rotaciones en un eje arbitrario vendrán dadas por
\begin{equation}
  R_{\hat{n}}(\theta) = \exp \left( -i \theta
    \frac{\boldrm{S}\cdot\hat{n}}{\hbar} \right) \in \mathbb{C}^2
\end{equation}
Resolviendo la serie de potencias se obtiene
\begin{equation}
  R_{\hat{n}}(\theta) = \mathbb{I}\cos \frac{\theta}{2} - i
  (\boldrm{\sigma}\cdot \hat{n}) \sin \frac{\theta}{2}
\end{equation}
donde $\boldrm{\sigma} = \frac{2}{\hbar}\boldrm{S}$. Un espinor $\chi$
``apunta'' en la dirección $\hat{n}$ si
$(\boldrm{\sigma}\cdot\hat{n})\chi = +1 \chi$

Puede comprobarse que, por ejemplo, un giro en $z$ no varía el valor
del espín en $z$. No obstante, aunque podría esperarse que el operador
$R_{\hat{n}}(2\pi) = \mathbb{I}$, resulta que lo que se obtiene es
$-\mathbb{I}$. Notar que esto es irrelevante, ya que lo que se mide es
siempre $\bra{\varphi|A}\ket{\varphi}$ o $|\varphi|^2$, que son siempre positivos.

\subsection{Espacios de partículas con espín $ \Ham \otimes \mathbb{C}^2$}
Resueltos los giros en $\mathbb{C}^2$ y en los espacios de Hilbert, ya
solo queda extender los resultados al producto de ambos espacios. Como
es razonable pensar, basta con girar primero la órbita y luego el
espín. De manera formal, el efecto del operador $U$ sobre el espinor
con función de onda $\varphi(\boldrm{r})\otimes \chi$ es:
\begin{equation}
  \left( e^{-i\theta \frac{1}{\hbar}\boldrm{L}\hat{n}} \otimes
    e^{-i\theta \frac{1}{\hbar}\boldrm{S}\hat{n}}\right)
  [\varphi(\boldrm{r})\otimes \chi] = \left( e^{-i\theta
      \frac{1}{\hbar}\boldrm{L}\hat{n}}\varphi(\boldrm{r}) \right)
  \otimes \left( e^{-i\theta \frac{1}{\hbar}\boldrm{S}\hat{n}}\chi \right)
\end{equation}
Realicemos una rotación infinitesimal, para ver la forma de los
generadores. Utilizamos $\hat{n}=z$; sea $\theta \to \varepsilon\sim 0$:
\begin{equation}
  \begin{split}
  U [\varphi(\boldrm{r})\otimes \chi] &= \left( e^{-i\theta
      \frac{1}{\hbar}\boldrm{L}\hat{n}}\varphi(\boldrm{r}) \right)
  \otimes \left( e^{-i\theta \frac{1}{\hbar}\boldrm{S}\hat{n}}\chi
  \right)= \\
  &= \left( \mathbb{I}-i\varepsilon \frac{L_z}{\hbar} \right)\varphi \otimes
  \left( \mathbb{I}-i\varepsilon \frac{S_z}{\hbar} \right) \chi = \\
  &= \varphi\otimes \chi - i\varepsilon \left( \frac{L_z}{\hbar}\varphi \right)\otimes \chi  - i \varepsilon
  \left( \frac{S_z}{\hbar}\chi \right)\otimes \varphi  +
  \order{\varepsilon^2} \cdots
  \end{split}
\end{equation}

Los operadores del segundo y tercer sumando se pueden reescribir como
extensiones al espacio producto vectorial; para el momento angular
tenemos $L_z \varphi \otimes \chi = (L_z\otimes \mathbb{I})[\varphi \otimes
\chi]$ y para la matriz de Pauli $ \varphi \otimes S_z\chi = ( \mathbb{I} \otimes
S_z)[\varphi \otimes \chi]$. Con ello, obtenemos
\begin{equation}
  =\varphi \otimes \chi - i \frac{\varepsilon}{\hbar} (L_z\otimes \mathbb{I}+
  \mathbb{I}\otimes S_z)
  [\varphi\otimes\chi] + \order{\varepsilon^2} = \cdots
\end{equation}
que con un una notación más laxa\footnote{No hay que dejar que esta
  notación nos haga olvidar que dos operadores de espacios vectoriales
distintos no se pueden sumar como si tal cosa. Por ejemplo, $S_{1z}+S_{2z}$
no es $\frac{\hbar}{2}\begin{pmatrix} 1 & 0 \\ 0 & -1 \end{pmatrix} +
\frac{\hbar}{2}\begin{pmatrix} 1 & 0 \\ 0 & -1 \end{pmatrix}  =
\hbar \begin{pmatrix} 1 & 0 \\ 0 & -1 \end{pmatrix} $, sino la matriz
$4\times 4$ que corresponde a $S_{1z}\otimes \mathbb{I} +
\mathbb{I}\otimes S_{2z} = \begin{pmatrix} (S_{1z}) & 0 \\ 0 & (S_{2z}) \end{pmatrix}$ } podemos escribir como
\begin{equation}
  =\varphi \otimes \chi - i \frac{\varepsilon}{\hbar} (L_z+
  S_z)
  [\varphi\otimes\chi] + \order{\varepsilon^2} = \cdots
\end{equation}
Teniendo en cuenta que $L_z + S_z = J_z$ (operador \emph{momento
  angular total}), obtenemos
\begin{equation}
  \cdots = \left( \mathbb{I} - i \varepsilon \frac{J_z}{\hbar}
  \right)[\varphi\otimes \chi]
\end{equation} 
Obtenemos que el momento angular total es el generador de las
rotaciones en el espacio $ \Ham \otimes \mathbb{C}^2$. En
un espacio general se utiliza $\boldrm{J}_T = \sum_{ }\boldrm{J}_i +
\sum_{ }\boldrm{S}_i$. Notar de nuevo que $\boldrm{J}$ y $\boldrm{S}$
pertenecen a espacios distintos y no son sumables; en realidad nos
referimos a sus extensiones a $ \Ham \otimes \mathbb{C}^2$. Las
rotaciones en un eje arbitrario vendrán dadas por
\begin{equation}
R_{\hat{n}} (\theta) = \exp \left( -i \varepsilon \frac{\boldrm{J}_T\cdot
    \hat{n}}{\hbar} \right) = e^{-i\theta \frac{\boldrm{L}_1 \cdot
    \hat{n}}{\hbar}} e^{-i\theta \frac{\boldrm{S}_1 \cdot \hat{n}}{\hbar}}
\cdots e^{-i\theta \frac{\boldrm{L}_N \cdot \hat{n}}{\hbar}} e^{-i\theta \frac{\boldrm{S}_N \cdot \hat{n}}{\hbar}}
\end{equation}
donde $N$ es el número de partículas del sistema. Cada operador actúa
sobre sus funciones de onda correspondientes, recordar que en esta
notación faltan muchas extensiones (varios factores ``$\otimes \mathbb{I}$'' ).

\section{Rotación de observables}
Imaginemos un montaje experimental como el de Stern-Gerlach; hacemos
pasar electrones (espín $\oh$) por un campo magnetico
$\boldrm{B}\propto \hat{z}$
fuerte y vemos como el haz se divide en dos, colisionando con una
pantalla fluorescente. Estamos midiendo el
autovalor del observable $S_z$, y se obtienen de manera equiprobable
los valores $\pm \nicefrac{\hbar}{2}$. Si polarizamos los espines
sólo se obtendrá una mancha, por ejemplo la del autovalor $\nicefrac{\hbar}{2}$.

Si rotamos todo el sistema de forma que el campo magnético esté
orientado ahora en $\hat{x}$ estaremos midiendo el observable $S_x$,
pero seguiremos obteniendo el autovalor $\nicefrac{\hbar}{2}$ (la
mancha seguirá en el mismo sitio de la pantalla).

Deducimos por tanto que rotar un sistema
físico\jokenote{Es nuestro hecho diferencial el no
girar el sistema de coordenadas. Nos hace diferentes, es
decir, nos hace mucho mejores que los demás.} cambia el
obserbable
pero no el autovalor. Sea un observable $A$ de autovalores $a_n$ y
autovectores $u_n$, si efectuamos una rotación en el espacio de
Hilbert $u_n \to Ru_n = u_n'$ obtenemos
\begin{equation}
  A u_n = a_n u_n \ \stackrel{R}{\rightarrow} \ A'u_n' = a_n u_n'
\end{equation}
Nos preguntamos la relación entre el nuevo operador y el original.
Para averiguarlo, utilizamos que $u' = Ru$:
\begin{equation}
  \begin{split}
    A' u'_n &= a_n u_n' \\
    A' Ru_n &= a_n Ru_n \\
    A' Ru_n &= R a_n u_n \\
    R^{-1}A' Ru_n &=  a_n u_n  \\
    R^{-1}A' Ru_n &=  A u_n  \\
  \end{split}
\end{equation}
Por lo tanto
\begin{equation}
  \boxed{A' = R A R^\dagger}
\end{equation}
donde se ha utilizado que para matrices hermíticas $R^{-1}MR=RMR^\dagger$.

\subsection{Observables escalares e invariancia bajo rotaciones}
Un ejemplo es $\boldrm{S}^2 = \frac{3}{4} \hbar \mathbb{I}$.
Supongamos ahora que $R$ es infinitesimal; en tal caso:
\begin{equation}
  R = \left( \mathbb{I} - \frac{i\varepsilon}{\hbar}\boldrm{J}\cdot\hat{n} \right)
\end{equation}
y por lo tanto
\begin{equation}
  \begin{split}
    A' &= \left( \mathbb{I} - \frac{i\varepsilon}{\hbar} \boldrm{J}\cdot
      \hat{n} \right) A \left( \mathbb{I} + \frac{i\varepsilon}{\hbar}
      \boldrm{J}\cdot \hat{n} \right) = \\
    &= A - \frac{i\varepsilon}{\hbar}(\boldrm{J}\cdot\hat{n})A + A
    \frac{i\varepsilon}{\hbar}(\boldrm{J}\cdot\hat{n}) +
    \order{\varepsilon^2} = \\
    &= A - \frac{i\varepsilon}{\hbar} [\boldrm{J}\hat{n},A]
  \end{split}
\end{equation}
Si $A$ conmuta con cualquier función de $J$ se obtiene $A'= A$. Esta
relación de conmutación con $\boldrm{J}$ es muy relevante cuando $A$
es el hamiltoniano, se le denota \emph{invariancia bajo rotaciones}.

Cuando un sistema es invariante bajo rotaciones para un observable es
señal de que este es escalar, evitando romperse la simetría cuando
rotamos el sistema. 

Imaginemos una partícula libre en el plano, si rotamos el sistema un
ángulo $\theta$ el movimiento seguirá siendo el mismo salvo por la
rotación. Si en cambio rotamos un sistema con una fuerza vectorial,
como un tiro parabólico en un campo gravitatorio, la trayectoria ya no
será la misma aunque rotemos las condiciones iniciales; habría que
rotar también la fuerza externa.

De manera formal, en los sistemas con invariancia entre rotaciones
esperamos la siguiente regla de transformación:
\begin{equation*}
  \begin{matrix}
    \psi(t_0) & \stackrel{ \Ham }{\rightarrow} & \psi(t) \\
    \downarrow R & & \downarrow R \\
    \psi'(t_0) & \stackrel{ \Ham }{\rightarrow} & \psi'(t)
  \end{matrix}
\end{equation*}
Notar que los dos operadores $R$ son el mismo.

Si hay invariancia ante rotaciones, utilizando la ecuación de
Schrödinger ($i \hbar \dot{\psi} =  \Ham \psi$) es esperable que
\begin{equation}
  \begin{split}
    \psi'(t_0 + \dd{t}) &= \psi'(t_0) - \frac{i}{\hbar} \dd{t}
     \Ham  \psi'(t_0) \\
    \psi(t_0 + \dd{t}) &= \psi(t_0) - \frac{i}{\hbar} \dd{t}  \Ham  \psi(t_0)
  \end{split}
\end{equation}
como $\psi'(t_0 + \dd{t}) = R \psi(t_0 + \dd{t})$, sustituyendo
obtenemos
\begin{equation}
  \psi'(t_0)-\frac{i}{\hbar}\dd{t} \Ham \psi'(t_0) = R \psi (t_0)
  - \frac{i}{\hbar}\dd{t}R  \Ham  \psi(t_0)
\end{equation}
y utilizando que $\psi'(t_0) = R \psi(t_0)$ obtenemos que
\begin{equation}
   \Ham R\psi = R  \Ham  \psi, \ \forall \psi,t_0,R
\end{equation}
Por lo tanto $[ \Ham ,R] = 0$. Hemos obtenido que el hamiltoniano
es un operador escalar bajo rotaciones. De igual forma conmuta con
$\boldrm{J}$ y $\boldrm{L}$. Una consecuencia inmediata es que sus
autovalores están degenerados; a esta degeneración se le denota
\emph{degeneración natural}. 

\begin{proof}[Degeneración de la energía]
  Sean $E$ y $\ket{\alpha,j,m}$ un autovalor y autovector del
  hamiltoniano. Aplicamos en la ecuación de Schrödinger independiente
  del tiempo un operador escalera al autovector:
  \begin{equation}
     \Ham  J_\pm \ket{\alpha,j,m} = \sqrt{j(j+1) - m(m\pm 1)}
    \hbar  \Ham  \ket{\alpha,j,m\pm1}
  \end{equation}
  Como $ \Ham $ conmuta con $\boldrm{J}$, también conmuta con
  cualquier combinación lineal de sus componentes. Como los operadores
  escalera se construyen como combinación lineal de los componentes de
  $\boldrm{J}$, el hamiltoniano conmuta con $J_\pm$ y podemos escribir
  que $ \Ham J_\pm = J_\pm  \Ham $. Por tanto,
  \begin{equation}
    \begin{split}
      J_\pm \Ham  \ket{\alpha,j,m} &= J_\pm E \ket{\alpha,j,m} =
      \\ &= E
      \sqrt{j(j+1) - m(m\pm 1)} \hbar \ket{\alpha,j,m\pm1} = \\
      &= \sqrt{j(j+1) - m(m\pm 1)}
    \hbar  \Ham  \ket{\alpha,j,m\pm1} =\\ &=  \Ham J_\pm \ket{\alpha,j,m}
    \end{split}
  \end{equation}
  obtenemos que $ \Ham \ket{\alpha,j,m\pm1} = E
  \ket{\alpha,j,m\pm1}$. Por tanto hay degeneración para una energía
  $E$ fija, como mínimo $2j+1$ (número de estados $m$).
\end{proof}

\subsection{Ejemplo: potenciales centrales}
Sea una partícula de masa $m$ en un potencial central $V(r)$. El
hamiltoniano\jokenote{Y os preguntaréis, ¿por qué
quiero medir eso? Pues porque sois físicos, no filósofos. Si
no queréis medir nada os vais a la facultad de enfrente, que
os acogerán con los brazos abiertos porque les faltan
alumnos.} es:
\begin{equation}
   \Ham  = \frac{-\hbar^2}{2m}\nabla^2 + V(r)
\end{equation}
Claramente conmuta con rotaciones respecto al centro de fuerzas por
simetría. Tenemos que $[ \Ham ,\boldrm{L}] = 0$, y por tanto
$ \Ham ,L^2,L_z$ comparten autovalores.

La simetría del problema nos permite escribir el hamiltoniano de
forma puramente radial, encerrando la dependencia angular en $\boldrm{L}^2$:
\begin{equation}
   \Ham  = \frac{-\hbar^2}{2m} \left[ \frac{1}{r^2} \pdv{}{r}
    \left( r^2 \pdv{}{r} \right) - \frac{\boldrm{L}^2}{\hbar^2 r^2} \right] + V(r)
\end{equation}
Podemos expresar sus soluciones como producto de una parte radial y
otra angular:
\begin{equation}
  \psi = R(r) Y_\ell^m(\theta,\varphi)
\end{equation}
Si sustituyo $\psi$ en la ecuación de Schrödinger independiente del
tiempo, se pueden cancelar de ambos lados de la ecuación los armónicos
esféricos, y se obtiene:
\begin{equation}
  \frac{-\hbar^2}{2m} \dv[2]{}{r} \chi_\ell(r) + \underbrace{\left[ V(r) +
    \frac{\hbar^2}{2mr^2} \ell (\ell+1) \right]}_{V_\text{eff}} \chi_\ell(r) = E \chi_\ell(r)
\end{equation}
donde hay una solución para cada $\ell$, por lo que hay que indexar las
$\chi$. Por último, imponemos condiciones de contorno:
\begin{equation}
  \int_{0}^\infty |R(r)|^2 r^2 \dd{r} = \int_{4\pi}
  |Y_\ell^m(\theta,\varphi)|^2 \dd{\Omega} = 1
\end{equation}
Si nos dan un autovalor podemos conocer el autoestado correspondiente,
que está degenerado con los estados de igual energía:
\begin{equation}
  R_\ell(r)Y_\ell^m(\theta,\varphi), \ m = 2\ell+1
\end{equation}
Notar que aunque es una ecuación de segundo orden sólo se ha hablado
de una de las soluciones. Esto es debido a que la otra no es
normalizable por poseer una divergencia en $r=0$.

%%% Local Variables:
%%% mode: latex
%%% TeX-master: "../resumen"
%%% End:
